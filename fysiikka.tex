\part{Fysiikka}

\section{Suureita ja yksiköitä}

\begin{table}[ht!]
\centering
\caption{\href{https://en.wikipedia.org/wiki/SI_base_unit}{SI-järjestelmän perusyksiköt} \cite[s. 64]{MAOL} }
\begin{tabular}{l|>{$} l <{$}|l|l}
Suure	& \text{Tunnus}	& Yksikkö	& Yksikön tunnus \\
\hline
pituus		& l, s	& metri			& m \\
massa		& m		& kilogramma	& kg \\
aika		& t		& sekunti		& s \\
sähkövirta	& I		& ampeeri		& A \\
lämpötila	& T		& kelvin		& K \\
ainemäärä	& n		& mooli			& mol \\
valovoima	& I		& kandela		& cd \\
\end{tabular}
\end{table}


\begin{table}[ht!]
\centering
\setlength{\extrarowheight}{2pt}
\caption{\href{https://en.wikipedia.org/wiki/Metric_prefix}{Kerrannaisyksiköiden etuliitteet} \cite[s. 65]{MAOL} }
\begin{tabu} to \textwidth {X|X| >{$} X <{$} |X|X| >{$} X <{$} }
Nimi	& Tunnus	& \text{Kerroin}	& Nimi	& Tunnus	& \text{Kerroin} \\
\hline
deka	& da	& 10^1		& desi		& d		& 10^{-1} \\
hehto	& h		& 10^2		& sentti	& c		& 10^{-2} \\
kilo	& k		& 10^3		& milli		& m		& 10^{-3} \\
mega	& M		& 10^6		& mikro		& $\mu$	& 10^{-6} \\
giga	& G		& 10^9		& nano		& n		& 10^{-9} \\
tera	& T		& 10^{12}	& piko		& p		& 10^{-12} \\
peta	& P		& 10^{15}	& femto		& f		& 10^{-15} \\
eksa	& E		& 10^{18}	& atto		& a		& 10^{-18} \\
tsetta	& Z		& 10^{21}	& tsepto	& z		& 10^{-21} \\
jotta	& Y 	& 10^{24}	& jokto		& y		& 10^{-24} \\
\end{tabu}
\end{table}


\begin{table}[ht!]
\centering
\setlength{\extrarowheight}{5pt}
\caption{\href{https://en.wikipedia.org/wiki/SI_base_unit}{Perusyksiköiden määritelmät} \cite[s. 64-65]{MAOL} }
\begin{tabu} to \textwidth {lX}
\hline
metri (m)		& Metri on sellaisen matkan pituus, jonka valo kulkee tyhjiössä aikavälissä 1/299 792 458 sekuntia. \newline (1983, 17. CGPM) \\

kilogramma (kg)	& Kilogramma on yhtä suuri kuin kansainvälisen kilogramman prototyypin massa. \newline (1889 ja 1901, 1. ja 3. CGPM) \\

sekunti (s)		& Sekunti on 9 192 631 770 kertaa sellaisen säteilyn jaksonaika, joka vastaa cesium-133-atomin siirtymää perustilan ylihienorakenteen kahden energiatason välillä. \newline (1967, 13. CGPM) \\

ampeeri (A)		& Ampeeri on sellainen ajallisesti muuttumaton sähkövirta, joka kulkiessaan kahdessa suorassa yhdensuuntaisessa, äärettömän pitkässä ja ohuessa johtimessa, joiden poikkileikkaus on ympyrä ja jotka ovat 1 metrin etäisyydellä toisistaan tyhjiössä, saa aikaan johtimien välille $2 \cdot 10^{-7}$ newtonin voiman johtimen metriä kohti. \newline (1948, 9. CGPM) \\

kelvin (K)		& Kelvin, termodynaamisen lämpötilan yksikkö, on 1/273,16 veden kolmoispisteen termodynaamisesta lämpötilasta. \newline (1967, 13. CGPM) \\

mooli (mol)		& Mooli on sellaisen systeemin ainemäärä, joka sisältää yhtä monta keskenään samanlaista perusosasta kuin on atomeja 0,012 kilogrammassa hiili-12:ta. Perusosaset voivat olla atomeja, molekyylejä, ioneja, elektroneja, muita hiukkasia tai sellaisten hiukkasten määriteltyjä ryhmiä. \newline (1971, 14. CGPM) \\

kandela (cd)	& Kandela on sellaisen säteilijän valovoima, joka lähettää tiettyyn suuntaan monokromaattista $540 \cdot 10^{12}$ hertsin taajuista säteilyä ja jonka säteilyintensiteetti tähän suuntaan on 1/683 wattia steradiaania kohti. \newline (1979, 16. CGPM) \\
\end{tabu}

\end{table}


\clearpage

\section{Vakioita}

\begin{consttable}{ \href{http://physics.nist.gov/cgi-bin/cuu/Category?view=pdf&Universal.x=95&Universal.y=7}{Universaalit luonnonvakiot} \cite{NIST-Constants} }
valon nopeus tyhjiössä		& c, c_0		& 2,9979 2458 \cdot 10^8	& m/s \\
tyhjiön permeabiliteetti \newline (magneettivakio)	& \mu_0	& 4 \pi \cdot 10^{-7} \newline 12,566 \ 370 \ 614 \ldots \cdot 10^{-7}	& N/A$^2$ \\
tyhjiön permittiivisyys	\newline (sähkövakio)	& \epsilon_0 = \frac{1}{\mu_0 c^2}	& 8,854 \ 187 \ 817 \ldots \cdot 10^{-12}	& F/m \\
% tyhjiön karakteristinen impedanssi \\
Newt. gravitaatiovakio	& G	& 6,674 \ 08 \cdot 10^{-11}	& N m$^2$/kg$^2$ \\
Planckin vakio	& h		& 6,626 \ 070 \ 040 \cdot 10^{-34}	\newline 4,135 \ 667 \ 662 \cdot 10^{-15} & J s \newline eV s \\
Diracin vakio	& \hbar	= \frac{h}{2 \pi} & 1,054 \ 571 \ 800 \cdot 10^{-34} \newline 6,582 \ 119 \ 514 \cdot 10^{-16}	& J s \newline eV s \\
\hline
Planckin massa & m_P = \sqrt{\frac{\hbar c}{G}}	& 2,176 \ 470 \cdot 10^{-8}	& kg \\
Planckin massanergia	& m_P c^2	& 1,220 \ 910 \cdot 10^{19}	& GeV \\
Planckin lämpötila		& T_P = \frac{1}{k} \sqrt{\frac{\hbar c^5}{G}}	& 1,416 \ 808 \cdot 10^{32}	& K \\
Planckin pituus			& l_P = \frac{\hbar}{m_P c} = \sqrt{\frac{\hbar G}{c^3}}	& 1,616 \ 229 \cdot 10^{-35}	& m \\
Planckin aika			& t_P = \frac{l_P}{c} = \sqrt{\frac{\hbar G}{c^5}}	& 5,391 \ 16 \cdot 10^{-44}	& s \\
\end{consttable}

\begin{consttable}{ \href{http://physics.nist.gov/cgi-bin/cuu/Category?view=pdf&Electromagnetic.x=76&Electromagnetic.y=17}{Sähkömagneettisia vakioita} \cite{NIST-Constants} , \cite[s. 70]{MAOL} }
alkeisvaraus	& e	& 1,602 \ 176 \ 6208 \cdot 10^{-19}	& C \\
				& \frac{e}{h}	& 2,417 \ 989 \ 262 \cdot 10^{14}	& A/J \\
Coulombin lain vakio	& k = \frac{1}{4 \pi \epsilon_0}	& 8,987 \ 551 \ 484 \cdot 10^9	& Nm$^2$ / C$^2$ \\ % \cite[s. 70]{MAOL}
Bohrin magnetoni	& \mu_B = \frac{e \hbar}{2m_e}	& 927,400 \ 9994 \cdot 10^{-26}	& J/T \\
\end{consttable}

\begin{consttable}{ \href{http://physics.nist.gov/cgi-bin/cuu/Category?view=pdf&Atomic+and+nuclear.x=47&Atomic+and+nuclear.y=15}{Atomi- ja ydinfysiikan vakioita} \cite{NIST-Constants} }
hienorakennevakio	& \alpha = \frac{e^2}{4 \pi \epsilon_0 \hbar c}	& 7,297 \ 352 \ 5664 \cdot 10^{-3} & \\
					& \alpha^{-1}	& 137,035 \ 999 \ 139 & \\
Rydbergin vakio		& R_\infty = \frac{\alpha^2 m_e c}{2h}	& 1,097 \ 373 \ 156 \ 8508 \cdot 10^7	& 1/m \\
					& R_\infty c	& 3,289 \ 841 \ 960 \ 355 \cdot 10^{15}	& Hz \\
					& R_\infty hc	& 2,179 \ 872 \ 325 \cdot 10^{-18} \newline 13,605 \ 693 \ 009	& J \newline eV \\
Bohrin säde		& a_0 = \frac{\alpha}{4 \pi R_\infty} = \frac{4 \pi \epsilon_0 \hbar^2}{m_e e^2} & 0,529 \ 177 \ 210 \ 67 \cdot 10^{-10}	& m \\
\end{consttable}

\begin{consttable}{ \href{http://physics.nist.gov/cgi-bin/cuu/Category?view=pdf&Atomic+and+nuclear.x=47&Atomic+and+nuclear.y=15}{Hiukkasten lepomassoja ja massaenergioita} \cite{NIST-Constants} }
elektronin lepomassa	& m_e	& 9,109 \ 383 56 \cdot 10^{-31}	\newline 5,485 \ 799 \ 090 \ 70 \cdot 10^{-4} & kg \newline u \\
					& m_e c^2	& 8,187 \ 105 \ 65 \cdot 10^{-14} \newline 0,510 \ 998 \ 9461 & J \newline MeV \\
protonin lepomassa	& m_p	& 1,672 \ 621 \ 898 \cdot 10^{-27} \newline 1,007 276 466 879	& kg \newline u \\
					& m_p c^2	& 1,503 \ 277 \ 593 \cdot 10^{-10} \newline 938,272 \ 0813	& J \newline MeV \\
neutronin lepomassa	& m_n	& 1,674 \ 927 \ 471 \cdot 10^{-27} \newline 1,008 \ 664 \ 915 \ 88	& kg \newline u \\
					& m_n c^2	& 1,505 \ 349 \ 739 \cdot 10^{-10} \newline 939,565 \ 4133	& J \newline MeV \\
deuteronin lepomassa	& m_d	& 3,343 \ 583 \ 719 \cdot 10^{-27} \newline 2,013 \ 553 \ 212 \ 745	& kg \newline u \\
						& m_d c^2	& 3,005 \ 063 \ 183 \cdot 10^{-10} \newline 1875,612 \ 928	& J \newline MeV \\
alfahiukkasen lepomassa	& m_\alpha	& 6,655 \ 657 \ 230 \cdot 10^{-27} \newline 4,001 \ 506 \ 179 \ 127	& kg \newline u \\
						& m_\alpha c^2	& 5,971 \ 920 \ 097 \cdot 10^{-10} \newline 3727,379 \ 378	& J \newline MeV \\
\end{consttable}

\begin{consttable}{ \href{http://physics.nist.gov/cgi-bin/cuu/Category?view=pdf&Physico-chemical.x=69&Physico-chemical.y=14}{Fysiokemialliset vakiot} \cite{NIST-Constants} }
Avogadron luku	& N_A, L	& 6,022 \ 140 \ 857 \cdot 10^{23}	& 1/mol \\
Faradayn vakio	& F = N_A e	& 96 \ 485,332 89	& C/mol \\
moolinen Planckin vakio	& N_A h	& 3,990 312 7110 \cdot 10^{-10}	& J s mol$^{-1}$ \\
moolinen kaasuvakio		& R	& 8,314 \ 4598	& J mol$^{-1}$ K$^{-1}$ \\
Boltzmannin vakio	& k_B = \frac{R}{N_A}	& 1,380 \ 648 \ 52 \cdot 10^{-23}	& J/K \\
Stefanin-Boltzmannin vakio	& \sigma = \frac{\pi^2 k_B^4}{60 \hbar^3 c^2}	& 5,670 367 \cdot 10^{-8}	& W m$^{-2}$ K$^{-4}$ \\
Wienin siirtymälain vakio	& b = \lambda_{max} T	& 2,897 \ 7729 \cdot 10^{-3}	& m K \\
\end{consttable}

\clearpage

\subsection{Muuntokertoimia}

\begin{table}[ht!]
\centering
\setlength{\extrarowheight}{2pt}
\caption{Muuntokertoimia, osa 1 \cite{MAOL}, \cite{IAU-Measuring} }
\begin{tabu} to \textwidth {l|lX}
\hline
pituus	& 1""	& = 1 in = tuuma = 25,40 mm \\
		& 1'	& = 1 ft = 1 jalka = 0,3048 m \\
		& 1 mi	& = 1 maili = 1,609 344 km \\
		& 1 Å	& = 1 ångström = $10^{-10}$ m = 0,1 nm \\
		& 1 mpk	& = 1 M = 1 meripeninkulma = 1 852 m \\
		% Light-year is defined using the julian year, not the gregorian one. MAOL has a wrong value. During the TUT astrophysics course (FYS-1670) of 2016, the lecturer Antti Vuorimäki recommended students to replace the wrong value from their table books with the correct one.
		& 1 ly	& = 1 valovuosi = 9,460 730 472 5808 Pm\\ % \cite{IAU-Measuring}
		& 1 pc	& = 1 parsek = 30,85678 Pm \\
		& 1 au	& = 1 tähtitieteellinen yksikkö = 149,597 870 700 Gm \\ % \cite{IAU-Measuring}
\hline
massa	& 1 ka	& = 1 karaatti = 200 mg \\
		& 1 u	& = 1,660 5389 $\cdot 10^{-27}$ kg \\
		& 1 lb	& = 1 naula = 0,4536 kg \\
		& 1 oz	& = 1 unssi = 28,35 kg \\
\hline
pituusmassa	& 1 denier	& = 0,111 mg/m \\
			& 1 tex		& = 1 mg/m \\
\hline
aika	& 1 d	& = 86400 s \\
		& 1 a	& = 31,536 $\cdot 10^6$ s \\
\hline
tasokulma	& 1\degree	& = $2 \pi /360$ rad \\
			& 1'		& = 1 kulmaminuutti = 1/60\degree \\
			& 1''		& = 1 kulmasekunti = 1/3600\degree \\
			& 1 h		& = 1 tunti = $2 \pi / 24$ rad \\
\hline
pinta-ala	& 1 b		& = 1 barni = $10^{-28}$ m$^2$ \\
			& 1 acre	& = 1 eekkeri = 4,0469$\cdot 10^3$ m$^2$ \\
\hline
tilavuus	& 1 l		& = 1 dm$^3$ \\
			& 1 bbl		& = 1 barreli = 0,158 9873 m$^3$ \\
			& 1 gal		& = 1 gallona (UK) = 4,546 092 l \\
			& 1 gal		& = 1 gallona (US) = 3,785 412 l \\
			& 1 rt		& = 1 rekisteritonni = 2,831 685 m$^3$ \\
			& 1 std		& = 1 standartti = 4,672 280 m$^3$ \\
\end{tabu}
\end{table}


\begin{table}[ht!]
\centering
\setlength{\extrarowheight}{2pt}
\caption{Muuntokertoimia, osa 2 \cite{MAOL}}
\begin{tabu} to \textwidth{l|lX}
\hline
nopeus		& 1 solmu	& = 1 mpk/h = 1,852 km/h = 0,5144 m/s \\
			& 1 mach	& = 1 M = 311 m/s \\
\hline
voima		& 1 kp		& = 9,806 65 N \\
			& 1 dyn		& = 10 $\mu$N \\
\hline
työ, energia	& 1 erg	& = 0,1 $\mu$J \\
			& 1 eV		& = 1,602 1773$\cdot 10^{-19}$ J \\
			& 1 cal		& = 4,1868 J \\
			& 1 kWh		& = 3,6 MJ \\
			& 1 kpm		& = 9,806 65 J \\
\hline
\multirow{2}{3.2cm}{massan ja energian vastaavuus}
	& 1 u	& = 149,241 91 pJ/c$^2$ = 931,494 32 MeV/c$^2$ \\
	& 1 kg	& = 89,875 517 PJ/c$^2$ \\
\hline
teho	& 1 kpm/s	& = 9,806 65 W \\
		& 1 hv		& = 735,5 W \\
		& 1 kcal/h	& = 1,163 W \\
\hline
paine	& 1 bar		& = 10$^5$ Pa \\
		& 1 at		& = 1 kp/cm$^2$ = 98,0665 kPa \\
		& 1 atm		& = 760 torr = 101,325 kPa \\
		& 1 mmH$_2$O	& = 9,806 65 Pa \\
		& 1 psi		& = 1 lbf/in$^2$ = 6,894 757$\cdot 10^3$ Pa \\
\hline
viskositeetti	& 1 P	& = 1 poisi = 0,1 Ns/m$^2$ \\
\hline
taittovoimakkuus	& 1 d	& = 1 diptri = 1/m \\
\hline
magneettivuo		& 1 Mx	& = 1 maxwell = 10 nWb \\
\hline
\multirow{2}{3cm}{Magneettivuon \newline tiheys}
	& 1 G	& = 1 gauss = 0,1 mT \\
	\\
\hline
\multirow{2}{3cm}{Magneettikentän \newline voimakkuus}
	& 1 Oe	& = 1 örsted = 10$^3 / 4 \pi$ A/m = 79,577 A/m \\
	\\
\hline
\multirow{2}{3.2cm}{Aktiivisuus (ionisoiva säteily)}
	& 1 Ci	& = 1 curie = $3,7 \cdot 10^10$ 1/s = 37 GBq \\
	\\
\hline
säteilytys	& 1 R	& = 1 röntgen = 0,258 mC/kg (ilmassa) \\
\hline
absorboitunut annos	& 1 rd	& = 1 rad = 0,01 J/kg = 0,01 gray = 10 mGy \\
\hline
ekvivalenttiannos	& 1 rem	& = 0,01 J/kg = 0,01 sievert = 10 mSv \\
\end{tabu}
\end{table}

\clearpage

\section{Kaavoja}

\subsection{Mekaniikka}



\begin{eqtable-units}{Etenemisiike}
\textbf{Etenemisliike} &&& \\
matka	&	x, s, r	& m		& x = vt \\
nopeus	&	v	& m/s	& v = \frac{dx}{dt} \\
kiihtyvyys	&	a	& m/s$^2$	& a = \frac{dv}{dt} = \frac{d^2x}{dt^2} \\
liikemäärä	& p	& kgm/s	& p = mv \\
Newton II	&&& \frac{d \bm{p}}{dt} = \bm{F} \Leftrightarrow \bm{a} = \frac{\bm{F}}{m} \\
työ			& W	& J	& W = \int \bm{F} \cdot \bm{dx} \\
kineettinen energia	& E_k, K	& J	& E_k = \frac{1}{2}mv^2 \\
potentiaalienergia	& E_p, U	& J	& E_p = \frac{1}{2}kx^2 \\
\hline
\textbf{Tasaisesti muuttuva etenemisliike} &&& \\
loppunopeus	& v	& m/s	& v = v_0 + at \\
paikka		& x	& m		& x = x_0 + v_0 t + \frac{1}{2} at^2 \\
\end{eqtable-units}

\begin{eqtable-units}{Pyörimisliike}
\textbf{Pyörimisliike} &&& \\
kaaren pituus	& s			& m		& s = r \theta \\
%								&&& \Delta \bm{s} = \Delta \bm{\theta} \times \bm{r} \\
kulmanopeus		& \omega	& rad/s	& \bm{\omega} = \frac{d \bm{\theta}}{dt} \\
kulmakiihtyvyys	& \alpha	& rad/s		& \bm{\alpha} = \frac{d \bm{\omega}}{dt} = \frac{d^2 \bm{\theta}}{dt^2}\\
ratanopeus		& v			& m/s	& v = \bm{\omega} \times \bm{r} \\
kierrosaika		& T			& s		& T = \frac{2 \pi}{\omega} \\
kierrostaajuus	& f, n		& 1/s, Hz	& f = \frac{1}{T} \\
tangenttikiihtyvyys	& a_t	& m/s$^2$	& a_t = r \alpha \\
normaalikiihtyvyys	& a_n	& m/s$^2$	& a_n = \frac{v^2}{r} \\
kiihtyvyys			& a		& m/s$^2$	& a_r \hat{\bm{r}} + a_t \hat{\bm{t}} \\
työ					& W	& J	& W = \int \bm{T} \cdot \bm{d\theta} \\
kulmaliikemäärä	& L	& kgm$^2$/s	& \bm{L} = \bm{r} \times \bm{p} \\
kineettinen energia	&E_k, K	& J	& E_k = \frac{1}{2} I \omega^2 \\
potentiaalienergia	&E_p, U	& J	& E_p = \frac{1}{2} c \theta^2 \\
\hline
\textbf{Tasaisesti muuttuva pyörimisliike} &&& \\
loppukulmanopeus	& \omega & rad/s	& \omega = \omega_0 + \alpha t \\
kiertokulma			& \theta	& rad	& \theta = \theta_0 + \omega_0 t + \frac{1}{2} \alpha t^2 \\
\end{eqtable-units}


\begin{eqtable-units}{Voima, energia}
\textbf{Voima} & F	& N	& \\
Newtonin gravitaatiolaki	&&& \bm{F} = -G \frac{m_1 m_2}{r^2} \hat{\bm{r}} \\
homogeeninen gravitaatiokenttä	&&& \bm{g} = \frac{\bm{F}}{m} \\
liikekitka	& F_\mu	&& F_\mu = \mu N \\
harmoninen voima	&&& F = -kx \\
\hline
\hline
impulssi	& I	& Ns	& \bm{I} = \int \bm{F} dt \\
teho		& P	& W		& P = \frac{dW}{dt} \\
\hline
\textbf{Potentiaalienergia} & E_p	& J	& \\
gravitaatiokenttä	&&& E_p = mgh \\
					&&& E_p = -G \frac{m_1 m_2}{r} \\
harmoninen voimakenttä	&&& E_p = \frac{1}{2} kx^2 \\
\hline
\textbf{Kineettinen energia}	& E_k, K	& J & \\
etenevän liikkeen energia	&&& E_k = \frac{1}{2} mv^2 \\
pyörimisenergia				&&& E_k = \frac{1}{2} J \omega^2 \\
\hline
mekaaninen hyötysuhde	& \eta	&& \eta = \frac{E_a}{E_o} = \frac{P_a}{P_o} \\
\hline
\textbf{Harmoninen värähtelijä} &&& \\
poikkeama	&&& x(t) = A \sin(\omega t + \phi ) \\
jaksonaika	&&& T = 2 \pi \sqrt{\frac{m}{k}} \\
\end{eqtable-units}


\begin{eqtable-units}{Heilureita ja hitausmomentteja}
\textbf{Heilahdusaika} &&& \\
matemaattinen heiluri	&&& T = 2 \pi \sqrt{\frac{l}{g}} \\
fysikaalinen heiluri	&&& T = 2 \pi \sqrt{\frac{I_A}{mgl}} \\
kiertoheiluri			&&& T = 2 \pi \sqrt{\frac{J}{D}} \\
\hline
voiman momentti	& M	& Nm	& \bm{M} = \bm{r} \times \bm{F} = \frac{d\bm{L}}{dt} \\
pyörimisen liikeyhtälö	&&& \sum M = I \alpha \\
impulssimomentti	& I	& kgm$^2$/s	& I_M = \Delta L = M \Delta t \\ \hline
\textbf{Hitausmomentteja}	& I, J	& kgm$^2$ & \\
pistemäinen kappale		&&& I = mr^2 \\
umpinainen sylinteri	&&& I = \frac{1}{2} mr^2 \\
ohutseinäinen rengas	&&& I = mr^2 \\
paksuseinäinen rengas	&&& I = \frac{1}{2}m(r^2_1+r^2_2) \\
ohut sauva (pään ympäri) &&& I = \frac{1}{3}ml^2 \\
ohut sauva (keskipisteen ympäri)	&&& I = {1}{12}ml^2 \\
suorakulmainen levy	&&& I = \frac{1}{12}m(a^2+b^2) \\
umpinainen pallo	&&& I = \frac{2}{5} mr^2 \\
ohutseinäinen pallo	&&& I = \frac{2}{3}mr^2 \\ \hline
Steinerin sääntö (akselin siirto)	&&& I_A = I_P + ma^2 \\
\end{eqtable-units}

\begin{eqtable-units}{Jatkumon mekaniikkaa \cite[TESTI]{MAOL}}
tiheys	& \rho & kg/m$^3$	& \rho = \frac{m}{V} \\
jännitys	& \sigma & N/$^2$ & \sigma = \frac{F}{A} \\
Hooken laki, kimmoisuus		& E	& N/m$^2$	& \frac{F}{A} = E \frac{\Delta l}{l} \\
paine	& p	& Pa	& p = \frac{F}{A} \\
hydrostaattinen paine	& p	& Pa	& p = h \rho g \\
noste		& F_N	& N	& F_N = \rho V g \\
\hline
\textbf{Pintajännitys} & \sigma	& N/m, J/m$^2$ & \\
voima	& F	& N	& F = \sigma l \\
energia	& E	& J	& E = \sigma A \\
\hline
\textbf{Viskositeetti} & \eta & Ns & \\
voima	& F	& N	& F = \frac{\eta A v}{d} \\
\hline
Bernoullin yhtälö	&&& p_0 + \frac{1}{2} \rho v^2 + h \rho g = vakio \\
\end{eqtable-units}


\begin{eqtable-units}{Aaltoliike ja valo-oppi \cite{MAOL}}
Aaltoliikkeen nopeus					&&& v = f \lambda \\
huojuntataajuus							&&& f = |f_1 - f_2 | \\
intensiteetti			& I	& W/m$^2$	& I = \frac{P}{A} \\
energiatiheys			& w	& J/m$^3$	& w = \frac{I}{v}, \quad w = kf^2A^2 \\
\hline
\textbf{Dopplerin ilmiö} &&& \\
aaltolähde liikkuu		&&& f = f_0 \frac{v}{v \pm v_1} \\
havaitsija liikkuu		&&& f = f_0 \frac{v \pm v_h}{v} \\
\hline
äänen nopeus kaasussa	& v	&& \frac{v_1}{v_2} = \sqrt{\frac{T_1}{T_2}} \\
äänen intensiteettitaso	& L	& dB	& L = 10 \log_10 \frac{I}{I_0} dB, \quad I_0 = 1 \text{pW/m}^2 \\
taittumislaki	&&& \frac{\sin \alpha_1}{\sin \alpha_2} = \frac{v_1}{v_2} = \frac{n_2}{n_1} = n_{12} \\
Brewsterin laki	&&& \tan \alpha_B = \frac{n_2}{n_1} \\
% hilayhtälö	&&& d \sin \alpha = k \lambda \\
kuvausyhtälö	&&& \frac{1}{a} + \frac{1}{b} = \frac{1}{f} \\
taittovoimakkuus	& D	& 1/m = d	& D = \frac{1}{f} \\
viivasuurennus		& m	&& m = \big| \frac{b}{a} \big| \\
kulmasuurennus		& M	&& M = \frac{\tan \alpha_2}{\tan \alpha_1} \\
\hline
\textbf{Suurennuksia} &&& \\
suurennuslasi	&&& M = \frac{s}{f} \\
mikroskooppi	&&& M = \frac{Ls}{f_{ob} f_{ok}} \\
kaukoputki		&&& M = \frac{f_{ob}}{f_{ob}} \\
\hline
valovoima	& I	& cd	& I = \frac{\Phi}{\omega} \\
luminanssi	& L	& cd/m$^2$	& L = \frac{I}{A} \\
valovirta	& \Phi	& lm	& \Phi = I \omega \\
valaistusvoimakkuus	& E	& lx	& E = \frac{\Phi}{A} \\
\end{eqtable-units}



\subsection{Sähkömagnetismi}

\begin{eqtable}{Sähkömagnetismi \cite{UPhysics}}
\textbf{Maxwellin yhtälöt} & \\
Gaussin laki sähkökentille		& \oiint_S \bm{D} \cdot d\bm{A} = \sum q \\
Gaussin laki magneettikentille	& \oiint_S \bm{B} \cdot d\bm{A} = 0 \\
Ampere-Maxwell					& \oint_C \bm{H} \cdot d\bm{l} = I + \frac{d}{dt} \iint_S \bm{D} \cdot d\bm{A} \\
Faradayn laki					& \oint_C \bm{E} \cdot d\bm{l} = - \frac{d}{dt} \iint_S \bm{B} \cdot d\bm{A} \\
\hline
& E = vB \\
aaltoyhtälö	& \frac{\partial^2 H}{\partial z^2} = \mu \epsilon \frac{\partial^2 H}{\partial t^2} \\
\end{eqtable}



\subsection{Suhteellisuus}

\begin{table}[ht!]
\centering
\caption{Suhteellisuus \cite{UPhysics}}
\begin{tabular}{| >{$\displaystyle} l <{$} | >{$\displaystyle} l <{$} |} \hline
\textbf{Klassinen suhteellisuus} & \textbf{Suppea suhteellisuusteoria} \\ \hline
x' = x + vt	& x' = \gamma (x+vt), \quad \gamma = \frac{1}{\sqrt{1 - (\frac{v}{c})^2}} \\ 
t' = t		& t' = \gamma (t + \frac{v}{c^2} x) \\
l = l'		& l = \frac{l'}{\gamma} \\
t = t'		& t = \gamma t' \\
u_x = u'_x + v, \quad u_y = u'_y	& u_x = \frac{u'_x+v}{1 + \frac{u'_x v}{c^2}}, \quad u_y = \frac{u'_y}{1+\frac{u'_xv}{c^2}} \\
\bm{p} = m\bm{u}	& \bm{p} = \gamma m \bm{u} \\
E = \frac{p^2}{2m_0} + m_ 0 c^2	& E = \gamma m_0 c^2 \Rightarrow E^2 = c^2p^2 + m^2c^4 \\
H(\bm{r}, \bm{p}) = \frac{p^2}{2m} + U(\bm{r})	& H(\bm{r}, \bm{p}) = \sqrt{c^2p^2 + m^2c^4} + U(\bm{r}) \\
\hline
\end{tabular}
\end{table}


\clearpage
\subsection{Varhainen kvanttimekaniikka}

\begin{eqtable}{\href{https://en.wikipedia.org/wiki/Black-body_radiation}{Mustan kappaleen säteily} \cite[s. 124-128]{ModernPhysics}}
Stefanin-Boltzmannin laki	& R = \sigma T^4 (=I=\frac{P}{A})\\ \hline
lämpötilan ja kin. energ. yhteys & E_{ave} = kT \\ \hline
Wienin siirtymälaki			& \lambda_m T = 2,898 \cdot 10^{-3} m \cdot K \\ \hline
Rayleigh-Jeans				& u( \lambda ) = kT n( \lambda ) = \frac{8 \pi kT}{\lambda^4} \\ \hline
Maxwell-Boltzmann-jakauma	& \phi (E) = A e^{-\frac{E}{kT}} \\ \hline
Planckin laki				& u(\lambda) = \frac{8 \pi hc \lambda^{-5}}{e^{\frac{hc}{\lambda kT}} - 1} \\
\end{eqtable}


\begin{eqtable}{\href{https://en.wikipedia.org/wiki/Compton_scattering}{Comptonin sironta} \cite[s. 142]{ModernPhysics}}
Comptonin yhtälö			& \Delta \lambda = \frac{h}{mc}(1 - \cos \theta ) \\ \hline
Comptonin aallonpituus		& \lambda_c = \frac{h}{mc} \\
\end{eqtable}


\begin{eqtable}{\href{https://en.wikipedia.org/wiki/Rutherford_scattering}{Rutherfordin sironta} \cite[s. 160-163]{ModernPhysics}}
törmäysparametri			& b = \frac{k q_\alpha Q}{m_\alpha v^2} \cot \frac{\theta}{2} \\ \hline
yli kulmaan $\theta$ siroavien osuus	& f = \pi b^2 nt \\ \hline
detektorilla havaittujen hiukkasten määrä	& \Delta N = ( \frac{I_0 A_{sc} nt}{r^2} ) ( \frac{kZe^2}{2 E_k} ) \frac{1}{\sin^4 \frac{\theta}{2}} \\ \hline
ydintä tilavuudessa	(yksiköt!) & n = \frac{\rho N_A}{M} \\
\end{eqtable}


% If you somehow find this article:
% Humphreys, C.J. (1953), "The Sixth Series in the Spectrum of Atomic Hydrogen", J. Research Natl. Bur. Standards
% then please add it as the source. So far the source of this information is Wikipedia:
% https://en.wikipedia.org/wiki/Hydrogen_spectral_series
\begin{table}[ht!]
\centering
\caption{Vedyn spektrisarjat}
\begin{tabular}{| >{$\displaystyle} l <{$} | l |} \hline
1 \rightarrow & Lyman \\
2 \rightarrow & Balmer \\
3 \rightarrow & Paschen \\
4 \rightarrow & Brackett \\
5 \rightarrow & Pfund \\
6 \rightarrow & Humphreys \\
\hline
\end{tabular}
\end{table}


\begin{eqtable}{\href{https://en.wikipedia.org/wiki/Bohr_model}{Bohrin atomimalli} \cite[s. 166-171]{ModernPhysics}}
elektronin ratanopeus	& v = \sqrt{ \frac{kZe^2}{mr}} \\ \hline
kulmaliikemäärän kvantittuminen	& | \mathbf{L} | = n \hbar, n \in \mathbb{N} \\ \hline
atomin energiatilat			& E_n = -\frac{mk^2 z^2 e^4}{2 \hbar^2 n^2} = -\frac{z^2 E_0}{n^2} \\ \hline
spektriviivat (yleistetty Rydberg-Ritz)	& \frac{1}{\lambda} = Z^2 R ( \frac{1}{n_f^2} - \frac{1}{n_i^2}) \\ \hline
Redusoitu massa					& \mu = \frac{mM}{m + M} \\ \hline
\end{eqtable}


\begin{eqtable}{Ytimellinen atomi \cite[s. 176-178]{ModernPhysics}}
Moseleyn laki	& \sqrt{f} = A_n (Z - b) \\ \hline
				& A_n^2 = c R_\infty (1 - \frac{1}{n^2}) \\ \hline
K$_\alpha$:lle	& n=2, b=1 \\
\end{eqtable}



\begin{eqtable}{Sekalaista}
valosähköinen ilmiö			& eV_0 = E_{k max} = hf - \phi \\ \hline
Braggin laki				& 2d \sin \theta = n \lambda \\ \hline
Duane-Hund (jarrutussäteily)	& \lambda_{min} = \frac{1,24 \cdot 10^3 nm}{V (V)} \\
ytimen säde	& r_d = \frac{k q_\alpha Q}{\frac{1}{2}m_\alpha v^2} \\ \hline

kulmaliikemäärän kvantittuminen	& L = mvr = n \hbar \\ \hline

Davisson-Germer ASDF		& n \lambda = D sin \phi \\
\end{eqtable}

\clearpage
\subsection{Kvanttimekaniikka}

\begin{eqtable-full}{\href{https://en.wikipedia.org/wiki/Wave\%E2\%80\%93particle_duality}{Aaltohiukkasdualismi} \cite[s. 193-233]{ModernPhysics}}
De Broglie -aallonpituus	& \lambda = \frac{h}{p} \\
							& f = \frac{E}{h} \\ \hline
yleinen aaltoyhtälö (1D)	& \frac{d^2 y}{dx^2} = \frac{1}{v^2} \frac{d^2 y}{dt^2} \\
ratkaisut muotoa			& y(x,t) = f(kx - \omega t) \\ \hline
vaihenopeus					& v_p = f \lambda = \frac{\omega}{k} \\ \hline
ryhmänopeus					& v_g = \frac{d \omega}{d k} = v_p + k \frac{d v_p}{dk} \\ \hline
Huom!						& k= \frac{2 \pi}{\lambda} \Rightarrow dk = - \frac{2 \pi}{\lambda^2} d \lambda \\
klassinen epätarkkuus		& \Delta k \Delta x \sim 1 \\
							& \Delta \omega \Delta t \sim 1 \\
epätarkkuusperiaate         & \Delta x \Delta p \sim \hbar \\
                           	& \Delta E \Delta t \sim \hbar \\
							& \Delta x \Delta p \geq \frac{1}{2} \hbar \\
							& \Delta E \Delta t \geq \frac{1}{2} \hbar \\ \hline
nollapiste-energia          & E \geq \frac{h^2}{2mL^2} \\ \hline
yleinen Schrödingerin aaltoyhtälö	& \hat{H} \Psi = E \Psi \\ \hline
yksiulotteinen Schrödinger	& - \frac{\hbar ^2}{2m} \frac{\partial^2 \Psi (x, t)}{\partial x^2} + V(x, t) \Psi (x, t) = i \hbar \frac{\partial \Psi (x, t)}{\partial t} \\ \hline
todennäköisyystulkinta (Kööpenhamina)	& P(x) dx = | \psi | ^2 dx \\ \hline
normalisointiehto			& \int_{-\infty}^{\infty} \Psi^* \Psi dx = 1 \\
\end{eqtable-full}

\begin{table}
\centering
\caption{\href{https://en.wikipedia.org/wiki/Operator_(physics)}{Kvanttimekaanisia operaattoreita} \cite[s. 252]{ModernPhysics}, \cite[s. 40]{SSED}}
\setlength{\extrarowheight}{10pt}
\begin{tabu} to \textwidth { D | l | Y } \hline
\text{Symboli}	& Suure & \text{Kvanttimekaaninen operaattori} \\ \hline
x		& paikka	& x \\
f(x)	& mielivaltainen x:n funktio	& f(x) \\
p_x	& liikemäärän x-komponentti	& \frac{\hbar}{i} \frac{\partial}{\partial x} \\
E		& aikariippumaton Hamiltonin operaattori	& \frac{p_{op}^2}{2m} + V(x) \\
E		& aikariippuva Hamiltonin operaattori & -\frac{\hbar}{i} \frac{\partial}{\partial t} \\
E_k		& kineettinen energia	& -\frac{\hbar^2}{2m} \frac{\partial^2}{\partial x^2} \\
L_z		& kulmaliikemäärän z-komponentti	& -i \hbar \frac{\partial}{\partial \phi} \\
		& odotusarvon laskeminen \cite[6-46]{ModernPhysics}	& \langle f(x) \rangle = \int_{-\infty}^\infty \psi^* f(x) \psi dx \\
\end{tabu}
\end{table}

\begin{eqtable}{Kvanttimekaanisia esimerkkitilanteita}
\href{https://en.wikipedia.org/wiki/Particle_in_a_box}{Ääretön potentiaalilaatikko}	\cite[6-2]{ModernPhysics} & n \frac{\lambda}{2} = L, \quad n \in \mathbb{Z}^+ \\
							& E = n^2 \frac{\pi^2 \hbar^2}{2mL^2} = n^2 E_1, \quad n \in \mathbb{Z}^+ \\
							& \psi_n (x) = \sqrt{\frac{2}{L}} \sin \frac{n \pi x}{L} \\ \hline
\href{https://en.wikipedia.org/wiki/Finite_potential_well}{Rajallinen potentiaalilaatikko} \cite[6-3]{ModernPhysics}	& \\
laatikossa						& \psi''(x) = -k^2 \psi(x), \quad k^2 = \frac{2mE}{\hbar^2} \\

laatikon ulkopuolella			& \psi'' (x) = \alpha^2 \psi (x) \\
								& \alpha^2 = \frac{2m}{\hbar^2}(V_0-E) \\ \hline
\end{eqtable}

\begin{eqtable}{Harmoninen oskillaattori \cite[6-5]{ModernPhysics}}
energia 		& E_n = (n + \frac{1}{2}) \hbar \omega, \quad n \in \mathbb{N} \\
aaltofunktiot	& \psi_n (x) = C_n e^{\frac{m \omega x^2}{2 \hbar}} H_n (x) \\
				& \text{jossa } H_n \text{ on Hermiten polynomi} \\
				& \psi_0 (x) = A_0 e^{\frac{-m \omega x^2}{2 \hbar}} \\
				& \psi_1 (x) = A_1 \sqrt{\frac{m \omega}{\hbar}} e^{\frac{-m \omega x^2}{2 \hbar}} \\
				& \psi_2 (x) = A_2 (1- \frac{2m \omega x^2)}{\hbar}) e^{\frac{-m \omega x^2}{2 \hbar}} \\
valintasäännön alkuperä	& \int_{-\infty}^{\infty} \psi^*_n x \psi_m dx = 0, \quad \text{ellei } n=m \pm 1 \\
\end{eqtable}

\begin{eqtable}{Potentiaaliaskel \cite[6-6]{ModernPhysics}}
aaltofunktiot	& x<0: \quad \frac{d^2 \psi (x)}{dx^2} = -k_1^2 \psi (x), \quad k_1 = \frac{\sqrt{2mE}}{\hbar} \\
				& x>0: \quad \frac{d^2 \psi (x)}{dx^2} = -k_2^2 \psi (x), \quad k_2 = \frac{\sqrt{2m(E-V_0)}}{\hbar} \\
yleiset ratkaisut	& x<0: \psi_1 (x) = Ae^{ik_1 x} + Be^{-ik_1 x} \\
					& x>0: \psi_2 (x) = Ce^{ik_1 x} + De^{-ik_1 x}, \quad \text{vasemmalta} \rightarrow D=0 \\
$\Rightarrow$	& B = \frac{k_1 - k_2}{k_1 + k_2} A = \frac{\sqrt{E} - \sqrt{E-V_0}}{\sqrt{E} + \sqrt{E-V_0}} A \\
				& C = \frac{2k_1}{k_1 + k_2} A = \frac{2 \sqrt{E}}{\sqrt{E} + \sqrt{E-V_0}} A \\
heijastus		& R = \frac{|B|^2}{|A|^2} = \Big( \frac{k_1-k_2}{k_1+k_2} \Big)^2 \\
transmissio		& T = \frac{k_2}{k_1} \frac{|C|^2}{|A|^2} = \frac{4k_1 k_2}{(k_1 + k_2)^2} \\
\end{eqtable}

\begin{eqtable}{\href{https://en.wikipedia.org/wiki/Rectangular_potential_barrier}{Potentiaalieste} \cite[6-6]{ModernPhysics}}
transmissio		& T = \frac{|F|^2}{|A|^2} = \Big(1 + \frac{\sinh^2 (\alpha a)}{4 \frac{E}{V_0} \big(1- \frac{E}{V_0} \big) } \Big)^{-1} \\
				& \approx 16 \frac{E}{V_0} \Big( 1- \frac{E}{V_0} \Big) e^{-2 \alpha a} \\
\end{eqtable}


\begin{eqtable}{Atomifysiikka \cite[s. 277->]{ModernPhysics} }
kolmiulotteinen Schrödinger	\cite[s. 41]{SSED} & - \frac{\hbar}{2m} \nabla^2 \Psi + V \Psi = -\frac{\hbar}{i}\frac{\partial \Psi}{\partial t} \\
							& \nabla^2 \Psi = \frac{\partial^2 \Psi}{\partial x^2} + \frac{\partial^2 \Psi}{\partial y^2} + \frac{\partial^2 \Psi}{\partial z^2} \\
\end{eqtable}

\begin{eqtable}{Kolmiulotteinen potentiaalilatikko \cite[7-1]{ModernPhysics}}
esimerkkiratkaisu	& \psi(x, y, z) = A \sin (k_1 x) \sin (k_2 y) \sin (k_3 z) \\
energiatilat	& E_{n_1 n_2 n_3} = \frac{\hbar^2 \pi^2}{2m} (\frac{n_1^2}{L_1^2} + \frac{n_2^2}{L_2^2} + \frac{n_3^2}{L_3^2}), \quad \forall n \in \mathbb{Z}^+ \\
\end{eqtable}

Atomifysiikan symbolikäytänteet polaarisessa koordinaatistossa \cite[7-8]{ModernPhysics}
% A picture of this (as in the page 280 of Modern Physics) is needed!
\begin{align*}
x &= r \sin \theta \cos \phi \\
y &= r \sin \theta \sin \phi \\
z &= r \cos \theta
\end{align*}

\begin{eqtable-full}{\href{https://en.wikipedia.org/wiki/Spherical_harmonics}{Palloharmonisten funktioiden matematiikkaa} \cite[r26-27, s. 337-342]{Physics227} \cite[s. 127-129]{JohdatusTaivaanmekaniikkaan}}
% \cite[r26-27, s. 336]{Physics227}
Legendren polynomit				& P_l (x) = \frac{(-1)^l}{2^l l!} \frac{d^l}{dx^l} (1-x^2)^l \\
% Symbols |_ and _| would be needed in the following line in the l/2 part: It should be |_l/2_|
%								& = \sum_{k=0}^{l/2} \frac{(-1)^k (2l-2k)!}{2^{2l}k!(l-k)!(l-2k)!} x^{l-2k} \\
% \cite[r26-27, s. 336]{Physics227}
Legendren liittofunktiot	& P_{l,m} (x) = (-1)^m \sqrt{ (1-x^2)^m } \frac{d^m}{dx^m} P_l(x) \\
% \cite[r26-27, s. 338]{Physics227}
Laguerren polynomit				& L_j (x) = e^x \frac{d^j}{d x^j} e^-x x^j \\
% \cite[r26-27, s. 339]{Physics227}
assosioidut Laguerren polynomit	& L_j^k (x) = (-1)^k \frac{d^k}{dx^k} L_{j+k} (x) \\
radiaalisen aaltoyhtälön ratkaiseminen	& y_j^k (x) = e^{-\frac{x}{2}} x^{\frac{k+1}{2}} L_j^k (x) \\
										& R(r) = \frac{y(r)}{r} \\
\end{eqtable-full}

\begin{table}[!ht]
\centering
\caption{\href{https://en.wikipedia.org/wiki/Table_of_spherical_harmonics}{Palloharmonisia funktioita} \cite[s.282]{ModernPhysics}}
\setlength{\extrarowheight}{15pt}
\begin{tabular}{ | >{$\displaystyle} l <{$} | >{$\displaystyle} l <{$} | >{$\displaystyle} l <{$} | } \hline
l=0	& m=0	& Y_{00} = \sqrt{\frac{1}{4 \pi}} \\
l=1	& m=1	& Y_{11} = -\sqrt{\frac{3}{8 \pi}} \sin \theta e^{i \phi} \\
	& m=0	& Y_{10} = \sqrt{\frac{3}{4 \pi}} \cos \theta \\
	& m=-1	& Y_{1-1} = \sqrt{\frac{3}{8 \pi}} \sin \theta e^{-i \phi} \\
l=2	& m=2	& Y_{22} = \sqrt{\frac{15}{32 \pi}} \sin^2 \theta e^{2i \phi} \\
	& m=1	& Y_{21} = -\sqrt{\frac{15}{8 \pi}} \sin \theta \cos \theta e^{i \phi} \\
	& m=0	& Y_{20} = \sqrt{\frac{5}{16 \pi}} (3 \cos^2 \theta - 1) \\
	& m=-1	& Y_{2-1} = \sqrt{\frac{15}{8 \pi}} \sin \theta \cos \theta e^{-i \phi} \\
	& m=-2	& Y_{2-2} = \sqrt{\frac{15}{32 \pi}} \sin^2 \theta e^{-2i \phi} \\
\hline
\end{tabular}
\end{table}

Schrödingerin aaltoyhtälö pallokoordinaatistossa
\begin{equation*}
- \frac{\hbar^2}{2\mu} \frac{1}{r^2} \frac{\partial \psi}{\partial r} \Big( r^2 \frac{\partial \psi}{\partial r} \Big)
- \frac{\hbar^2}{2 \mu r^2}
\Big(
\frac{1}{\sin \theta} \frac{\partial}{\partial \theta} \big( \sin \theta \frac{\partial \psi}{\partial \theta} \big)
+ \frac{1}{\sin^2 \theta} \frac{\partial^2 \psi}{\partial \phi^2}
\Big)
+ V(r) \psi = E \psi
\end{equation*}

Palloharmonisten funktioiden generointi \cite[r26-27, (10-10)]{Physics227} \cite[(10.10)]{JohdatusTaivaanmekaniikkaan}
\begin{equation*}
Y_{l,m} (\theta, \phi) = (-1)^m \sqrt{ \frac{(2l+1)(l-m)!}{4 \pi (l+m)!} } P_{l,m}(\cos \theta ) e^{im \phi}, \quad m \geq 0
\end{equation*}

\begin{eqtable}{Puolijohteet}
Efektiivinen massa \cite[s. 73]{SSED}	& m^* = \frac{\hbar^2}{d^2 E / d \bm{k}^2} \\
\end{eqtable}
