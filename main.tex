% Dokumentin määrittely
\documentclass[a4paper,12pt,finnish]{book}
\usepackage[utf8]{inputenc}
\usepackage[T1]{fontenc}
\usepackage[finnish]{babel}

% Käytettävät paketit
\usepackage{amsmath}	% Yhtälöt
\usepackage{amsfonts}	% ?
\usepackage{amssymb}	% Symbolit, \mathbb{}
\usepackage{array}		% Taulukot
\usepackage{bm}			% Vektorien lihavointi
\usepackage{cite}		% Lähdeviitteet
\usepackage{esint}		% Kiertointegraalit \oint \oiint
\usepackage{graphicx}	% Kuvat
\usepackage{hyperref}	% Hyperlinkit
\usepackage{pslatex}	% Parantaa pdf:n ulkonäköä
% \usepackage{xkeyval}	% Parametrit omille komennoille

% Ehkäistään unicode non-breaking space -merkin aiheuttamat ongelmat
\DeclareUnicodeCharacter{00A0}{UNICODE_ERROR}

% Komentomäärittelyt
\newenvironment{taulukko}[1]{
	\begin{table}[!ht]
	\centering
	\caption{#1}
	\setlength{\extrarowheight}{10pt}
	\begin{tabular}{| l | >{$\displaystyle} l <{$} |} \hline
}{
	\end{tabular}
	\end{table}
}

\newenvironment{kaavataulukko}[1]{
	\begin{table}[!ht]
	\centering
	\caption{#1}
	\setlength{\extrarowheight}{10pt}
	\begin{tabular}{| l | >{$} l <{$} | l | >{$\displaystyle} l <{$} |} \hline
}{
	\end{tabular}
	\end{table}
}

\newenvironment{matriisi}{
	\setlength{\extrarowheight}{0pt}
	\begin{bmatrix}
}{
	\end{bmatrix}
}

\newcommand{\sisatulo}[2]{
	\langle \bm{#1}, \bm{#2} \rangle
}

\newenvironment{yhtaloryhma}{
	\setlength{\extrarowheight}{0pt}
	\begin{cases}
}{
	\end{cases}
}
%-----

% Metadata
\title{Linnunradan käsikirja teekkareille}
\author{Tampereen teekkarit}
\date{\the\day.\the\month.\the\year}


\begin{document}

% Otsikkosivu
\begin{titlepage}
\centering
{\huge Linnunradan käsikirja teekkareille \par}
% {\large Tampereen teekkarit \par}
{\the\day.\the\month.\the\year \par}
\vfill
\includegraphics{by-sa.eps}
\end{titlepage}

% Esipuhe
Maailma on kaavoja ja taulukoita täynnä, mutta ne ovat olleet hajallaan useissa kirjoissa,
joista tiedon etsiminen on hidasta - varsinkin paperiversioista, joissa Ctrl+F ei luonnollisesti toimi.
Lisäksi monet niistä ovat tiukkojen tekijänoikeuksien ja kopiokieltojen alaisia,
vaikka tieto itsessään on vapaata. Nyt niille on vaihtoehto. Ole hyvä.

% Disclaimer
Tämä kirja on opiskelijaprojekti, joten emme voi ottaa vastuuta mahdollisista virheistä.
Toivommekin, että ilmoitat niistä vaikkapa GitHubin kautta, jotta voimme korjata ne.

% Tekijät
\begin{table*}[h!]
\centering
\begin{tabular}{ll}
Mika \textit{AgenttiX} Mäki		& Projektin perustaja \\
Daniel \textit{Ohems} Saarimäki \\
% Esimerkki					& Lisäykset
\end{tabular}
\end{table*}

% Lisenssi
\vfill
\href{https://creativecommons.org/licenses/by-sa/4.0/}{Creative Commons Attribution-ShareAlike 4.0 International}

% Sisällysluettelo
\tableofcontents

% Itse asia
\part{Matematiikka}

\section{Merkintöjä}

\begin{table}[ht!]
\centering
\caption{Kreikkalaiset aakkoset \cite[s. 8]{MAOL}, \cite[sisäkansi]{ModernPhysics}}
\begin{tabular}{  >{$} l <{$}  >{$} l <{$} l l} \hline
\text{Iso}			& \text{Pieni}		& Nimi suomeksi	& Nimi englanniksi \\ \hline
\text{A}	& \alpha	& alfa			& alpha \\
\text{B}	& \beta		& beeta			& beta \\
\Gamma		& \gamma	& gamma			& gamma \\
\Delta		& \delta	& delta			& delta \\
\text{E}	& \epsilon, \varepsilon	& epsilon		& epsilon \\
\text{Z}	& \zeta		& zeeta			& zeta \\
\text{H}	& \eta		& eeta			& eta \\
\Theta		& \theta, \vartheta	& theeta		& theta \\
\text{I}	& \iota		& ioota			& iota \\
\text{K}	& \kappa	& kappa			& kappa \\
\Lambda		& \lambda	& lambda		& lambda \\
\text{M}	& \mu		& myy			& mu \\
\text{N}	& \nu		& nyy			& nu \\
\Xi			& \xi		& ksii			& xi \\
\text{O}	& o	& omikron		& omicron \\
\Pi			& \pi		& pii			& pi \\
\text{P}	& \rho		& rhoo			& rho \\
\text{T}	& \tau		& tau			& tau \\
\Upsilon, \text{Y}	& \upsilon	& ypsilon	& upsilon \\
\Phi	& \phi, \varphi	& fii			& phi \\
\text{X}	& \chi		& khii			& chi \\
\Psi		& \psi		& psii			& psi \\
\Omega		& \omega	& oomega		& omega \\
\end{tabular}
\end{table}

\section{Logiikka}

\begin{table}[ht!]
\centering
\caption{Loogisia operaatioita \cite[s. 6, 9]{MAT-01160}}
\begin{tabular}{
>{$} l <{$}  >{$} l <{$} |
>{$} l <{$}  >{$} l <{$} >{$} l <{$} >{$} l <{$} >{$} l <{$} >{$} l <{$} >{$} l <{$} >{$} l <{$}
}
p & q & \neg p	& p \land q	& p \lor q	& p \rightarrow q	& p \leftrightarrow q	& \overline{p}	& pq	& p + q \\ \hline
1 & 1 & 0 & 1 & 1 & 1 & 1 & 0 & 1 & 1 \\
1 & 0 & 0 & 0 & 1 & 0 & 0 & 0 & 0 & 1 \\
0 & 1 & 1 & 0 & 1 & 1 & 0 & 1 & 0 & 1 \\
0 & 0 & 1 & 0 & 0 & 1 & 1 & 1 & 0 & 0 \\
\end{tabular}
\end{table}


\begin{taulukko}{Päättelysääntöjä \cite[s. 8]{MAT-01160}}
kaksoisnegaation poisto	& \neg \neg p \Leftrightarrow p \\
						\hline
vaihdantalait			& p \lor q \Leftrightarrow q \lor p \\
						& p \land q \Leftrightarrow q \land p \\
						\hline
liitäntälait			& p \lor (q \lor r) \Leftrightarrow (p \lor q) \lor r \\
						& p \land (q \land r) \Leftrightarrow (p \land q) \land r \\
						\hline
osittelulait			& p \land (q \lor r) \Leftrightarrow (p \land q) \lor (p \land r) \\
						& p \lor (q \land r) \Leftrightarrow (p \lor q) \land (p \lor r) \\
						\hline
de Morganin lait		& \neg (p \lor q) \Leftrightarrow \neg p \land \neg q \\
						& \neg (p \land q) \Leftrightarrow \neg p \lor \neg q \\
						\hline
ekvivalenssilaki		& (p \leftrightarrow q) \Leftrightarrow (p \rightarrow q) \land (q \rightarrow p) \\
						\hline
suora todistus			& p \land (p \rightarrow q) \Rightarrow q \\
						\hline
epäsuora todistus		& (p \rightarrow q) \Leftrightarrow (\neg q \rightarrow \neg p) \\
						& \big( p \land ((p \land \neg q) \rightarrow (r \land \neg r)) \big) \Rightarrow q \\ \hline
\end{taulukko}


\begin{taulukko}{Joukko-operaatioiden laskulakeja \cite[s. 15]{MAT-01160}}
	& (A^c)^c = A \\
vaihdantalait	& A \cup B = B \cup A \\
				& A \cap B = B \cap A \\
				\hline
liitäntälait	& A \cup (B \cup C) = (A \cup B) \cup C \\
				& A \cap (B \cap C) = (A \cap B) \cap C \\
				\hline
osittelulait	& A \cap (B \cup C) = (A \cap B) \cup (A \cap C) \\
				& A \cup (B \cap C) = (A \cup B) \cap (A \cup C) \\
				\hline
de Morganin lait	& (A \cup B)^c = A^c \cap B^c \\
					& (A \cap B^c = A^c \cup B^c \\ \hline
\end{taulukko}

Negaation ja kvanttorin vaihtosääntö \cite[s. 17]{MAT-01160}
\begin{align}
\neg (\forall x : p(x)) &\Leftrightarrow \exists x : \neg p(x) \\
\neg (\exists x : p(x)) &\Leftrightarrow \forall x : \neg p(x)
\end{align}

\section{Funktiot}

\begin{taulukko}{Funktiot \cite[s. 25-26]{MAT-01160}}
käänteisfunktio	& y = f(x) \Leftrightarrow x = f^{-1}(y) \\
				& f^{-1}(f(x)) = x \quad \land \quad f(f^{-1}(y)) = y \\ \hline
kuvaaja eli graafi ($f: A \rightarrow \mathbb{R}$)
				& G_f = \{(x, f(x)) \in \mathbb{R}^2 : x \in A \}
				= \{(x,y) \in \mathbb{R}^2 : x \in A, y = f(x) \} \\ \hline
\end{taulukko}

\section{Differentiaali- ja integraalilaskenta}

\subsection{Derivointi}

\subsection{Integrointi}


\section{Vektorit ja matriisit}

\subsection{Vektorit}

\begin{taulukko}{Vektorien perusteet \cite[s. 2-6]{MAT-60000}}
Vektori						& \bm{x} = \begin{matriisi} x_1 \\ x_2 \\ \vdots \\ x_n \end{matriisi} \\ \hline
Luonnolliset kantavektorit	& \bm{e}_i = \begin{matriisi} 0 \\ \vdots \\ 0 \\ 1 \\ 0 \\ \vdots \\ 0 \end{matriisi}, \qquad i \in \mathbb{Z}^+ \\ \hline
Nollavektori				& \bm{0} = \begin{matriisi} 0 \\ \vdots \\ 0 \end{matriisi} \\ \hline
Vektorien summa				& \bm{x} + \bm{y} = \begin{matriisi} x_1 \\ x_2 \\ \vdots \\ x_n \end{matriisi} + 
							\begin{matriisi} y_1 \\ y_2 \\ \vdots \\ y_n \end{matriisi} =
                            \begin{matriisi} x_1 + y_1 \\ x_2 + y_2 \\ \vdots \\ x_n + y_n \end{matriisi}
                            \\ \hline
Vektorien erotus			& \bm{x} - \bm{y} = \bm{x} + (-\bm{y}) \\ \hline
\end{taulukko}

Vektoriavaruuksien aksioomat \cite[s. 8]{MAT-60000}
\begin{align*}
(1)	\quad & \bm{x} + \bm{y} = \bm{y} + \bm{x} \\
(2)	\quad & (\bm{x} + \bm{y}) + \bm{z} = \bm{x} + (\bm{y} + \bm{z}) \\
(3)	\quad & \bm{x} + \bm{0} = \bm{0} + \bm{x} = \bm{x} \\
(4) \quad & \bm{x} + (-\bm{x}) = (-\bm{x}) + \bm{x} = \bm{0} \\
(5) \quad & \alpha ( \bm{x} + \bm{y} ) = \alpha \bm{x} + \alpha \bm{y} \\
(6) \quad & (\alpha + \beta ) \bm{x} = \alpha \bm{x} + \beta \bm{x} \\
(7) \quad & \alpha ( \beta \bm{x} ) = (\alpha \beta) \bm{x} \\
(8) \quad & 1 \bm{x} = \bm{x}
\end{align*}

\begin{taulukko}{Sisätulo ja normi \cite[s. 9-16]{MAT-60000}}
Sisätulo					& \sisatulo{x}{y} = \sum^n_{i=1} \overline{x_i} y_i = \bm{x}^* \bm{y} \\ \hline
Sisätulon perusominaisuudet	& (1) \quad \sisatulo{x}{y} \geq 0 \land (\sisatulo{x}{y} = 0 \rightarrow \bm{x} = \bm{0}) \\
							& (2) \quad \langle \bm{x} + \bm{y}, \bm{z} \rangle = \sisatulo{x}{z} + \sisatulo{y}{z} \\
                           	& (3) \quad \langle \bm{x}, \alpha \bm{y} \rangle = \alpha \sisatulo{x}{y} \\
                            & (4) \quad \sisatulo{x}{y} = \overline{\sisatulo{y}{x}} \\
                            & \forall \bm{x}, \bm{y}, \bm{z} \in \mathbb{C}^n, \alpha \in \mathbb{C} \\ \hline
Ortogonaalisuus				& \sisatulo{x}{y} = 0 \\ \hline
Kroneckerin symboli			& \delta_{ij} \begin{yhtaloryhma} 0 \quad i \neq j \\ 1 \quad i=j \end{yhtaloryhma} \\ \hline
Vektorijoukon ortogonaalisuus	& \langle \bm{x}_i, \bm{x}_j \rangle \begin{yhtaloryhma} 0 \quad i \neq j \\ \neq 0 \quad i=j \end{yhtaloryhma} \\
Vektorijoukon ortonormaalius	& \langle \bm{x}_i, \bm{x}_j \rangle = \delta_{ij} \\ \hline
Normi						& ||\bm{x}|| = \sqrt{\sisatulo{x}{x}} \\ \hline
Normin ominaisuudet			& (1) \quad || \bm{x} || \geq 0 \land (|| \bm{x} || = 0 \leftrightarrow \bm{x} = \bm{0}) \\
							& (2) \quad || \alpha \bm{x} || = | \alpha | || \bm{x} || \quad \forall \alpha \in \mathbb{C}^n \\
Kolmioepäyhtälö				& (3) \quad || \bm{x} + \bm{y} || \leq || \bm{x} || + || \bm{y} || \\ \hline
Kolmioepäyhtälö alaspäin	& || \bm{x} - \bm{y} || \geq \big| || \bm{x} || - || \bm{y} || \big| \\ \hline
Cauchy-Schwarzin epäyhtälö	& |\sisatulo{x}{y}| \leq || \bm{x} || \cdot || \bm{y} || \\ \hline
Vektorien välinen kulma		& \cos(\phi) = \frac{\sisatulo{x}{y}}{|| \bm{x} || \cdot || \bm{y} ||} \\ \hline
\end{taulukko}


\subsection{Matriisit}

\begin{taulukko}{Matriisien perusteet \cite[s. 16-26]{MAT-60000}}
Lineaarikuvaus				& L(a \bm{x} + b \bm{y}) = a L(\bm{x}) + b L({\bm{y})} \\ \hline
Matriisin indeksointi		& A=[a_{ij}]^{m\times n} = 
                			\begin{matriisi} a_{11}  & a_{12}  & \dots  & a_{1n} \\ 
							a_{21}  & a_{22}  & \dots  & a_{2n} \\ 
							\vdots  & \vdots & \ddots & \vdots \\ 
							a_{m1}  & a_{m2}  & \dots  & a_{mn} \end{matriisi} \\
                            
							& \text{m on korkeus (vaakarivien määrä)} \\
                           	& \text{n on leveys (pystyrivien määrä)} \\ \hline

Neliömatriisi				& m=n \\
Korkea matriisi				& m>n \\
Leveä matriisi				& m<n \\ \hline
Sarakevektorit				& A = \begin{matriisi} \bm{a}_1, \bm{a}_2, \ldots, \bm{a}_n\end{matriisi} \\
Vaakarivivektorit			& A = \begin{matriisi} \bm{a}^T_1 \\ \bm{a}^T_2 \\ \vdots \\ \bm{a}^T_m \end{matriisi} \\ \hline
Neliömatriisin jälki		& \text{tr}(A) = \sum^n_{i=1} a_{ii} \text{ eli diagonaalialkioiden summa} \\ \hline
Matriisien summa			& A + B = (a_{ij}) + (b_{ij}) = (a_{ij} + b_{ij}) \\
Skalaarilla kertominen		& \alpha A = (\alpha a_{ij}) \\
Matriisien erotus			& A - B = A + (-1)B = (a_{ij}) + (-b_{ij}) = (a_{ij} - b_{ij}) \\ \hline
Matriisien tulo				& C = (c_{ij})_{m \times n} = \big( \sum^p_{k=1} a_{ik} b_{kj} \big)_{m \times n} \\ \hline
Matriisien laskusäännöt		& (A+B)+C = A+(B+C) \\
							& (AB)C = A(BC) \\
                            & A(B+C) = AB + AC \\
                            & (A+B)C = AC + BC \\
                            & A + B = B + A \\ \hline
Neliömatriisin potenssit	& A^k = AA \cdots A \\ \hline
\end{taulukko}


\begin{taulukko}{Matriisityyppejä \cite[s. 18-21, 34]{MAT-60000}}
Identiteettimatriisi		& I = \begin{matriisi}
							\bm{e}_1, \bm{e}_2, \ldots, \bm{e}_n
							\end{matriisi} = 
							\begin{matriisi}
							1 & 0 & \cdots & 0 \\
                            0 & 1 & \cdots & 0 \\
                            \vdots & \vdots & \ddots & \vdots \\
                            0 & 0 & \cdots & 1 \\
							\end{matriisi} \\ \hline
Lohkomatriisi				& A =
							\begin{matriisi}
                            A_{11} & A_{12} & \cdots & A_{1q} \\
                            \vdots & & & \vdots \\
                            A_{p1} & A_{p2} & \cdots & A_{pq} \\
                            \end{matriisi} \\ \hline
                            
Diagonaalimatriisi			& A = \text{diag}(a_{11}, a_{22}, \ldots, a_{nn}) = 
							\begin{matriisi}
                            a_{11} & 0 & \cdots & 0 \\
                            0 & a_{22} & \cdots & 0 \\
                            \vdots & \vdots & \ddots & \vdots \\
                            0 & 0 & \cdots & a_{nn}
                            \end{matriisi} \\ \hline

Yläkolmiomatriisi			& (i>j \rightarrow a_{ij} = 0) \Leftrightarrow A =
							\begin{matriisi}
                            a_{11} & a_{12} & \cdots & a_{1n} \\
                            0 & a_{22} & \cdots & a_{2n} \\
                            \vdots & \vdots & \ddots & \vdots \\
                            0 & 0 & \cdots & a_{mn}
                            \end{matriisi}
							\\
Alakolmiomatriisi			& (i<j \rightarrow a_{ij} = 0) \Leftrightarrow A = 
							\begin{matriisi}
                            a_{11} & 0 & \cdots & 0 \\
                            a_{21} & a_{22} & \cdots & 0 \\
                            \vdots & \vdots & \ddots & \vdots \\
                            a_{m1} & a_{m2} & \cdots & a_{mn}
                            \end{matriisi}
                            \\ \hline
Lohkodiagonaalinen matriisi	& A = diag(A_{11}, A_{22}, \ldots, A_{pp})
							\begin{matriisi}
                            A_{11} & O & \cdots & O \\
                            O & A_{22} & \cdots & O \\
                            \vdots & \vdots & \ddots & \vdots \\
                            O & O & \cdots & A_{pp}
                            \end{matriisi} \\ \hline
Lohkoalakolmiomatriisi		& A = 
							\begin{matriisi}
                            A_{11} & A_{12} & \cdots & A_{1q} \\
                            O & A_{22} & \cdots & A_{2q} \\
                            \vdots & \vdots & \ddots & \vdots \\
                            O & O & \cdots & A_{pq}
                            \end{matriisi}
							\\
Lohkoyläkolmiomatriisi		& A =
							\begin{matriisi}
                            A_{11} & O & \cdots & O \\
                            A_{21} & A_{22} & \cdots & O \\
                            \vdots & \vdots & \ddots & \vdots \\
                            A_{p1} & A_{p2} & \cdots & A_{pq}
                            \end{matriisi}
							\\ \hline
Permutaatiomatriisi			& P = \begin{matriisi}
							\bm{e}^T_{r_1} \\ \bm{e}^T_{r_1} \\ \vdots \\ \bm{e}^T_{r_n}
                            \end{matriisi}
							= \text{esim.} \begin{matriisi} 0&0&1 \\ 0&1&0 \\ 1&0&0 									\end{matriisi} \\ \hline
\end{taulukko}


\begin{taulukko}{Transpoosi ja konjugaattitranspoosi \cite[s. 21, 26]{MAT-60000}}
Transpoosi					& A^T = (a_{ji})_{n \times m} =
							\begin{matriisi}
                            a_{11} & a_{21} & \cdots & a_{m1} \\
                            a_{12} & a_{22} & \cdots & a_{m2} \\
                            \vdots & \vdots & \ddots & \vdots \\
                            a_{1n} & a_{2n} & \cdots & a_{nm} \\
                            \end{matriisi} \\

Konjugaattitranspoosi       & A^* = (\overline{a}_{ji})_{m \times n} =
							\begin{matriisi}
                            \overline{a}_{11} & \overline{a}_{21} & \cdots & \overline{a}_{m1} \\
                            \overline{a}_{12} & \overline{a}_{22} & \cdots & \overline{a}_{m2} \\
                            \vdots & \vdots & \ddots & \vdots \\
                            \overline{a}_{1n} & \overline{a}_{2n} & \cdots & \overline{a}_{nm} \\
                            \end{matriisi}\\ \hline

Transpoosille				& (AB)^T = B^T A^T \\
							& (A+B)^T = A^T + B^T \\
                            & (\alpha A)^T = \alpha A^T \\ \hline

Konjugaattitranspoosille	& (AB)^* = B^* A^* \\
							& (A+B)^* = A^* + B^* \\
							& (\alpha A)^* = \overline{\alpha} A^* \\ \hline
\end{taulukko}


\begin{taulukko}{Matriisien ominaisuuksia \cite[s. 27-30]{MAT-60000}}
Kommutoivuus				& AB = BA \\ \hline

Symmetrisyys				& A = A^T \\
Hermiittisyys				& A = A^* \\
Vinosymmetrisyys			& A = -A^T \\
Vinohermiittisyys			& A = -A^* \\ \hline

Singulaarisuus				& det(A) = 0 \\
Ei-singulaarisuus			& det(A) \neq 0 \\
							& \exists A^{-1} \\ \hline

Ortogonaalisuus				& A^T A = I = AA^T \\
Unitaarisuus				& U^* U = U U^* = I \\ \hline

Inverssi					& A A^{-1} = I = A^{-1} A \\
							& (AB)^{-1} = B^{-1} A^{-1} \\
                           	& (A^{-1})^{-1} = A \\ \hline
\end{taulukko}


\begin{taulukko}{Lineaarinen yhtälöryhmä \cite[s. 31]{MAT-60000}}
Lineaarinen yhtälöryhmä		& A \bm{x} = \bm{b} \\
							& \bm{x} = A^{-1} \bm{b} \\ \hline
\end{taulukko}


\begin{taulukko}{Hermiittisille matriiseille \cite[s. 35]{MAT-60000}}
Positiivisesti definiitti		& \langle \bm{x}, A \bm{x}\rangle > 0, \forall \bm{x} \neq \bm{0} \\
Positiivisesti semidefiniitti	& \langle \bm{x}, A \bm{x}\rangle \geq 0, \forall \bm{x} \\
Negatiivisesti definiitti		& \langle \bm{x}, A \bm{x}\rangle < 0, \forall \bm{x} \neq \bm{0} \\
Negatiivisesti semidefiniitti	& \langle \bm{x}, A \bm{x}\rangle \leq 0, \forall \bm{x} \\
Indefiniitti					& \text{Jos ei ole mitään näistä} \\ \hline
\end{taulukko}


\begin{taulukko}{LU-hajotelma \cite[s. 45-56]{MAT-60000}}
LU-hajotelma	& A = LU \\
				& L_k = I - \bm{l}_k \bm{e}^T_k \\
                & \bm{l}_k = \frac{1}{x_k} \begin{matriisi} 0 \\ \vdots \\ 0 \\ x_{k+1} \\ \vdots \\ x_n \end{matriisi}, x_k \neq 0 \\
                & L_{n-1} L_{n-2} \cdots L_1 A = U \\
                & \hat{L} = L_{n-1} L_{n-2} \cdots L_1 \\
                & A = \hat{L}^{-1} U = LU \\ \hline
Jos ei onnistu suoraan, niin viimeistään	& PA = LU, \text{ jossa P on permutaatiomatriisi} \\ \hline
Lineaarisen yhtälöryhmän ratkaiseminen	& A \bm{x} = P^T LU \bm{x} = \bm{b} \\
											& LU \bm{x} = P \bm{b} \\
                                            & L \bm{y} = P \bm{b} \land U \bm{x} = \bm{y} \\
                                            & x = U^{-1} \bm{y} \land \bm{y} = L^{-1} P \bm{b} \\ \hline
\end{taulukko}


\begin{taulukko}{Aliavaruus \cite[s. 59-83]{MAT-60000}}
Aliavaruus					& \alpha \bm{x} + \beta \bm{y} \in S \quad \forall \bm{x}, \bm{y} \in \mathcal{S} \\
Arvojoukko (kuva-avaruus)	& \mathcal{R} (A) = \{y | \exists \bm{x} \in \mathbb{F}^n s.e. \bm{y} = A \bm{x} \} \in \mathbb{F}^m \\
Ydin (nolla-avaruus)		& \mathcal{N} (A) = \{ \bm{x} | A \bm{x} = \bm{0} \} \\ \hline

Ortogonaalikomplementti		& \mathcal{S}^\perp = \{ \bm{y} | \langle \bm{y}, \bm{x} \rangle = 0, \forall \bm{x} \in \mathcal{S} \} \\ \hline

							& \mathcal{N} (A) = \mathcal{R} (A^*)^\perp \\
                            & \mathcal{N} (A^*) = \mathcal{R} (A)^\perp \\ \hline

Dimensio eli kantavektorien lukumäärä	& \dim (S) \\
Matriisin aste							& rank (A) = \dim \mathcal{R} (A) \\
										& rank (A) + \dim \mathcal{N} (A) = n = \text{ sarakkeiden määrä} \\ \hline

Projektorimatriisi			& P^2 = P \\
							& \text{Projisoi vektorit } \mathcal{R} (P) \text{:lle pitkin } \mathcal{N} (P) \text{:tä} \\ \hline
\end{taulukko}


\begin{taulukko}{Ominaisarvot ja -vektorit \cite[s. 90-100]{MAT-60000}}
Ominaisarvot ja -vektorit	& A \bm{x} = \lambda \bm{x}, \bm{x} \neq \bm{0} \\
Ominaisarvot				& \det (A - \lambda I) = 0 \\
Karakteristinen polynomi	& \prod^n_{k=1} (\lambda - \lambda_k) \\
Spektri						& \sigma (A) = \{ \lambda_ 1, \lambda_2, \ldots , \lambda_n \} \\
Similaarisuus				& B = S^{-1} A S \text{ jossa S ei-singulaarinen}\\
Unitaarinen similaarisuus	& B = U^* AU \\

Hermiten matriisille		& diag(\sigma(A)) = U^* A U \\
							& \forall \lambda \in \mathbb{R} \land n \text{ ortonormaalia ominaisvektoria} \Rightarrow \text{Hermiten matriisi} \\ \hline

Normaali matriisi			& A = UDU^* \text{ jossa D on diagonaalimatriisi} \\ \hline
\end{taulukko}


\begin{taulukko}{Jordanin kanoninen muoto \cite[s. 101-107]{MAT-60000}}
				& A = SJS^{-1} \\
				& J = diag(J_1, J_2, \ldots, J_p) \\
Defektiivisyys	& \exists \lambda : geom(\lambda) < alg(\lambda) \\ \hline
				& A \bm{s}^j_1 = \lambda_j \bm{s}^j_1 \\
                & A \bm{s}^j_2 = \lambda_j \bm{s}^j_2 \bm{s}^j_1 \\
                & A \bm{s}^j_i = \lambda_j \bm{s}^j_i \bm{s}^j_{i-1} \\
Jordanin ketju	& \{ \bm{s}^j_1, \bm{s}^j_2, \ldots \bm{s}^j_{r_j} \} \\
Ominaisvektori	& \bm{s}^j_1 \\
Yleistetyt ominaisvektorit	& \bm{s}^j_2, \ldots \bm{s}^j_{r_j} \\
Kompleksiselle ominaisarvolle	& J_i = \begin{matriisi} \alpha & \beta \\ - \beta & \alpha \end{matriisi} \in \mathbb{R}^{2 \times 2} \\ \hline
Käyttöohje		& \text{Etsi } \sigma (A) \\
				& \text{Valitse mielivaltainen } \lambda \\
                & alg(\lambda) = 1 \Rightarrow \bm{s}^1_1 = \bm{x}_1 \land J_1 = \lambda_1 \\
                & alg(\lambda) = geom(\lambda) = l \Rightarrow \text{Otetaan ominaisvektorit} \land \forall J = \lambda_1 \\
                & alg(\lambda) > geom(\lambda) \Rightarrow \text{Muodostetaan ketjut erikseen} \\
               	& \lambda \text{ kompleksinen} \Rightarrow (\bm{s}^1_1 = \text{Re } \bm{x}, \bm{s}^1_2 = \text{Im } \bm{x}) \land (\overline{\lambda} \in \sigma(A)) \\ \hline
\end{taulukko}


\begin{taulukko}{Singulaariarvohajotelma \cite[s. 117-128]{MAT-60000}}
Singulaariarvohajotelma		& U^*AV = \Lambda \\
							& A = U \Lambda V^* \\
							& \Lambda = diag(\sigma_1, \sigma_2, \ldots , \sigma_r, 0, \ldots, 0) \\
                            & \sigma(A^*A) = \{ \sigma^2_1, \sigma^2_2, \ldots, \sigma^2_{r}, \sigma^2_{r+1}, \sigma^2_n \} \\
                            & V = [\bm{v}_1, \bm{v}_2, \ldots, \bm{v}_n] \text{ jossa v:t ortonormaaleja ominaisvektoreita} \\
                            & \bm{u}_i = \frac{A \bm{v}_i}{\sigma_i} \\
                            & \text{Toimii samoin, vaikka valittaisiin } AA^* \\ \hline
                            & rank(A) = r \\
                            & \mathcal{R}(A) = span\{ \bm{u}_1, \bm{u}_2, \ldots, \bm{u}_r \} \\
                            & \mathcal{N}(A) = span\{ \bm{v}_{r+1}, \bm{v}_{r+2}, \ldots, \bm{v}_n \} \\ \hline
                            
                            & || A || = \sigma_1 \\
Jos A on ei-singulaarinen	& \sigma_n || \bm{x} || \leq || A \bm{x} || \leq \sigma_1 || \bm{x} ||, \forall \bm{x} \in \mathbb{F}^n \\
							& || A^{-1} || = \frac{1}{\sigma_n} \\ \hline

Approksimointi				& B = U diag(\sigma_1, \sigma_2, \ldots, \sigma_{r-1}, 0, \ldots, 0) V^* \\
							& ||A - B|| = \sigma_r \\ \hline

Pseudoinverssi				& A^\dagger = V \Lambda^\dagger U^*\\
							& \Lambda^\dagger \text{ on } \Lambda^T, \text{jossa } \sigma_i \Rightarrow \frac{1}{\sigma_i} \\
                            & AA^\dagger A = A \\
                            & A^\dagger A A^\dagger = A^\dagger \\
Lin. yhtälöryhmän 			& A\bm{x} = \bm{b} \\
yleinen ratkaisu			& \bm{x} = A^\dagger \bm{b} \\ \hline
\end{taulukko}



\part{Fysiikka}

\section{Suureita ja yksiköitä}

\begin{table}[ht!]
\centering
\caption{\href{https://en.wikipedia.org/wiki/SI_base_unit}{SI-järjestelmän perusyksiköt} \cite[s. 64]{MAOL} }
\begin{tabular}{l|>{$} l <{$}|l|l}
Suure	& \text{Tunnus}	& Yksikkö	& Yksikön tunnus \\
\hline
pituus		& l, s	& metri			& m \\
massa		& m		& kilogramma	& kg \\
aika		& t		& sekunti		& s \\
sähkövirta	& I		& ampeeri		& A \\
lämpötila	& T		& kelvin		& K \\
ainemäärä	& n		& mooli			& mol \\
valovoima	& I		& kandela		& cd \\
\end{tabular}
\end{table}


\begin{table}[ht!]
\centering
\setlength{\extrarowheight}{2pt}
\caption{\href{https://en.wikipedia.org/wiki/Metric_prefix}{Kerrannaisyksiköiden etuliitteet} \cite[s. 65]{MAOL} }
\begin{tabularx}{\textwidth}{X|X| >{$} X <{$} |X|X| >{$} X <{$} }
Nimi	& Tunnus	& \text{Kerroin}	& Nimi	& Tunnus	& \text{Kerroin} \\
\hline
deka	& da	& 10^1		& desi		& d		& 10^{-1} \\
hehto	& h		& 10^2		& sentti	& c		& 10^{-2} \\
kilo	& k		& 10^3		& milli		& m		& 10^{-3} \\
mega	& M		& 10^6		& mikro		& $\mu$	& 10^{-6} \\
giga	& G		& 10^9		& nano		& n		& 10^{-9} \\
tera	& T		& 10^{12}	& piko		& p		& 10^{-12} \\
peta	& P		& 10^{15}	& femto		& f		& 10^{-15} \\
eksa	& E		& 10^{18}	& atto		& a		& 10^{-18} \\
tsetta	& Z		& 10^{21}	& tsepto	& z		& 10^{-21} \\
jotta	& Y 	& 10^{24}	& jokto		& y		& 10^{-24} \\
\hline
\end{tabularx}
\end{table}


\begin{table}[ht!]
\centering
\setlength{\extrarowheight}{5pt}
\caption{\href{https://en.wikipedia.org/wiki/SI_base_unit}{Perusyksiköiden määritelmät} \cite[s. 64-65]{MAOL} }
\begin{tabularx}{\textwidth}{lX}
\hline
metri (m)		& Metri on sellaisen matkan pituus, jonka valo kulkee tyhjiössä aikavälissä 1/299 792 458 sekuntia. \newline (1983, 17. CGPM) \\

kilogramma (kg)	& Kilogramma on yhtä suuri kuin kansainvälisen kilogramman prototyypin massa. \newline (1889 ja 1901, 1. ja 3. CGPM) \\

sekunti (s)		& Sekunti on 9 192 631 770 kertaa sellaisen säteilyn jaksonaika, joka vastaa cesium-133-atomin siirtymää perustilan ylihienorakenteen kahden energiatason välillä. \newline (1967, 13. CGPM) \\

ampeeri (A)		& Ampeeri on sellainen ajallisesti muuttumaton sähkövirta, joka kulkiessaan kahdessa suorassa yhdensuuntaisessa, äärettömän pitkässä ja ohuessa johtimessa, joiden poikkileikkaus on ympyrä ja jotka ovat 1 metrin etäisyydellä toisistaan tyhjiössä, saa aikaan johtimien välille $2 \cdot 10^{-7}$ newtonin voiman johtimen metriä kohti. \newline (1948, 9. CGPM) \\

kelvin (K)		& Kelvin, termodynaamisen lämpötilan yksikkö, on 1/273,16 veden kolmoispisteen termodynaamisesta lämpötilasta. \newline (1967, 13. CGPM) \\

mooli (mol)		& Mooli on sellaisen systeemin ainemäärä, joka sisältää yhtä monta keskenään samanlaista perusosasta kuin on atomeja 0,012 kilogrammassa hiili-12:ta. Perusosaset voivat olla atomeja, molekyylejä, ioneja, elektroneja, muita hiukkasia tai sellaisten hiukkasten määriteltyjä ryhmiä. \newline (1971, 14. CGPM) \\

kandela (cd)	& Kandela on sellaisen säteilijän valovoima, joka lähettää tiettyyn suuntaan monokromaattista $540 \cdot 10^{12}$ hertsin taajuista säteilyä ja jonka säteilyintensiteetti tähän suuntaan on 1/683 wattia steradiaania kohti. \newline (1979, 16. CGPM) \\
\end{tabularx}

\end{table}


\clearpage

\section{Vakioita}

\begin{consttable}{Universaalit luonnonvakiot \cite[s. 70-71]{MAOL} }
valon nopeus tyhjiössä		& c, c_0		& 2,9979 2458 $\cdot 10^8$ m/s \\
gravitaatiovakio			& \gamma, G		& 6,674 28 $\cdot 10^{-11}$ Nm$^2$/kg$^2$ \\
Planckin vakio				& h				& 6,626 0693 $\cdot 10^{-34}$ Js \\
							& 				& 4,135 6654 $\cdot 10^{-15}$ eVs \\
redusoitu Planckin vakio (Diracin vakio)	& \hbar = \frac{h}{2 \pi}	& 1,054 5717 $\cdot 10^{-34}$ Js \\
											&							& 6,582 1159 $\cdot 10^{-16}$ eVs \\ % calculated h (eVs) / 2 \pi
\end{consttable}

\begin{consttable}{Sähkömagneettisia vakioita \cite[s. 70-71]{MAOL} }
tyhjiön permeabiliteetti (magneettivakio)	& \mu_0	& $4 \pi \cdot 10^{-7} \frac{\text{Vs}}{\text{Am}}$ \\
											& 		& 1,256 637 061 $\cdot 10^{-6}$ H/m \\
tyhjiön permittiivisyys (sähkövakio)		& \epsilon_0 = \frac{1}{\mu_0 c^2}	& 8,854 187 818 $\cdot 10^{-12}$ F/m \\
Coulombin lain vakio						& k_e = \frac{1}{4 \pi \epsilon_0}	& 8,987 551 787 $\cdot 10^9$ Nm$^2$/C$^2$ \\
alkeisvaraus								& e		& 1,602 176 487 $\cdot 10^{-19}$ C \\
\end{consttable}

\begin{consttable}{Atomi- ja ydinfysiikan vakioita \cite[s. 70-71]{MAOL} }
elektronin lepomassa	& m_e	& 9,109 3822 $\cdot 10^{-31}$ kg \\
						&		& 5,485 7991 $\cdot 10^{-4}$ u \\
protonin lepomassa		& m_p	& 1,672 6216 $\cdot 10^{-27}$ kg \\
						&		& 1,007 2765 u \\
neutronin lepomassa		& m_n	& 1,674 9273 $\cdot 10^{-27}$ kg \\
						&		& 1,008 6650 u \\
deuteronin lepomassa	& m_d	& 3,343 5835 $\cdot 10^{-27}$ kg \\
						&		& 2,013 5532 u \\
alfahiukkasen lepomassa	& m_\alpha	& 6,644 656 $\cdot 10^{-27}$ kg \\
						&		& 4,001 5062 u \\
Rydbergin vakio	\cite[s. 171]{ModernPhysics}		& R_\infty	& 1,097 3732 $\cdot 10^7$ 1/m \\
\end{consttable}

\begin{consttable}{Sekalaisia vakioita \cite[s. 70-71]{MAOL} }
Faradayn vakio				& F			& 96 485,338 C/mol \\
Stefanin-Boltzmannin vakio	& \sigma	& 5,670 400 $\cdot 10^{-8} \frac{\text{W}}{\text{m}^2 \text{K}^4} $ \\
Wienin siirtymislain vakio	& b			& 2,897 768 $\cdot 10^{-3}$ m$\cdot$K \\
Boltzmannin vakio			& k_b		& 1,380 6505 $\cdot 10^{-23}$ J/K \\
moolinen kaasuvakio			& R			& 8,314 510 $\frac{\text{Pa} \cdot \text{m}^3}{\text{mol} \cdot \text{K}} $ \\
							&			& 0,083 145 10 $\frac{\text{bar} \cdot \text{dm}^3}{\text{mol} \cdot \text{K}} $ \\
\end{consttable}


\begin{table}
\centering
\setlength{\extrarowheight}{2pt}
\caption{Muuntokertoimia, osa 1 \cite{MAOL}}
\begin{tabular}{l|ll}
\hline
pituus	& 1""	& = 1 in = tuuma = 25,40 mm \\
		& 1'	& = 1 ft = 1 jalka = 0,3048 m \\
		& 1 mi	& = 1 maili = 1,609 344 km \\
		& 1 Å	& = 1 ångström = $10^{-10}$ m = 0,1 nm \\
		& 1 mpk	& = 1 M = 1 meripeninkulma = 1 852 m \\
		% Light-year is defined using the julian year, not the gregorian one. MAOL has a wrong value. During the TUT astrophysics course (FYS-1670) of 2016, the lecturer Antti Vuorimäki recommended students to replace the wrong value from their table books with the correct one.
		& 1 ly	& = 1 valovuosi = 9,460 730 472 5808 Pm \quad \cite{IAU-Measuring} \\
		& 1 pc	& = 1 parsek = 30,85678 Pm \\
		& 1 au	& = 1 tähtitieteellinen yksikkö = 149,597 870 700 Gm \quad \cite{IAU-Measuring} \\
\hline
massa	& 1 ka	& = 1 karaatti = 200 mg \\
		& 1 u	& = 1,660 5389 $\cdot 10^{-27}$ kg \\
		& 1 lb	& = 1 naula = 0,4536 kg \\
		& 1 oz	& = 1 unssi = 28,35 kg \\
\hline
pituusmassa	& 1 denier	& = 0,111 mg/m \\
			& 1 tex		& = 1 mg/m \\
\hline
aika	& 1 d	& = 86400 s \\
		& 1 a	& = 31,536 $\cdot 10^6$ s \\
\hline
tasokulma	& 1\degree	& = $2 \pi /360$ rad \\
			& 1'		& = 1 kulmaminuutti = 1/60\degree \\
			& 1''		& = 1 kulmasekunti = 1/3600\degree \\
			& 1 h		& = 1 tunti = $2 \pi / 24$ rad \\
\hline
pinta-ala	& 1 b		& = 1 barni = $10^{-28}$ m$^2$ \\
			& 1 acre	& = 1 eekkeri = 4,0469$\cdot 10^3$ m$^2$ \\
\hline
tilavuus	& 1 l		& = 1 dm$^3$ \\
			& 1 bbl		& = 1 barreli = 0,158 9873 m$^3$ \\
			& 1 gal		& = 1 gallona (UK) = 4,546 092 l \\
			& 1 gal		& = 1 gallona (US) = 3,785 412 l \\
			& 1 rt		& = 1 rekisteritonni = 2,831 685 m$^3$ \\
			& 1 std		& = 1 standartti = 4,672 280 m$^3$ \\
\hline
nopeus		& 1 solmu	& = 1 mpk/h = 1,852 km/h = 0,5144 m/s \\
			& 1 mach	& = 1 M = 311 m/s \\
\hline
voima		& 1 kp		& = 9,806 65 N \\
			& 1 dyn		& = 10 $\mu$N \\
\hline
\end{tabular}
\end{table}


\begin{table}
\centering
\setlength{\extrarowheight}{2pt}
\caption{Muuntokertoimia, osa 2 \cite{MAOL}}
\begin{tabular}{l|ll}
\hline
työ, energia	& 1 erg	& = 0,1 $\mu$J \\
			& 1 eV		& = 1,602 1773$\cdot 10^{-19}$ J \\
			& 1 cal		& = 4,1868 J \\
			& 1 kWh		& = 3,6 MJ \\
			& 1 kpm		& = 9,806 65 J \\
\hline
\multirow{2}{3.2cm}{massan ja energian vastaavuus}
	& 1 u	& = 149,241 91 pJ/c$^2$ = 931,494 32 MeV/c$^2$ \\
	& 1 kg	& = 89,875 517 PJ/c$^2$ \\
\hline
teho	& 1 kpm/s	& = 9,806 65 W \\
		& 1 hv		& = 735,5 W \\
		& 1 kcal/h	& = 1,163 W \\
\hline
paine	& 1 bar		& = 10$^5$ Pa \\
		& 1 at		& = 1 kp/cm$^2$ = 98,0665 kPa \\
		& 1 atm		& = 760 torr = 101,325 kPa \\
		& 1 mmH$_2$O	& = 9,806 65 Pa \\
		& 1 psi		& = 1 lbf/in$^2$ = 6,894 757$\cdot 10^3$ Pa \\
\hline
viskositeetti	& 1 P	& = 1 poisi = 0,1 Ns/m$^2$ \\
\hline
taittovoimakkuus	& 1 d	& = 1 diptri = 1/m \\
\hline
magneettivuo		& 1 Mx	& = 1 maxwell = 10 nWb \\
\hline
\multirow{2}{3cm}{Magneettivuon \newline tiheys}
	& 1 G	& = 1 gauss = 0,1 mT \\
	\\
\hline
\multirow{2}{3cm}{Magneettikentän \newline voimakkuus}
	& 1 Oe	& = 1 örsted = 10$^3 / 4 \pi$ A/m = 79,577 A/m \\
	\\
\hline
\multirow{2}{3.2cm}{Aktiivisuus (ionisoiva säteily)}
	& 1 Ci	& = 1 curie = $3,7 \cdot 10^10$ 1/s = 37 GBq \\
	\\
\hline
säteilytys	& 1 R	& = 1 röntgen = 0,258 mC/kg (ilmassa) \\
\hline
absorboitunut annos	& 1 rd	& = 1 rad = 0,01 J/kg = 0,01 gray = 10 mGy \\
\hline
ekvivalenttiannos	& 1 rem	& = 0,01 J/kg = 0,01 sievert = 10 mSv \\
\end{tabular}
\end{table}

\clearpage

\section{Kaavoja}

\subsection{Mekaniikka}



\begin{eqtable-units}{Etenemisiike}
\textbf{Etenemisliike} &&& \\
matka	&	x, s, r	& m		& x = vt \\
nopeus	&	v	& m/s	& v = \frac{dx}{dt} \\
kiihtyvyys	&	a	& m/s$^2$	& a = \frac{dv}{dt} = \frac{d^2x}{dt^2} \\
liikemäärä	& p	& kgm/s	& p = mv \\
Newton II	&&& \frac{d \bm{p}}{dt} = \bm{F} \Leftrightarrow \bm{a} = \frac{\bm{F}}{m} \\
työ			& W	& J	& W = \int \bm{F} \cdot \bm{dx} \\
kineettinen energia	& E_k, K	& J	& E_k = \frac{1}{2}mv^2 \\
potentiaalienergia	& E_p, U	& J	& E_p = \frac{1}{2}kx^2 \\
\hline
\textbf{Tasaisesti muuttuva etenemisliike} &&& \\
loppunopeus	& v	& m/s	& v = v_0 + at \\
paikka		& x	& m		& x = x_0 + v_0 t + \frac{1}{2} at^2 \\
\end{eqtable-units}

\begin{eqtable-units}{Pyörimisliike}
\textbf{Pyörimisliike} &&& \\
kaaren pituus	& s			& m		& s = r \theta \\
%								&&& \Delta \bm{s} = \Delta \bm{\theta} \times \bm{r} \\
kulmanopeus		& \omega	& rad/s	& \bm{\omega} = \frac{d \bm{\theta}}{dt} \\
kulmakiihtyvyys	& \alpha	& rad/s		& \bm{\alpha} = \frac{d \bm{\omega}}{dt} = \frac{d^2 \bm{\theta}}{dt^2}\\
ratanopeus		& v			& m/s	& v = \bm{\omega} \times \bm{r} \\
kierrosaika		& T			& s		& T = \frac{2 \pi}{\omega} \\
kierrostaajuus	& f, n		& 1/s, Hz	& f = \frac{1}{T} \\
tangenttikiihtyvyys	& a_t	& m/s$^2$	& a_t = r \alpha \\
normaalikiihtyvyys	& a_n	& m/s$^2$	& a_n = \frac{v^2}{r} \\
kiihtyvyys			& a		& m/s$^2$	& a_r \hat{\bm{r}} + a_t \hat{\bm{t}} \\
työ					& W	& J	& W = \int \bm{T} \cdot \bm{d\theta} \\
kulmaliikemäärä	& L	& kgm$^2$/s	& \bm{L} = \bm{r} \times \bm{p} \\
kineettinen energia	&E_k, K	& J	& E_k = \frac{1}{2} I \omega^2 \\
potentiaalienergia	&E_p, U	& J	& E_p = \frac{1}{2} c \theta^2 \\
\hline
\textbf{Tasaisesti muuttuva pyörimisliike} &&& \\
loppukulmanopeus	& \omega & rad/s	& \omega = \omega_0 + \alpha t \\
kiertokulma			& \theta	& rad	& \theta = \theta_0 + \omega_0 t + \frac{1}{2} \alpha t^2 \\
\end{eqtable-units}


\begin{eqtable-units}{Voima, energia}
\textbf{Voima} & F	& N	& \\
Newtonin gravitaatiolaki	&&& \bm{F} = -G \frac{m_1 m_2}{r^2} \hat{\bm{r}} \\
homogeeninen gravitaatiokenttä	&&& \bm{g} = \frac{\bm{F}}{m} \\
liikekitka	& F_\mu	&& F_\mu = \mu N \\
harmoninen voima	&&& F = -kx \\
\hline
\hline
impulssi	& I	& Ns	& \bm{I} = \int \bm{F} dt \\
teho		& P	& W		& P = \frac{dW}{dt} \\
\hline
\textbf{Potentiaalienergia} & E_p	& J	& \\
gravitaatiokenttä	&&& E_p = mgh \\
					&&& E_p = -G \frac{m_1 m_2}{r} \\
harmoninen voimakenttä	&&& E_p = \frac{1}{2} kx^2 \\
\hline
\textbf{Kineettinen energia}	& E_k, K	& J & \\
etenevän liikkeen energia	&&& E_k = \frac{1}{2} mv^2 \\
pyörimisenergia				&&& E_k = \frac{1}{2} J \omega^2 \\
\hline
mekaaninen hyötysuhde	& \eta	&& \eta = \frac{E_a}{E_o} = \frac{P_a}{P_o} \\
\hline
\textbf{Harmoninen värähtelijä} &&& \\
poikkeama	&&& x(t) = A \sin(\omega t + \phi ) \\
jaksonaika	&&& T = 2 \pi \sqrt{\frac{m}{k}} \\
\end{eqtable-units}


\begin{eqtable-units}{Heilureita ja hitausmomentteja}
\textbf{Heilahdusaika} &&& \\
matemaattinen heiluri	&&& T = 2 \pi \sqrt{\frac{l}{g}} \\
fysikaalinen heiluri	&&& T = 2 \pi \sqrt{\frac{I_A}{mgl}} \\
kiertoheiluri			&&& T = 2 \pi \sqrt{\frac{J}{D}} \\
\hline
voiman momentti	& M	& Nm	& \bm{M} = \bm{r} \times \bm{F} = \frac{d\bm{L}}{dt} \\
pyörimisen liikeyhtälö	&&& \sum M = I \alpha \\
impulssimomentti	& I	& kgm$^2$/s	& I_M = \Delta L = M \Delta t \\ \hline
\textbf{Hitausmomentteja}	& I, J	& kgm$^2$ & \\
pistemäinen kappale		&&& I = mr^2 \\
umpinainen sylinteri	&&& I = \frac{1}{2} mr^2 \\
ohutseinäinen rengas	&&& I = mr^2 \\
paksuseinäinen rengas	&&& I = \frac{1}{2}m(r^2_1+r^2_2) \\
ohut sauva (pään ympäri) &&& I = \frac{1}{3}ml^2 \\
ohut sauva (keskipisteen ympäri)	&&& I = {1}{12}ml^2 \\
suorakulmainen levy	&&& I = \frac{1}{12}m(a^2+b^2) \\
umpinainen pallo	&&& I = \frac{2}{5} mr^2 \\
ohutseinäinen pallo	&&& I = \frac{2}{3}mr^2 \\ \hline
Steinerin sääntö (akselin siirto)	&&& I_A = I_P + ma^2 \\
\end{eqtable-units}

\begin{eqtable-units}{Jatkumon mekaniikkaa \cite[TESTI]{MAOL}}
tiheys	& \rho & kg/m$^3$	& \rho = \frac{m}{V} \\
jännitys	& \sigma & N/$^2$ & \sigma = \frac{F}{A} \\
Hooken laki, kimmoisuus		& E	& N/m$^2$	& \frac{F}{A} = E \frac{\Delta l}{l} \\
paine	& p	& Pa	& p = \frac{F}{A} \\
hydrostaattinen paine	& p	& Pa	& p = h \rho g \\
noste		& F_N	& N	& F_N = \rho V g \\
\hline
\textbf{Pintajännitys} & \sigma	& N/m, J/m$^2$ & \\
voima	& F	& N	& F = \sigma l \\
energia	& E	& J	& E = \sigma A \\
\hline
\textbf{Viskositeetti} & \eta & Ns & \\
voima	& F	& N	& F = \frac{\eta A v}{d} \\
\hline
Bernoullin yhtälö	&&& p_0 + \frac{1}{2} \rho v^2 + h \rho g = vakio \\
\end{eqtable-units}


\begin{eqtable-units}{Aaltoliike ja valo-oppi \cite{MAOL}}
Aaltoliikkeen nopeus					&&& v = f \lambda \\
huojuntataajuus							&&& f = |f_1 - f_2 | \\
intensiteetti			& I	& W/m$^2$	& I = \frac{P}{A} \\
energiatiheys			& w	& J/m$^3$	& w = \frac{I}{v}, \quad w = kf^2A^2 \\
\hline
\textbf{Dopplerin ilmiö} &&& \\
aaltolähde liikkuu		&&& f = f_0 \frac{v}{v \pm v_1} \\
havaitsija liikkuu		&&& f = f_0 \frac{v \pm v_h}{v} \\
\hline
äänen nopeus kaasussa	& v	&& \frac{v_1}{v_2} = \sqrt{\frac{T_1}{T_2}} \\
äänen intensiteettitaso	& L	& dB	& L = 10 \log_10 \frac{I}{I_0} dB, \quad I_0 = 1 \text{pW/m}^2 \\
taittumislaki	&&& \frac{\sin \alpha_1}{\sin \alpha_2} = \frac{v_1}{v_2} = \frac{n_2}{n_1} = n_{12} \\
Brewsterin laki	&&& \tan \alpha_B = \frac{n_2}{n_1} \\
% hilayhtälö	&&& d \sin \alpha = k \lambda \\
kuvausyhtälö	&&& \frac{1}{a} + \frac{1}{b} = \frac{1}{f} \\
taittovoimakkuus	& D	& 1/m = d	& D = \frac{1}{f} \\
viivasuurennus		& m	&& m = \big| \frac{b}{a} \big| \\
kulmasuurennus		& M	&& M = \frac{\tan \alpha_2}{\tan \alpha_1} \\
\hline
\textbf{Suurennuksia} &&& \\
suurennuslasi	&&& M = \frac{s}{f} \\
mikroskooppi	&&& M = \frac{Ls}{f_{ob} f_{ok}} \\
kaukoputki		&&& M = \frac{f_{ob}}{f_{ob}} \\
\hline
valovoima	& I	& cd	& I = \frac{\Phi}{\omega} \\
luminanssi	& L	& cd/m$^2$	& L = \frac{I}{A} \\
valovirta	& \Phi	& lm	& \Phi = I \omega \\
valaistusvoimakkuus	& E	& lx	& E = \frac{\Phi}{A} \\
\end{eqtable-units}



\subsection{Sähkömagnetismi}

\begin{eqtable}{Sähkömagnetismi \cite{UPhysics}}
\textbf{Maxwellin yhtälöt} & \\
Gaussin laki sähkökentille		& \oiint_S \bm{D} \cdot d\bm{A} = \sum q \\
Gaussin laki magneettikentille	& \oiint_S \bm{B} \cdot d\bm{A} = 0 \\
Ampere-Maxwell					& \oint_C \bm{H} \cdot d\bm{l} = I + \frac{d}{dt} \iint_S \bm{D} \cdot d\bm{A} \\
Faradayn laki					& \oint_C \bm{E} \cdot d\bm{l} = - \frac{d}{dt} \iint_S \bm{B} \cdot d\bm{A} \\
\hline
& E = vB \\
aaltoyhtälö	& \frac{\partial^2 H}{\partial z^2} = \mu \epsilon \frac{\partial^2 H}{\partial t^2} \\
\end{eqtable}



\subsection{Suhteellisuus}

\begin{table}[ht!]
\centering
\caption{Suhteellisuus \cite{UPhysics}}
\begin{tabular}{| >{$\displaystyle} l <{$} | >{$\displaystyle} l <{$} |} \hline
\textbf{Klassinen suhteellisuus} & \textbf{Suppea suhteellisuusteoria} \\ \hline
x' = x + vt	& x' = \gamma (x+vt), \quad \gamma = \frac{1}{\sqrt{1 - (\frac{v}{c})^2}} \\ 
t' = t		& t' = \gamma (t + \frac{v}{c^2} x) \\
l = l'		& l = \frac{l'}{\gamma} \\
t = t'		& t = \gamma t' \\
u_x = u'_x + v, \quad u_y = u'_y	& u_x = \frac{u'_x+v}{1 + \frac{u'_x v}{c^2}}, \quad u_y = \frac{u'_y}{1+\frac{u'_xv}{c^2}} \\
\bm{p} = m\bm{u}	& \bm{p} = \gamma m \bm{u} \\
E = \frac{p^2}{2m_0} + m_ 0 c^2	& E = \gamma m_0 c^2 \Rightarrow E^2 = c^2p^2 + m^2c^4 \\
H(\bm{r}, \bm{p}) = \frac{p^2}{2m} + U(\bm{r})	& H(\bm{r}, \bm{p}) = \sqrt{c^2p^2 + m^2c^4} + U(\bm{r}) \\
\hline
\end{tabular}
\end{table}


\clearpage
\subsection{Varhainen kvanttimekaniikka}

\begin{eqtable}{\href{https://en.wikipedia.org/wiki/Black-body_radiation}{Mustan kappaleen säteily} \cite[s. 124-128]{ModernPhysics}}
Stefanin-Boltzmannin laki	& R = \sigma T^4 (=I=\frac{P}{A})\\ \hline
lämpötilan ja kin. energ. yhteys & E_{ave} = kT \\ \hline
Wienin siirtymälaki			& \lambda_m T = 2,898 \cdot 10^{-3} m \cdot K \\ \hline
Rayleigh-Jeans				& u( \lambda ) = kT n( \lambda ) = \frac{8 \pi kT}{\lambda^4} \\ \hline
Maxwell-Boltzmann-jakauma	& \phi (E) = A e^{-\frac{E}{kT}} \\ \hline
Planckin laki				& u(\lambda) = \frac{8 \pi hc \lambda^{-5}}{e^{\frac{hc}{\lambda kT}} - 1} \\
\end{eqtable}


\begin{eqtable}{\href{https://en.wikipedia.org/wiki/Compton_scattering}{Comptonin sironta} \cite[s. 142]{ModernPhysics}}
Comptonin yhtälö			& \Delta \lambda = \frac{h}{mc}(1 - \cos \theta ) \\ \hline
Comptonin aallonpituus		& \lambda_c = \frac{h}{mc} \\
\end{eqtable}


\begin{eqtable}{\href{https://en.wikipedia.org/wiki/Rutherford_scattering}{Rutherfordin sironta} \cite[s. 160-163]{ModernPhysics}}
törmäysparametri			& b = \frac{k q_\alpha Q}{m_\alpha v^2} \cot \frac{\theta}{2} \\ \hline
yli kulmaan $\theta$ siroavien osuus	& f = \pi b^2 nt \\ \hline
detektorilla havaittujen hiukkasten määrä	& \Delta N = ( \frac{I_0 A_{sc} nt}{r^2} ) ( \frac{kZe^2}{2 E_k} ) \frac{1}{\sin^4 \frac{\theta}{2}} \\ \hline
ydintä tilavuudessa	(yksiköt!) & n = \frac{\rho N_A}{M} \\
\end{eqtable}


% If you somehow find this article:
% Humphreys, C.J. (1953), "The Sixth Series in the Spectrum of Atomic Hydrogen", J. Research Natl. Bur. Standards
% then please add it as the source. So far the source of this information is Wikipedia:
% https://en.wikipedia.org/wiki/Hydrogen_spectral_series
\begin{table}[ht!]
\centering
\caption{Vedyn spektrisarjat}
\begin{tabular}{| >{$\displaystyle} l <{$} | l |} \hline
1 \rightarrow & Lyman \\
2 \rightarrow & Balmer \\
3 \rightarrow & Paschen \\
4 \rightarrow & Brackett \\
5 \rightarrow & Pfund \\
6 \rightarrow & Humphreys \\
\hline
\end{tabular}
\end{table}


\begin{eqtable}{\href{https://en.wikipedia.org/wiki/Bohr_model}{Bohrin atomimalli} \cite[s. 166-171]{ModernPhysics}}
elektronin ratanopeus	& v = \sqrt{ \frac{kZe^2}{mr}} \\ \hline
kulmaliikemäärän kvantittuminen	& | \mathbf{L} | = n \hbar, n \in \mathbb{N} \\ \hline
atomin energiatilat			& E_n = -\frac{mk^2 z^2 e^4}{2 \hbar^2 n^2} = -\frac{z^2 E_0}{n^2} \\ \hline
spektriviivat (yleistetty Rydberg-Ritz)	& \frac{1}{\lambda} = Z^2 R ( \frac{1}{n_f^2} - \frac{1}{n_i^2}) \\ \hline
Redusoitu massa					& \mu = \frac{mM}{m + M} \\ \hline
\end{eqtable}


\begin{eqtable}{Ytimellinen atomi \cite[s. 176-178]{ModernPhysics}}
Moseleyn laki	& \sqrt{f} = A_n (Z - b) \\ \hline
				& A_n^2 = c R_\infty (1 - \frac{1}{n^2}) \\ \hline
K$_\alpha$:lle	& n=2, b=1 \\
\end{eqtable}



\begin{eqtable}{Sekalaista}
valosähköinen ilmiö			& eV_0 = E_{k max} = hf - \phi \\ \hline
Braggin laki				& 2d \sin \theta = n \lambda \\ \hline
Duane-Hund (jarrutussäteily)	& \lambda_{min} = \frac{1,24 \cdot 10^3 nm}{V (V)} \\
ytimen säde	& r_d = \frac{k q_\alpha Q}{\frac{1}{2}m_\alpha v^2} \\ \hline

kulmaliikemäärän kvantittuminen	& L = mvr = n \hbar \\ \hline

Davisson-Germer ASDF		& n \lambda = D sin \phi \\
\end{eqtable}

\clearpage
\subsection{Kvanttimekaniikka}

\begin{eqtable}{\href{https://en.wikipedia.org/wiki/Wave\%E2\%80\%93particle_duality}{Aaltohiukkasdualismi} \cite[s. 193-233]{ModernPhysics}}
De Broglie -aallonpituus	& \lambda = \frac{h}{p} \\
							& f = \frac{E}{h} \\ \hline
yleinen aaltoyhtälö (1D)	& \frac{d^2 y}{dx^2} = \frac{1}{v^2} \frac{d^2 y}{dt^2} \\
ratkaisut muotoa			& y(x,t) = f(kx - \omega t) \\ \hline
vaihenopeus					& v_p = f \lambda = \frac{\omega}{k} \\ \hline
ryhmänopeus					& v_g = \frac{d \omega}{d k} = v_p + k \frac{d v_p}{dk} \\ \hline
Huom!						& k= \frac{2 \pi}{\lambda} \Rightarrow dk = - \frac{2 \pi}{\lambda^2} d \lambda \\
klassinen epätarkkuus		& \Delta k \Delta x \sim 1 \\
							& \Delta \omega \Delta t \sim 1 \\
epätarkkuusperiaate         & \Delta x \Delta p \sim \hbar \\
                           	& \Delta E \Delta t \sim \hbar \\
							& \Delta x \Delta p \geq \frac{1}{2} \hbar \\
							& \Delta E \Delta t \geq \frac{1}{2} \hbar \\ \hline
nollapiste-energia          & E \geq \frac{h^2}{2mL^2} \\ \hline
yleinen Schrödingerin aaltoyhtälö	& \hat{H} \Psi = E \Psi \\ \hline
yksiulotteinen Schrödinger	& - \frac{\hbar ^2}{2m} \frac{\partial^2 \Psi (x, t)}{\partial x^2} + V(x, t) \Psi (x, t) = i \hbar \frac{\partial \Psi (x, t)}{\partial t} \\ \hline
todennäköisyystulkinta (Kööpenhamina)	& P(x) dx = | \psi | ^2 dx \\ \hline
normalisointiehto			& \int_{-\infty}^{\infty} \Psi^* \Psi dx = 1 \\
\end{eqtable}

\begin{table}
\centering
\caption{\href{https://en.wikipedia.org/wiki/Operator_(physics)}{Kvanttimekaanisia operaattoreita} \cite[s. 252]{ModernPhysics}, \cite[s. 40]{SSED}}
\setlength{\extrarowheight}{10pt}
\begin{tabular}{ >{$\displaystyle} l <{$} | l | >{$\displaystyle} l <{$} } \hline
\text{Symboli}	& Suure & \text{Kvanttimekaaninen operaattori} \\ \hline
x		& paikka	& x \\
f(x)	& mielivaltainen x:n funktio	& f(x) \\
p_x	& liikemäärän x-komponentti	& \frac{\hbar}{i} \frac{\partial}{\partial x} \\
E		& aikariippumaton Hamiltonin operaattori	& \frac{p_{op}^2}{2m} + V(x) \\
E		& aikariippuva Hamiltonin operaattori & -\frac{\hbar}{i} \frac{\partial}{\partial t} \\
E_k		& kineettinen energia	& -\frac{\hbar^2}{2m} \frac{\partial^2}{\partial x^2} \\
L_z		& kulmaliikemäärän z-komponentti	& -i \hbar \frac{\partial}{\partial \phi} \\
		& odotusarvon laskeminen \cite[6-46]{ModernPhysics}	& \langle f(x) \rangle = \int_{-\infty}^\infty \psi^* f(x) \psi dx \\
\end{tabular}
\end{table}

\begin{eqtable}{Kvanttimekaanisia esimerkkitilanteita}
\href{https://en.wikipedia.org/wiki/Particle_in_a_box}{Ääretön potentiaalilaatikko}	\cite[6-2]{ModernPhysics} & n \frac{\lambda}{2} = L, \quad n \in \mathbb{Z}^+ \\
							& E = n^2 \frac{\pi^2 \hbar^2}{2mL^2} = n^2 E_1, \quad n \in \mathbb{Z}^+ \\
							& \psi_n (x) = \sqrt{\frac{2}{L}} \sin \frac{n \pi x}{L} \\ \hline
\href{https://en.wikipedia.org/wiki/Finite_potential_well}{Rajallinen potentiaalilaatikko} \cite[6-3]{ModernPhysics}	& \\
laatikossa						& \psi''(x) = -k^2 \psi(x), \quad k^2 = \frac{2mE}{\hbar^2} \\

laatikon ulkopuolella			& \psi'' (x) = \alpha^2 \psi (x) \\
								& \alpha^2 = \frac{2m}{\hbar^2}(V_0-E) \\ \hline
\end{eqtable}

\begin{eqtable}{Harmoninen oskillaattori \cite[6-5]{ModernPhysics}}
energia 		& E_n = (n + \frac{1}{2}) \hbar \omega, \quad n \in \mathbb{N} \\
aaltofunktiot	& \psi_n (x) = C_n e^{\frac{m \omega x^2}{2 \hbar}} H_n (x) \\
				& \text{jossa } H_n \text{ on Hermiten polynomi} \\
				& \psi_0 (x) = A_0 e^{\frac{-m \omega x^2}{2 \hbar}} \\
				& \psi_1 (x) = A_1 \sqrt{\frac{m \omega}{\hbar}} e^{\frac{-m \omega x^2}{2 \hbar}} \\
				& \psi_2 (x) = A_2 (1- \frac{2m \omega x^2)}{\hbar}) e^{\frac{-m \omega x^2}{2 \hbar}} \\
valintasäännön alkuperä	& \int_{-\infty}^{\infty} \psi^*_n x \psi_m dx = 0, \quad \text{ellei } n=m \pm 1 \\
\end{eqtable}

\begin{eqtable}{Potentiaaliaskel \cite[6-6]{ModernPhysics}}
aaltofunktiot	& x<0: \quad \frac{d^2 \psi (x)}{dx^2} = -k_1^2 \psi (x), \quad k_1 = \frac{\sqrt{2mE}}{\hbar} \\
				& x>0: \quad \frac{d^2 \psi (x)}{dx^2} = -k_2^2 \psi (x), \quad k_2 = \frac{\sqrt{2m(E-V_0)}}{\hbar} \\
yleiset ratkaisut	& x<0: \psi_1 (x) = Ae^{ik_1 x} + Be^{-ik_1 x} \\
					& x>0: \psi_2 (x) = Ce^{ik_1 x} + De^{-ik_1 x}, \quad \text{vasemmalta} \rightarrow D=0 \\
$\Rightarrow$	& B = \frac{k_1 - k_2}{k_1 + k_2} A = \frac{\sqrt{E} - \sqrt{E-V_0}}{\sqrt{E} + \sqrt{E-V_0}} A \\
				& C = \frac{2k_1}{k_1 + k_2} A = \frac{2 \sqrt{E}}{\sqrt{E} + \sqrt{E-V_0}} A \\
heijastus		& R = \frac{|B|^2}{|A|^2} = \Big( \frac{k_1-k_2}{k_1+k_2} \Big)^2 \\
transmissio		& T = \frac{k_2}{k_1} \frac{|C|^2}{|A|^2} = \frac{4k_1 k_2}{(k_1 + k_2)^2} \\
\end{eqtable}

\begin{eqtable}{\href{https://en.wikipedia.org/wiki/Rectangular_potential_barrier}{Potentiaalieste} \cite[6-6]{ModernPhysics}}
transmissio		& T = \frac{|F|^2}{|A|^2} = \Big(1 + \frac{\sinh^2 (\alpha a)}{4 \frac{E}{V_0} \big(1- \frac{E}{V_0} \big) } \Big)^{-1} \\
				& \approx 16 \frac{E}{V_0} \Big( 1- \frac{E}{V_0} \Big) e^{-2 \alpha a} \\
\end{eqtable}


\begin{eqtable}{Atomifysiikka \cite[s. 277->]{ModernPhysics} }
kolmiulotteinen Schrödinger	\cite[s. 41]{SSED} & - \frac{\hbar}{2m} \nabla^2 \Psi + V \Psi = -\frac{\hbar}{i}\frac{\partial \Psi}{\partial t} \\
							& \nabla^2 \Psi = \frac{\partial^2 \Psi}{\partial x^2} + \frac{\partial^2 \Psi}{\partial y^2} + \frac{\partial^2 \Psi}{\partial z^2} \\
\end{eqtable}

\begin{eqtable}{Kolmiulotteinen potentiaalilatikko \cite[7-1]{ModernPhysics}}
esimerkkiratkaisu	& \psi(x, y, z) = A \sin (k_1 x) \sin (k_2 y) \sin (k_3 z) \\
energiatilat	& E_{n_1 n_2 n_3} = \frac{\hbar^2 \pi^2}{2m} (\frac{n_1^2}{L_1^2} + \frac{n_2^2}{L_2^2} + \frac{n_3^2}{L_3^2}), \quad \forall n \in \mathbb{Z}^+ \\
\end{eqtable}

Atomifysiikan symbolikäytänteet polaarisessa koordinaatistossa \cite[7-8]{ModernPhysics}
% A picture of this (as in the page 280 of Modern Physics) is needed!
\begin{align*}
x &= r \sin \theta \cos \phi \\
y &= r \sin \theta \sin \phi \\
z &= r \cos \theta
\end{align*}

\begin{eqtable}{\href{https://en.wikipedia.org/wiki/Spherical_harmonics}{Palloharmonisten funktioiden matematiikkaa} \cite[r26-27, s. 337-342]{Physics227} \cite[s. 127-129]{JohdatusTaivaanmekaniikkaan}}
% \cite[r26-27, s. 336]{Physics227}
Legendren polynomit				& P_l (x) = \frac{(-1)^l}{2^l l!} \frac{d^l}{dx^l} (1-x^2)^l \\
% Symbols |_ and _| would be needed in the following line in the l/2 part: It should be |_l/2_|
%								& = \sum_{k=0}^{l/2} \frac{(-1)^k (2l-2k)!}{2^{2l}k!(l-k)!(l-2k)!} x^{l-2k} \\
% \cite[r26-27, s. 336]{Physics227}
Legendren liittofunktiot	& P_{l,m} (x) = (-1)^m \sqrt{ (1-x^2)^m } \frac{d^m}{dx^m} P_l(x) \\
% \cite[r26-27, s. 338]{Physics227}
Laguerren polynomit				& L_j (x) = e^x \frac{d^j}{d x^j} e^-x x^j \\
% \cite[r26-27, s. 339]{Physics227}
assosioidut Laguerren polynomit	& L_j^k (x) = (-1)^k \frac{d^k}{dx^k} L_{j+k} (x) \\
radiaalisen aaltoyhtälön ratkaiseminen	& y_j^k (x) = e^{-\frac{x}{2}} x^{\frac{k+1}{2}} L_j^k (x) \\
										& R(r) = \frac{y(r)}{r} \\
\end{eqtable}

\begin{table}
\centering
\caption{\href{https://en.wikipedia.org/wiki/Table_of_spherical_harmonics}{Palloharmonisia funktioita} \cite[s.282]{ModernPhysics}}
\setlength{\extrarowheight}{15pt}
\begin{tabular}{ | >{$\displaystyle} l <{$} | >{$\displaystyle} l <{$} | >{$\displaystyle} l <{$} | } \hline
l=0	& m=0	& Y_{00} = \sqrt{\frac{1}{4 \pi}} \\
l=1	& m=1	& Y_{11} = -\sqrt{\frac{3}{8 \pi}} \sin \theta e^{i \phi} \\
	& m=0	& Y_{10} = \sqrt{\frac{3}{4 \pi}} \cos \theta \\
	& m=-1	& Y_{1-1} = \sqrt{\frac{3}{8 \pi}} \sin \theta e^{-i \phi} \\
l=2	& m=2	& Y_{22} = \sqrt{\frac{15}{32 \pi}} \sin^2 \theta e^{2i \phi} \\
	& m=1	& Y_{21} = -\sqrt{\frac{15}{8 \pi}} \sin \theta \cos \theta e^{i \phi} \\
	& m=0	& Y_{20} = \sqrt{\frac{5}{16 \pi}} (3 \cos^2 \theta - 1) \\
	& m=-1	& Y_{2-1} = \sqrt{\frac{15}{8 \pi}} \sin \theta \cos \theta e^{-i \phi} \\
	& m=-2	& Y_{2-2} = \sqrt{\frac{15}{32 \pi}} \sin^2 \theta e^{-2i \phi} \\
\hline
\end{tabular}
\end{table}

Schrödingerin aaltoyhtälö pallokoordinaatistossa
\begin{equation*}
- \frac{\hbar^2}{2\mu} \frac{1}{r^2} \frac{\partial \psi}{\partial r} \Big( r^2 \frac{\partial \psi}{\partial r} \Big)
- \frac{\hbar^2}{2 \mu r^2}
\Big(
\frac{1}{\sin \theta} \frac{\partial}{\partial \theta} \big( \sin \theta \frac{\partial \psi}{\partial \theta} \big)
+ \frac{1}{\sin^2 \theta} \frac{\partial^2 \psi}{\partial \phi^2}
\Big)
+ V(r) \psi = E \psi
\end{equation*}

Palloharmonisten funktioiden generointi \cite[r26-27, (10-10)]{Physics227} \cite[(10.10)]{JohdatusTaivaanmekaniikkaan}
\begin{equation*}
Y_{l,m} (\theta, \phi) = (-1)^m \sqrt{ \frac{(2l+1)(l-m)!}{4 \pi (l+m)!} } P_{l,m}(\cos \theta ) e^{im \phi}, \quad m \geq 0
\end{equation*}

\begin{eqtable}{Puolijohteet}
Efektiivinen massa \cite[s. 73]{SSED}	& m^* = \frac{\hbar^2}{d^2 E / d \bm{k}^2} \\
\end{eqtable}

\part{Kemia}

\begin{table}
\centering
\caption{Termodynaamisia arvoja \cite[A19-A22]{Zumdahl} }
\begin{tabular}{| l | >{$} l <{$} | >{$} l <{$} | >{$} l <{$} |}
\hline
Aine ja olomuoto
& \frac{ \Delta H_f^{\degree} }{ \text{kJ/mol} }
& \frac{ \Delta G_f^{\degree} }{ \text{kJ/mol} }
& \frac{ \Delta S^{\degree} }{\text{J/K} \cdot \text{mol} } \\
\hline
alumiini &&& \\
Al(s)			& 0		& 0		& 28 \\
Al$_2$O$_3$(s)	& -1676	& -1582	& 51 \\
Al(OH)$_3$(s)	& -1277 && \\
AlCl$_3$(s)		& -704	& -629	& 111 \\
&&& \\
barium &&& \\
Ba(s)			& 0		& 0		& 67 \\
BaCO$_3$(s)		& -1219	& -1139	& 112 \\
BaO(s)			& -582	& -552	& 70 \\
Ba(OH)$_2$(s)	& -946 && \\
WORK IN PROGRESS	&&& \\
\hline
\end{tabular}
\end{table}


\part{Tietotekniikka}

% Ideoita
% c++ loogiset operaattorit
% LaTeX-komentoja
% Irssi-opas


% Lähdeluettelo
\bibliography{lahdeluettelo}	% Lähdeluettelotiedosto
\bibliographystyle{IEEEtran}	% Tyyli


\end{document}
