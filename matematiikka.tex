\part{Matematiikka}

\section{Merkintöjä}

\begin{table}[ht!]
\centering
\caption{Kreikkalaiset aakkoset \cite[s. 8]{MAOL}, \cite[sisäkansi]{ModernPhysics}}
\begin{tabular}{  >{$} l <{$}  >{$} l <{$} l l} \hline
\text{Iso}			& \text{Pieni}		& Nimi suomeksi	& Nimi englanniksi \\ \hline
\text{A}	& \alpha	& alfa			& alpha \\
\text{B}	& \beta		& beeta			& beta \\
\Gamma		& \gamma	& gamma			& gamma \\
\Delta		& \delta	& delta			& delta \\
\text{E}	& \epsilon, \varepsilon	& epsilon		& epsilon \\
\text{Z}	& \zeta		& zeeta			& zeta \\
\text{H}	& \eta		& eeta			& eta \\
\Theta		& \theta, \vartheta	& theeta		& theta \\
\text{I}	& \iota		& ioota			& iota \\
\text{K}	& \kappa	& kappa			& kappa \\
\Lambda		& \lambda	& lambda		& lambda \\
\text{M}	& \mu		& myy			& mu \\
\text{N}	& \nu		& nyy			& nu \\
\Xi			& \xi		& ksii			& xi \\
\text{O}	& o	& omikron		& omicron \\
\Pi			& \pi		& pii			& pi \\
\text{P}	& \rho		& rhoo			& rho \\
\text{T}	& \tau		& tau			& tau \\
\Upsilon, \text{Y}	& \upsilon	& ypsilon	& upsilon \\
\Phi	& \phi, \varphi	& fii			& phi \\
\text{X}	& \chi		& khii			& chi \\
\Psi		& \psi		& psii			& psi \\
\Omega		& \omega	& oomega		& omega \\
\end{tabular}
\end{table}

\section{Logiikka}

\begin{table}[ht!]
\centering
\caption{Loogisia operaatioita \cite[s. 6, 9]{MAT-01160}}
\begin{tabular}{
>{$} l <{$}  >{$} l <{$} |
>{$} l <{$}  >{$} l <{$} >{$} l <{$} >{$} l <{$} >{$} l <{$} >{$} l <{$} >{$} l <{$} >{$} l <{$}
}
p & q & \neg p	& p \land q	& p \lor q	& p \rightarrow q	& p \leftrightarrow q	& \overline{p}	& pq	& p + q \\ \hline
1 & 1 & 0 & 1 & 1 & 1 & 1 & 0 & 1 & 1 \\
1 & 0 & 0 & 0 & 1 & 0 & 0 & 0 & 0 & 1 \\
0 & 1 & 1 & 0 & 1 & 1 & 0 & 1 & 0 & 1 \\
0 & 0 & 1 & 0 & 0 & 1 & 1 & 1 & 0 & 0 \\
\end{tabular}
\end{table}


\begin{taulukko}{Päättelysääntöjä \cite[s. 8]{MAT-01160}}
kaksoisnegaation poisto	& \neg \neg p \Leftrightarrow p \\
						\hline
vaihdantalait			& p \lor q \Leftrightarrow q \lor p \\
						& p \land q \Leftrightarrow q \land p \\
						\hline
liitäntälait			& p \lor (q \lor r) \Leftrightarrow (p \lor q) \lor r \\
						& p \land (q \land r) \Leftrightarrow (p \land q) \land r \\
						\hline
osittelulait			& p \land (q \lor r) \Leftrightarrow (p \land q) \lor (p \land r) \\
						& p \lor (q \land r) \Leftrightarrow (p \lor q) \land (p \lor r) \\
						\hline
de Morganin lait		& \neg (p \lor q) \Leftrightarrow \neg p \land \neg q \\
						& \neg (p \land q) \Leftrightarrow \neg p \lor \neg q \\
						\hline
ekvivalenssilaki		& (p \leftrightarrow q) \Leftrightarrow (p \rightarrow q) \land (q \rightarrow p) \\
						\hline
suora todistus			& p \land (p \rightarrow q) \Rightarrow q \\
						\hline
epäsuora todistus		& (p \rightarrow q) \Leftrightarrow (\neg q \rightarrow \neg p) \\
						& \big( p \land ((p \land \neg q) \rightarrow (r \land \neg r)) \big) \Rightarrow q \\ \hline
\end{taulukko}


\begin{taulukko}{Joukko-operaatioiden laskulakeja \cite[s. 15]{MAT-01160}}
	& (A^c)^c = A \\
vaihdantalait	& A \cup B = B \cup A \\
				& A \cap B = B \cap A \\
				\hline
liitäntälait	& A \cup (B \cup C) = (A \cup B) \cup C \\
				& A \cap (B \cap C) = (A \cap B) \cap C \\
				\hline
osittelulait	& A \cap (B \cup C) = (A \cap B) \cup (A \cap C) \\
				& A \cup (B \cap C) = (A \cup B) \cap (A \cup C) \\
				\hline
de Morganin lait	& (A \cup B)^c = A^c \cap B^c \\
					& (A \cap B^c = A^c \cup B^c \\ \hline
\end{taulukko}

Negaation ja kvanttorin vaihtosääntö \cite[s. 17]{MAT-01160}
\begin{align}
\neg (\forall x : p(x)) &\Leftrightarrow \exists x : \neg p(x) \\
\neg (\exists x : p(x)) &\Leftrightarrow \forall x : \neg p(x)
\end{align}

\section{Funktiot}

\begin{taulukko}{Funktiot \cite[s. 25-26]{MAT-01160}}
käänteisfunktio	& y = f(x) \Leftrightarrow x = f^{-1}(y) \\
				& f^{-1}(f(x)) = x \quad \land \quad f(f^{-1}(y)) = y \\ \hline
kuvaaja eli graafi ($f: A \rightarrow \mathbb{R}$)
				& G_f = \{(x, f(x)) \in \mathbb{R}^2 : x \in A \}
				= \{(x,y) \in \mathbb{R}^2 : x \in A, y = f(x) \} \\ \hline
\end{taulukko}

\section{Differentiaali- ja integraalilaskenta}

\subsection{Derivointi}

\subsection{Integrointi}


\section{Vektorit ja matriisit}

\subsection{Vektorit}

\begin{taulukko}{Vektorien perusteet \cite[s. 2-6]{MAT-60000}}
Vektori						& \bm{x} = \begin{matriisi} x_1 \\ x_2 \\ \vdots \\ x_n \end{matriisi} \\ \hline
Luonnolliset kantavektorit	& \bm{e}_i = \begin{matriisi} 0 \\ \vdots \\ 0 \\ 1 \\ 0 \\ \vdots \\ 0 \end{matriisi}, \qquad i \in \mathbb{Z}^+ \\ \hline
Nollavektori				& \bm{0} = \begin{matriisi} 0 \\ \vdots \\ 0 \end{matriisi} \\ \hline
Vektorien summa				& \bm{x} + \bm{y} = \begin{matriisi} x_1 \\ x_2 \\ \vdots \\ x_n \end{matriisi} + 
							\begin{matriisi} y_1 \\ y_2 \\ \vdots \\ y_n \end{matriisi} =
                            \begin{matriisi} x_1 + y_1 \\ x_2 + y_2 \\ \vdots \\ x_n + y_n \end{matriisi}
                            \\ \hline
Vektorien erotus			& \bm{x} - \bm{y} = \bm{x} + (-\bm{y}) \\ \hline
\end{taulukko}

Vektoriavaruuksien aksioomat \cite[s. 8]{MAT-60000}
\begin{align*}
(1)	\quad & \bm{x} + \bm{y} = \bm{y} + \bm{x} \\
(2)	\quad & (\bm{x} + \bm{y}) + \bm{z} = \bm{x} + (\bm{y} + \bm{z}) \\
(3)	\quad & \bm{x} + \bm{0} = \bm{0} + \bm{x} = \bm{x} \\
(4) \quad & \bm{x} + (-\bm{x}) = (-\bm{x}) + \bm{x} = \bm{0} \\
(5) \quad & \alpha ( \bm{x} + \bm{y} ) = \alpha \bm{x} + \alpha \bm{y} \\
(6) \quad & (\alpha + \beta ) \bm{x} = \alpha \bm{x} + \beta \bm{x} \\
(7) \quad & \alpha ( \beta \bm{x} ) = (\alpha \beta) \bm{x} \\
(8) \quad & 1 \bm{x} = \bm{x}
\end{align*}

\begin{taulukko}{Sisätulo ja normi \cite[s. 9-16]{MAT-60000}}
Sisätulo					& \sisatulo{x}{y} = \sum^n_{i=1} \overline{x_i} y_i = \bm{x}^* \bm{y} \\ \hline
Sisätulon perusominaisuudet	& (1) \quad \sisatulo{x}{y} \geq 0 \land (\sisatulo{x}{y} = 0 \rightarrow \bm{x} = \bm{0}) \\
							& (2) \quad \langle \bm{x} + \bm{y}, \bm{z} \rangle = \sisatulo{x}{z} + \sisatulo{y}{z} \\
                           	& (3) \quad \langle \bm{x}, \alpha \bm{y} \rangle = \alpha \sisatulo{x}{y} \\
                            & (4) \quad \sisatulo{x}{y} = \overline{\sisatulo{y}{x}} \\
                            & \forall \bm{x}, \bm{y}, \bm{z} \in \mathbb{C}^n, \alpha \in \mathbb{C} \\ \hline
Ortogonaalisuus				& \sisatulo{x}{y} = 0 \\ \hline
Kroneckerin symboli			& \delta_{ij} \begin{yhtaloryhma} 0 \quad i \neq j \\ 1 \quad i=j \end{yhtaloryhma} \\ \hline
Vektorijoukon ortogonaalisuus	& \langle \bm{x}_i, \bm{x}_j \rangle \begin{yhtaloryhma} 0 \quad i \neq j \\ \neq 0 \quad i=j \end{yhtaloryhma} \\
Vektorijoukon ortonormaalius	& \langle \bm{x}_i, \bm{x}_j \rangle = \delta_{ij} \\ \hline
Normi						& ||\bm{x}|| = \sqrt{\sisatulo{x}{x}} \\ \hline
Normin ominaisuudet			& (1) \quad || \bm{x} || \geq 0 \land (|| \bm{x} || = 0 \leftrightarrow \bm{x} = \bm{0}) \\
							& (2) \quad || \alpha \bm{x} || = | \alpha | || \bm{x} || \quad \forall \alpha \in \mathbb{C}^n \\
Kolmioepäyhtälö				& (3) \quad || \bm{x} + \bm{y} || \leq || \bm{x} || + || \bm{y} || \\ \hline
Kolmioepäyhtälö alaspäin	& || \bm{x} - \bm{y} || \geq \big| || \bm{x} || - || \bm{y} || \big| \\ \hline
Cauchy-Schwarzin epäyhtälö	& |\sisatulo{x}{y}| \leq || \bm{x} || \cdot || \bm{y} || \\ \hline
Vektorien välinen kulma		& \cos(\phi) = \frac{\sisatulo{x}{y}}{|| \bm{x} || \cdot || \bm{y} ||} \\ \hline
\end{taulukko}


\subsection{Matriisit}

\begin{taulukko}{Matriisien perusteet \cite[s. 16-26]{MAT-60000}}
Lineaarikuvaus				& L(a \bm{x} + b \bm{y}) = a L(\bm{x}) + b L({\bm{y})} \\ \hline
Matriisin indeksointi		& A=[a_{ij}]^{m\times n} = 
                			\begin{matriisi} a_{11}  & a_{12}  & \dots  & a_{1n} \\ 
							a_{21}  & a_{22}  & \dots  & a_{2n} \\ 
							\vdots  & \vdots & \ddots & \vdots \\ 
							a_{m1}  & a_{m2}  & \dots  & a_{mn} \end{matriisi} \\
                            
							& \text{m on korkeus (vaakarivien määrä)} \\
                           	& \text{n on leveys (pystyrivien määrä)} \\ \hline

Neliömatriisi				& m=n \\
Korkea matriisi				& m>n \\
Leveä matriisi				& m<n \\ \hline
Sarakevektorit				& A = \begin{matriisi} \bm{a}_1, \bm{a}_2, \ldots, \bm{a}_n\end{matriisi} \\
Vaakarivivektorit			& A = \begin{matriisi} \bm{a}^T_1 \\ \bm{a}^T_2 \\ \vdots \\ \bm{a}^T_m \end{matriisi} \\ \hline
Neliömatriisin jälki		& \text{tr}(A) = \sum^n_{i=1} a_{ii} \text{ eli diagonaalialkioiden summa} \\ \hline
Matriisien summa			& A + B = (a_{ij}) + (b_{ij}) = (a_{ij} + b_{ij}) \\
Skalaarilla kertominen		& \alpha A = (\alpha a_{ij}) \\
Matriisien erotus			& A - B = A + (-1)B = (a_{ij}) + (-b_{ij}) = (a_{ij} - b_{ij}) \\ \hline
Matriisien tulo				& C = (c_{ij})_{m \times n} = \big( \sum^p_{k=1} a_{ik} b_{kj} \big)_{m \times n} \\ \hline
Matriisien laskusäännöt		& (A+B)+C = A+(B+C) \\
							& (AB)C = A(BC) \\
                            & A(B+C) = AB + AC \\
                            & (A+B)C = AC + BC \\
                            & A + B = B + A \\ \hline
Neliömatriisin potenssit	& A^k = AA \cdots A \\ \hline
\end{taulukko}


\begin{taulukko}{Matriisityyppejä \cite[s. 18-21, 34]{MAT-60000}}
Identiteettimatriisi		& I = \begin{matriisi}
							\bm{e}_1, \bm{e}_2, \ldots, \bm{e}_n
							\end{matriisi} = 
							\begin{matriisi}
							1 & 0 & \cdots & 0 \\
                            0 & 1 & \cdots & 0 \\
                            \vdots & \vdots & \ddots & \vdots \\
                            0 & 0 & \cdots & 1 \\
							\end{matriisi} \\ \hline
Lohkomatriisi				& A =
							\begin{matriisi}
                            A_{11} & A_{12} & \cdots & A_{1q} \\
                            \vdots & & & \vdots \\
                            A_{p1} & A_{p2} & \cdots & A_{pq} \\
                            \end{matriisi} \\ \hline
                            
Diagonaalimatriisi			& A = \text{diag}(a_{11}, a_{22}, \ldots, a_{nn}) = 
							\begin{matriisi}
                            a_{11} & 0 & \cdots & 0 \\
                            0 & a_{22} & \cdots & 0 \\
                            \vdots & \vdots & \ddots & \vdots \\
                            0 & 0 & \cdots & a_{nn}
                            \end{matriisi} \\ \hline

Yläkolmiomatriisi			& (i>j \rightarrow a_{ij} = 0) \Leftrightarrow A =
							\begin{matriisi}
                            a_{11} & a_{12} & \cdots & a_{1n} \\
                            0 & a_{22} & \cdots & a_{2n} \\
                            \vdots & \vdots & \ddots & \vdots \\
                            0 & 0 & \cdots & a_{mn}
                            \end{matriisi}
							\\
Alakolmiomatriisi			& (i<j \rightarrow a_{ij} = 0) \Leftrightarrow A = 
							\begin{matriisi}
                            a_{11} & 0 & \cdots & 0 \\
                            a_{21} & a_{22} & \cdots & 0 \\
                            \vdots & \vdots & \ddots & \vdots \\
                            a_{m1} & a_{m2} & \cdots & a_{mn}
                            \end{matriisi}
                            \\ \hline
Lohkodiagonaalinen matriisi	& A = diag(A_{11}, A_{22}, \ldots, A_{pp})
							\begin{matriisi}
                            A_{11} & O & \cdots & O \\
                            O & A_{22} & \cdots & O \\
                            \vdots & \vdots & \ddots & \vdots \\
                            O & O & \cdots & A_{pp}
                            \end{matriisi} \\ \hline
Lohkoalakolmiomatriisi		& A = 
							\begin{matriisi}
                            A_{11} & A_{12} & \cdots & A_{1q} \\
                            O & A_{22} & \cdots & A_{2q} \\
                            \vdots & \vdots & \ddots & \vdots \\
                            O & O & \cdots & A_{pq}
                            \end{matriisi}
							\\
Lohkoyläkolmiomatriisi		& A =
							\begin{matriisi}
                            A_{11} & O & \cdots & O \\
                            A_{21} & A_{22} & \cdots & O \\
                            \vdots & \vdots & \ddots & \vdots \\
                            A_{p1} & A_{p2} & \cdots & A_{pq}
                            \end{matriisi}
							\\ \hline
Permutaatiomatriisi			& P = \begin{matriisi}
							\bm{e}^T_{r_1} \\ \bm{e}^T_{r_1} \\ \vdots \\ \bm{e}^T_{r_n}
                            \end{matriisi}
							= \text{esim.} \begin{matriisi} 0&0&1 \\ 0&1&0 \\ 1&0&0 									\end{matriisi} \\ \hline
\end{taulukko}


\begin{taulukko}{Transpoosi ja konjugaattitranspoosi \cite[s. 21, 26]{MAT-60000}}
Transpoosi					& A^T = (a_{ji})_{n \times m} =
							\begin{matriisi}
                            a_{11} & a_{21} & \cdots & a_{m1} \\
                            a_{12} & a_{22} & \cdots & a_{m2} \\
                            \vdots & \vdots & \ddots & \vdots \\
                            a_{1n} & a_{2n} & \cdots & a_{nm} \\
                            \end{matriisi} \\

Konjugaattitranspoosi       & A^* = (\overline{a}_{ji})_{m \times n} =
							\begin{matriisi}
                            \overline{a}_{11} & \overline{a}_{21} & \cdots & \overline{a}_{m1} \\
                            \overline{a}_{12} & \overline{a}_{22} & \cdots & \overline{a}_{m2} \\
                            \vdots & \vdots & \ddots & \vdots \\
                            \overline{a}_{1n} & \overline{a}_{2n} & \cdots & \overline{a}_{nm} \\
                            \end{matriisi}\\ \hline

Transpoosille				& (AB)^T = B^T A^T \\
							& (A+B)^T = A^T + B^T \\
                            & (\alpha A)^T = \alpha A^T \\ \hline

Konjugaattitranspoosille	& (AB)^* = B^* A^* \\
							& (A+B)^* = A^* + B^* \\
							& (\alpha A)^* = \overline{\alpha} A^* \\ \hline
\end{taulukko}


\begin{taulukko}{Matriisien ominaisuuksia \cite[s. 27-30]{MAT-60000}}
Kommutoivuus				& AB = BA \\ \hline

Symmetrisyys				& A = A^T \\
Hermiittisyys				& A = A^* \\
Vinosymmetrisyys			& A = -A^T \\
Vinohermiittisyys			& A = -A^* \\ \hline

Singulaarisuus				& det(A) = 0 \\
Ei-singulaarisuus			& det(A) \neq 0 \\
							& \exists A^{-1} \\ \hline

Ortogonaalisuus				& A^T A = I = AA^T \\
Unitaarisuus				& U^* U = U U^* = I \\ \hline

Inverssi					& A A^{-1} = I = A^{-1} A \\
							& (AB)^{-1} = B^{-1} A^{-1} \\
                           	& (A^{-1})^{-1} = A \\ \hline
\end{taulukko}


\begin{taulukko}{Lineaarinen yhtälöryhmä \cite[s. 31]{MAT-60000}}
Lineaarinen yhtälöryhmä		& A \bm{x} = \bm{b} \\
							& \bm{x} = A^{-1} \bm{b} \\ \hline
\end{taulukko}


\begin{taulukko}{Hermiittisille matriiseille \cite[s. 35]{MAT-60000}}
Positiivisesti definiitti		& \langle \bm{x}, A \bm{x}\rangle > 0, \forall \bm{x} \neq \bm{0} \\
Positiivisesti semidefiniitti	& \langle \bm{x}, A \bm{x}\rangle \geq 0, \forall \bm{x} \\
Negatiivisesti definiitti		& \langle \bm{x}, A \bm{x}\rangle < 0, \forall \bm{x} \neq \bm{0} \\
Negatiivisesti semidefiniitti	& \langle \bm{x}, A \bm{x}\rangle \leq 0, \forall \bm{x} \\
Indefiniitti					& \text{Jos ei ole mitään näistä} \\ \hline
\end{taulukko}


\begin{taulukko}{LU-hajotelma \cite[s. 45-56]{MAT-60000}}
LU-hajotelma	& A = LU \\
				& L_k = I - \bm{l}_k \bm{e}^T_k \\
                & \bm{l}_k = \frac{1}{x_k} \begin{matriisi} 0 \\ \vdots \\ 0 \\ x_{k+1} \\ \vdots \\ x_n \end{matriisi}, x_k \neq 0 \\
                & L_{n-1} L_{n-2} \cdots L_1 A = U \\
                & \hat{L} = L_{n-1} L_{n-2} \cdots L_1 \\
                & A = \hat{L}^{-1} U = LU \\ \hline
Jos ei onnistu suoraan, niin viimeistään	& PA = LU, \text{ jossa P on permutaatiomatriisi} \\ \hline
Lineaarisen yhtälöryhmän ratkaiseminen	& A \bm{x} = P^T LU \bm{x} = \bm{b} \\
											& LU \bm{x} = P \bm{b} \\
                                            & L \bm{y} = P \bm{b} \land U \bm{x} = \bm{y} \\
                                            & x = U^{-1} \bm{y} \land \bm{y} = L^{-1} P \bm{b} \\ \hline
\end{taulukko}


\begin{taulukko}{Aliavaruus \cite[s. 59-83]{MAT-60000}}
Aliavaruus					& \alpha \bm{x} + \beta \bm{y} \in S \quad \forall \bm{x}, \bm{y} \in \mathcal{S} \\
Arvojoukko (kuva-avaruus)	& \mathcal{R} (A) = \{y | \exists \bm{x} \in \mathbb{F}^n s.e. \bm{y} = A \bm{x} \} \in \mathbb{F}^m \\
Ydin (nolla-avaruus)		& \mathcal{N} (A) = \{ \bm{x} | A \bm{x} = \bm{0} \} \\ \hline

Ortogonaalikomplementti		& \mathcal{S}^\perp = \{ \bm{y} | \langle \bm{y}, \bm{x} \rangle = 0, \forall \bm{x} \in \mathcal{S} \} \\ \hline

							& \mathcal{N} (A) = \mathcal{R} (A^*)^\perp \\
                            & \mathcal{N} (A^*) = \mathcal{R} (A)^\perp \\ \hline

Dimensio eli kantavektorien lukumäärä	& \dim (S) \\
Matriisin aste							& rank (A) = \dim \mathcal{R} (A) \\
										& rank (A) + \dim \mathcal{N} (A) = n = \text{ sarakkeiden määrä} \\ \hline

Projektorimatriisi			& P^2 = P \\
							& \text{Projisoi vektorit } \mathcal{R} (P) \text{:lle pitkin } \mathcal{N} (P) \text{:tä} \\ \hline
\end{taulukko}


\begin{taulukko}{Ominaisarvot ja -vektorit \cite[s. 90-100]{MAT-60000}}
Ominaisarvot ja -vektorit	& A \bm{x} = \lambda \bm{x}, \bm{x} \neq \bm{0} \\
Ominaisarvot				& \det (A - \lambda I) = 0 \\
Karakteristinen polynomi	& \prod^n_{k=1} (\lambda - \lambda_k) \\
Spektri						& \sigma (A) = \{ \lambda_ 1, \lambda_2, \ldots , \lambda_n \} \\
Similaarisuus				& B = S^{-1} A S \text{ jossa S ei-singulaarinen}\\
Unitaarinen similaarisuus	& B = U^* AU \\

Hermiten matriisille		& diag(\sigma(A)) = U^* A U \\
							& \forall \lambda \in \mathbb{R} \land n \text{ ortonormaalia ominaisvektoria} \Rightarrow \text{Hermiten matriisi} \\ \hline

Normaali matriisi			& A = UDU^* \text{ jossa D on diagonaalimatriisi} \\ \hline
\end{taulukko}


\begin{taulukko}{Jordanin kanoninen muoto \cite[s. 101-107]{MAT-60000}}
				& A = SJS^{-1} \\
				& J = diag(J_1, J_2, \ldots, J_p) \\
Defektiivisyys	& \exists \lambda : geom(\lambda) < alg(\lambda) \\ \hline
				& A \bm{s}^j_1 = \lambda_j \bm{s}^j_1 \\
                & A \bm{s}^j_2 = \lambda_j \bm{s}^j_2 \bm{s}^j_1 \\
                & A \bm{s}^j_i = \lambda_j \bm{s}^j_i \bm{s}^j_{i-1} \\
Jordanin ketju	& \{ \bm{s}^j_1, \bm{s}^j_2, \ldots \bm{s}^j_{r_j} \} \\
Ominaisvektori	& \bm{s}^j_1 \\
Yleistetyt ominaisvektorit	& \bm{s}^j_2, \ldots \bm{s}^j_{r_j} \\
Kompleksiselle ominaisarvolle	& J_i = \begin{matriisi} \alpha & \beta \\ - \beta & \alpha \end{matriisi} \in \mathbb{R}^{2 \times 2} \\ \hline
Käyttöohje		& \text{Etsi } \sigma (A) \\
				& \text{Valitse mielivaltainen } \lambda \\
                & alg(\lambda) = 1 \Rightarrow \bm{s}^1_1 = \bm{x}_1 \land J_1 = \lambda_1 \\
                & alg(\lambda) = geom(\lambda) = l \Rightarrow \text{Otetaan ominaisvektorit} \land \forall J = \lambda_1 \\
                & alg(\lambda) > geom(\lambda) \Rightarrow \text{Muodostetaan ketjut erikseen} \\
               	& \lambda \text{ kompleksinen} \Rightarrow (\bm{s}^1_1 = \text{Re } \bm{x}, \bm{s}^1_2 = \text{Im } \bm{x}) \land (\overline{\lambda} \in \sigma(A)) \\ \hline
\end{taulukko}


\begin{taulukko}{Singulaariarvohajotelma \cite[s. 117-128]{MAT-60000}}
Singulaariarvohajotelma		& U^*AV = \Lambda \\
							& A = U \Lambda V^* \\
							& \Lambda = diag(\sigma_1, \sigma_2, \ldots , \sigma_r, 0, \ldots, 0) \\
                            & \sigma(A^*A) = \{ \sigma^2_1, \sigma^2_2, \ldots, \sigma^2_{r}, \sigma^2_{r+1}, \sigma^2_n \} \\
                            & V = [\bm{v}_1, \bm{v}_2, \ldots, \bm{v}_n] \text{ jossa v:t ortonormaaleja ominaisvektoreita} \\
                            & \bm{u}_i = \frac{A \bm{v}_i}{\sigma_i} \\
                            & \text{Toimii samoin, vaikka valittaisiin } AA^* \\ \hline
                            & rank(A) = r \\
                            & \mathcal{R}(A) = span\{ \bm{u}_1, \bm{u}_2, \ldots, \bm{u}_r \} \\
                            & \mathcal{N}(A) = span\{ \bm{v}_{r+1}, \bm{v}_{r+2}, \ldots, \bm{v}_n \} \\ \hline
                            
                            & || A || = \sigma_1 \\
Jos A on ei-singulaarinen	& \sigma_n || \bm{x} || \leq || A \bm{x} || \leq \sigma_1 || \bm{x} ||, \forall \bm{x} \in \mathbb{F}^n \\
							& || A^{-1} || = \frac{1}{\sigma_n} \\ \hline

Approksimointi				& B = U diag(\sigma_1, \sigma_2, \ldots, \sigma_{r-1}, 0, \ldots, 0) V^* \\
							& ||A - B|| = \sigma_r \\ \hline

Pseudoinverssi				& A^\dagger = V \Lambda^\dagger U^*\\
							& \Lambda^\dagger \text{ on } \Lambda^T, \text{jossa } \sigma_i \Rightarrow \frac{1}{\sigma_i} \\
                            & AA^\dagger A = A \\
                            & A^\dagger A A^\dagger = A^\dagger \\
Lin. yhtälöryhmän 			& A\bm{x} = \bm{b} \\
yleinen ratkaisu			& \bm{x} = A^\dagger \bm{b} \\ \hline
\end{taulukko}


