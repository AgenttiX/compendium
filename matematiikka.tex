\part{Matematiikka}

\section{Merkintöjä}

\begin{table}[ht!]
\centering
\caption{\href{https://en.wikipedia.org/wiki/Greek_alphabet}{Kreikkalaiset aakkoset} \cite[s. 8]{MAOL}, \cite[sisäkansi]{ModernPhysics}}
\begin{tabular}{ >{$} l <{$}  >{$} l <{$} l l } \hline
\text{Iso}			& \text{Pieni}		& Nimi suomeksi	& Nimi englanniksi \\ \hline
\text{A}	& \alpha	& alfa			& alpha \\
\text{B}	& \beta		& beeta			& beta \\
\Gamma		& \gamma	& gamma			& gamma \\
\Delta		& \delta	& delta			& delta \\
\text{E}	& \epsilon, \varepsilon	& epsilon		& epsilon \\
\text{Z}	& \zeta		& zeeta			& zeta \\
\text{H}	& \eta		& eeta			& eta \\
\Theta		& \theta, \vartheta	& theeta		& theta \\
\text{I}	& \iota		& ioota			& iota \\
\text{K}	& \kappa	& kappa			& kappa \\
\Lambda		& \lambda	& lambda		& lambda \\
\text{M}	& \mu		& myy			& mu \\
\text{N}	& \nu		& nyy			& nu \\
\Xi			& \xi		& ksii			& xi \\
\text{O}	& o	& omikron		& omicron \\
\Pi			& \pi		& pii			& pi \\
\text{P}	& \rho		& rhoo			& rho \\
\text{T}	& \tau		& tau			& tau \\
\Upsilon, \text{Y}	& \upsilon	& ypsilon	& upsilon \\
\Phi	& \phi, \varphi	& fii			& phi \\
\text{X}	& \chi		& khii			& chi \\
\Psi		& \psi		& psii			& psi \\
\Omega		& \omega	& oomega		& omega \\
\end{tabular}
\end{table}

\section{Logiikka}

\begin{table}[ht!]
\centering
\caption{Loogisia operaatioita \cite[s. 6, 9]{MAT-01160}}
\begin{tabular}{
>{$} l <{$}  >{$} l <{$} |
>{$} l <{$}  >{$} l <{$} >{$} l <{$} >{$} l <{$} >{$} l <{$} >{$} l <{$} >{$} l <{$} >{$} l <{$}
}
p & q & \neg p	& p \land q	& p \lor q	& p \rightarrow q	& p \leftrightarrow q	& \overline{p}	& pq	& p + q \\ \hline
1 & 1 & 0 & 1 & 1 & 1 & 1 & 0 & 1 & 1 \\
1 & 0 & 0 & 0 & 1 & 0 & 0 & 0 & 0 & 1 \\
0 & 1 & 1 & 0 & 1 & 1 & 0 & 1 & 0 & 1 \\
0 & 0 & 1 & 0 & 0 & 1 & 1 & 1 & 0 & 0 \\
\end{tabular}
\end{table}


\begin{eqtable}{Päättelysääntöjä \cite[s. 8]{MAT-01160}}
kaksoisnegaation poisto	& \neg \neg p \Leftrightarrow p \\
						\hline
vaihdantalait			& p \lor q \Leftrightarrow q \lor p \\
						& p \land q \Leftrightarrow q \land p \\
						\hline
liitäntälait			& p \lor (q \lor r) \Leftrightarrow (p \lor q) \lor r \\
						& p \land (q \land r) \Leftrightarrow (p \land q) \land r \\
						\hline
osittelulait			& p \land (q \lor r) \Leftrightarrow (p \land q) \lor (p \land r) \\
						& p \lor (q \land r) \Leftrightarrow (p \lor q) \land (p \lor r) \\
						\hline
de Morganin lait		& \neg (p \lor q) \Leftrightarrow \neg p \land \neg q \\
						& \neg (p \land q) \Leftrightarrow \neg p \lor \neg q \\
						\hline
ekvivalenssilaki		& (p \leftrightarrow q) \Leftrightarrow (p \rightarrow q) \land (q \rightarrow p) \\
						\hline
suora todistus			& p \land (p \rightarrow q) \Rightarrow q \\
						\hline
epäsuora todistus		& (p \rightarrow q) \Leftrightarrow (\neg q \rightarrow \neg p) \\
						& \big( p \land ((p \land \neg q) \rightarrow (r \land \neg r)) \big) \Rightarrow q \\
\end{eqtable}


\begin{eqtable}{Joukko-operaatioiden laskulakeja \cite[s. 15]{MAT-01160}}
	& (A^c)^c = A \\
vaihdantalait	& A \cup B = B \cup A \\
				& A \cap B = B \cap A \\
				\hline
liitäntälait	& A \cup (B \cup C) = (A \cup B) \cup C \\
				& A \cap (B \cap C) = (A \cap B) \cap C \\
				\hline
osittelulait	& A \cap (B \cup C) = (A \cap B) \cup (A \cap C) \\
				& A \cup (B \cap C) = (A \cup B) \cap (A \cup C) \\
				\hline
de Morganin lait	& (A \cup B)^c = A^c \cap B^c \\
					& (A \cap B)^c = A^c \cup B^c \\
\end{eqtable}

Negaation ja kvanttorin vaihtosääntö \cite[s. 17]{MAT-01160}
\begin{align*}
\neg (\forall x : p(x)) &\Leftrightarrow \exists x : \neg p(x) \\
\neg (\exists x : p(x)) &\Leftrightarrow \forall x : \neg p(x)
\end{align*}

\section{Funktiot}

\begin{eqtable}{Funktiot \cite[s. 25-26]{MAT-01160}}
käänteisfunktio	& y = f(x) \Leftrightarrow x = f^{-1}(y) \\
				& f^{-1}(f(x)) = x \quad \land \quad f(f^{-1}(y)) = y \\ \hline
kuvaaja eli graafi ($f: A \rightarrow \mathbb{R}$)
				& G_f = \{(x, f(x)) \in \mathbb{R}^2 : x \in A \}
				= \{(x,y) \in \mathbb{R}^2 : x \in A, y = f(x) \} \\
\end{eqtable}

\section{Aritmetiikka ja algebra}

Raja-arvoja \cite[s. 20]{MAOL}
\begin{align*}
& \lim_{n \rightarrow \infty} \sqrt[n]{n} = 1 \\
& \lim_{n \rightarrow \infty} \Big( 1 + \frac{1}{n} \Big)^n = e \approx 2,718 282 \text{ (Neperin luku)} \\
& \lim_{n \rightarrow \infty} \Big( \big( \sum_{k=1}^n \frac{1}{k} \big) - \ln n \Big) \approx 0,577 216 \text{ (Eulerin vakio)} \\
& \lim_{n \rightarrow \infty} \frac{a^n}{n!} = 0 \\
& \lim_{x \rightarrow 0+} \big( x^a \cdot \log x \big) = 0 \\
& \lim_{x \rightarrow \infty} \frac{\log x}{x^a} = 0 \\
& \lim_{x \rightarrow \infty} \frac{x^a}{e^x} = 0 \\
& \lim_{n \rightarrow \infty} \Big( n \cdot \sin \frac{x}{n} \Big) = x \\
& \lim_{x \rightarrow 0} \frac{\sin x}{x} = 1
\end{align*}

\begin{eqtable}{Aritmeettinen ja geometrinen lukujono \cite[s. 20]{MAOL} }
aritmeettinen jono	& a_1, a_1 + d, a_1 + 2d \ldots \\
yleinen termi		& a_n = a_1 + (n-1)d \\
aritmeettinen summa	& S_n = \sum_{i=1}^n a_i = n \frac{a_1+a_n}{2} \\
\hline
geometrinen jono	& a_1, a_1 q, a_1 q^2 \ldots \\
yleinen termi		& a_1 q^{n-1} \\
geometrinen summa	& S_n = \sum_{i=1}{n} a_i = \frac{a_1 (1-q^n)}{1-q}, \text{ jos } q \neq 1 \\
					& S_n = n a_1, \text{ jos } q = 1 \\
suppeneva geometrinen sarja	& S = \sum_{i=1}^\infty a_i = a_1 + a_1 q + a_1 q^2 0 \ldots = \frac{a_1}{1-q}, \text{ jos } |q| \leq 1 \\
\end{eqtable}

\clearpage

\section{Trigonometria}

Trigonometristen funktioiden väliset yhteydet \cite[s. 31]{MAOL}
\begin{align*}
& \sin x = \pm \sqrt{1- \cos^2 x} = \pm \frac{\tan x}{\sqrt{1 + \tan^2 x}} \\
& \cos x = \pm \sqrt{1 - \sin^2 x} = \pm \frac{1}{\sqrt{1 + \tan^2 x}} \\
& \tan x = \pm \frac{ \sin x}{\sqrt{1 - \sin^2 x}} = \pm \frac{\sqrt{1 - \cos^2 x}}{\cos x}
\end{align*}

Palautuskaavat \cite[s. 31]{MAOL}
\begin{align*}
& \sin x = - \sin(-x) = \cos \Big( \frac{\pi}{2} - x \Big) = \sin(\pi -x) = \sin(x + n2\pi) \\
& \cos x = \cos(-x) = \sin \Big( \frac{\pi}{2} - x \Big) = -\cos(\pi-x) = \cos(x + n2\pi) \\
& \tan x = - \tan(-x) = -\tan(\pi - x) = \tan(x + n\pi)
\end{align*}

Kaksinkertaiset kulmat \cite[s. 32]{MAOL}
\begin{align*}
& \sin 2x = 2 \sin x \cos x \\
& \cos 2x = \cos^2 x - \sin^2 x = 2 \cos^2 x - 1 = 1 - 2 \sin^2 x \\
& \tan 2x = \frac{2 \tan x}{1 - \tan^2 x}
\end{align*}

Kolminkertaiset kulmat \cite[s. 32]{MAOL}
\begin{align*}
& \sin 3x = 3 \sin x - 3 \sin^3 x = \sin x (4 \cos^2 x - 1) \\
& \cos 3x = 4 \cos^3 x - 3 \cos x = \cos x (1 - 4 \sin^2 x ) \\
& \tan 3x = \frac{3 \tan x - \tan^3 x}{1 - 3 \tan^2 x}
\end{align*}

Puolikkaat kulmat \cite[s. 32]{MAOL}
\begin{align*}
& \sin \frac{x}{2} = \pm \sqrt{\frac{1 - \cos x}{2}} \\
& \cos \frac{x}{2} = \pm \sqrt{\frac{1 + \cos x}{2}} \\
& \tan \frac{x}{2} = \pm \sqrt{\frac{1 - \cos x}{1 + \cos x}} = \frac{\sin x}{1 + \cos x} = \frac{1 - \cos x}{\sin x}
\end{align*}

Trigonometristen funktioiden potensseja \cite[s. 33]{MAOL}
\begin{align*}
& \sin^2 x = \frac{1}{2} (1- \cos 2x) \\
& \cos^2 x = \frac{1}{2} (1+ \cos 2x) \\
& \sin^3 x = \frac{1}{4} (3 \sin x - \sin 3x) \\
& \cos^3 x = \frac{1}{4} (\cos 3x + 3 \cos x) \\
& \sin^3 x = \frac{1}{8} (\cos 4x - 4 \cos 2x + 3) \\
& \cos^4 x = \frac{1}{8} (\cos 4x + 4 \cos 2x + 3)
\end{align*}

Summakaavoja \cite[s. 33]{MAOL}
\begin{align*}
& \sin ( x \pm y ) = \sin x \cos y \pm \cos x \sin y \\
& \cos (x \pm y ) = \cos x \cos y \mp \sin x \sin y \\
& \tan (x \pm y ) = \frac{\tan x \pm \tan y}{1 \mp \tan x \tan y} \\
& \sin x + \sin y = 2 \sin \frac{x+y}{2} \cos \frac{x-y}{2} \\
& \sin x - \sin y = 2 \cos \frac{x+y}{2} \sin \frac{x-y}{2} \\
& \cos x + \cos y = 2 \cos \frac{x+y}{2} \cos \frac{x-y}{2} \\
& \cos x - \cos y = -2 \sin \frac{x+y}{2} \sin \frac{x-y}{2} \\
& \tan x \pm \tan y = \frac{\sin(x \pm y)}{\cos x \cos y} \\
& \cos x \pm \sin x = \sqrt{2} \sin \Big( \frac{\pi}{4} \pm x \Big) = \sqrt{2} \cos \Big( \frac{\pi}{4} \mp x \Big) 
\end{align*}

\section{Differentiaali- ja integraalilaskenta}

\subsection{Derivointi}

\begin{eqtable}{Derivaatan määritelmä \cite[s. 41]{MAOL} }
derivaattafunktio			& f'(x) = \lim_{h \rightarrow 0} \frac{f(x+h)-f(x)}{h} \\
derivaatta kohdassa $x_0$	& f'(x_0 = \lim_{h \rightarrow 0} \frac{f(x_0 + h)-f(x_0)}{h} = \lim_{x \rightarrow x_0} \frac{f(x)-f(x_0)}{x-x_0} \\
\end{eqtable}

Derivoimissääntöjä \cite[s. 41]{MAOL}
\begin{align*}
& Dk					= 0 \\
& Dkf(x)				= kf'(x) \\
& D(f(x) + g(x))		= f'(x) + g'(x) \\
& Df(x)g(x)			= f'(x)g(x)+f(x)g'(x) \\
& D \frac{f(x)}{g(x)}	= \frac{f'(x)g(x)-f(x)g'(x)}{g(x)^2} \\
& D g(f(x))			= g'(f(x))f'(x) \\
& (f^{-1})'(y_0)		= \frac{1}{f'(x_0)}, \quad \text{jossa } y_0 = f(x_0)
\end{align*}

Derivoimiskaavoja \cite[s. 41]{MAOL}
\begin{align*}
& Dx^n	= nx^{n-1} \\
& D(f(x))^n	= n(f(x))^{n-1} f'(x) \\
& D \sin x	= \cos x \\
& D \cos x	= -\sin x \\
& D \tan x	= \frac{1}{\cos^2 x} = 1 + \tan^2 x \\
& D e^x		= e^x \\
& D a^x		= a^x \ln a, \quad a>0 \\
& D \ln |x|	= \frac{1}{x} \\
& D \log_a |x|	= \frac{1}{x \ln a}, \quad a>0, a \neq 1 \\
& D \arcsin x	= \frac{1}{\sqrt{1-x^2}} \\
& D \arccos x	= - \frac{1}{\sqrt{1-x^2}} \\
& D \arctan x	= \frac{1}{1+x^2} \\
\end{align*}

\subsection{Integrointi}
Määritelmä \cite[(1.8)]{MAT-01360}
\begin{equation*}
\int f'(x) dx = f(x) + C
\end{equation*}

\begin{eqtable}{Integroinnin ominaisuuksia}
integroinnin lineaarisuus \cite[(1.10)]{MAT-01360}		& \int c f(x) dx = c \int f(x) dx, \quad c \in \mathbb{R} \\
														& \int (f(x)+g(x))dx = \int f(x)dx + \int g(x)dx \\
osittaisintegrointi \cite[(1.13)]{MAT-01360}			& \int f'(x)g(x)dx = f(x)g(x) - \int f(x)g'(x)dx \\
integrointi sijoituksen avulla \cite[(1.18)]{MAT-01360}	& \int f(g(x))g'(x)dx = F(g(x)) + C \\
\end{eqtable}

\iffalse
Integroimiskaavoja \cite[s.42]{MAOL}
\begin{align*}
& \int 0 dx = C \\
& \int kdx = kx + C \\
& \int f'(x)(f(x))^n dx = \frac{(f(x))^{n+1}}{n+1} + C, \quad n \neq -1 \\
& \int \frac{f'(x)}{f(x)} dx = \ln |f(x)| + C, \quad f(x) \neq 0 \\
& \int \sin x dx = - \cos x + C \\
& \int \cos x dx = sin x + C \\
& \int \tan x dx = - \ln | \cos x | + C \\
& \int e^x dx = e^x + C \\
& \int f'(x)e^{f(x)} dx = e^{f(x)} + C \\
& \int a^x dx = \frac{a^x}{\ln a} + C, \quad a>0, a \neq 1 \\
& \log_a |x| dx = (log_a e)(x \ln |x| - x) + C, \quad a>0, a \neq 1 \\
& \int \frac{dx}{1+x^2} = \arctan x + C \\
& \int \frac{dx}{\sqrt{1-x^2}} = \arcsin x + C
\end{align*}
\fi

\begin{table}
\centering
\caption{Integroimiskaavoja \cite[s.130]{MAT-01360} }
\setlength{\extrarowheight}{5pt}
\begin{tabular}{| >{$} l <{$} | >{$} l <{$} | >{$} l <{$} | }
\hline
f(x)	& \int f(x) dx	& I \\
\hline
x^n, n \in \mathbb{Z} \setminus \{-1\}	& \frac{1}{n+1}x^{n+1} + C			& (-\infty,0) \text{ tai } (0,\infty), \text{ kun } n<0 \\
										& 									& \mathbb{R}, \text{ kun } n \geq 0 \\
x^a, a \in \mathbb{R} \setminus \{-1\}	& \frac{1}{a+1}x^{a+1} + C			& (0,\infty) \\
\frac{1}{x}								& \ln |x| + C						& (-\infty,0) \lor (0,\infty) \\
e^x										& e^x + C							& \mathbb{R} \\
a^x, \quad a>0 \land a \neq 1			& \frac{a^x}{\ln a} + C				& \mathbb{R} \\
\sin x									& -\cos x + C						& \mathbb{R} \\
\cos x									& \sin x + C						& \mathbb{R} \\
\tan x									& - \ln |\cos x| + C				& (- \frac{\pi}{2}+n\pi, \frac{\pi}{2}+n\pi), n \in \mathbb{Z} \\
\cot x = \frac{1}{\tan x}				& \ln |\sin x| + C					& (n\pi, (n+1)\pi), n \in \mathbb{Z} \\
\frac{1}{\cos^2 x}						& \tan x + C						& (- \frac{\pi}{2}+n\pi, \frac{\pi}{2}+n\pi), n \in \mathbb{Z} \\
\frac{1}{\sin^2 x}						& -\cot x + C						& (n\pi, (n+1)\pi), n \in \mathbb{Z} \\
\frac{1}{\sqrt{1-x^2}}					& \arcsin x + C						& (-1,1) \\
\frac{1}{1+x^2}							& \arctan x + C						& \mathbb{R} \\
\frac{1}{\sqrt{x^2+1}}					& \text{ar} \sinh x + C				& \mathbb{R} \\
										& = \ln ( x + \sqrt(x^2+1)+C		& \\
\frac{1}{\sqrt{x^2-1}}					& \text{ar} \cosh x + C				& (-\infty,-1) \lor (1, \infty) \\
										& = \ln | x + \sqrt{x^2-1}|+C		& \\
\frac{1}{1-x^2}							& \text{ar} \tanh x + C				& (-1,1) \\
										& = \frac{1}{2} \ln \frac{x+1}{x-1} + C	& \\
\hline
\end{tabular}
\end{table}

\begin{eqtable}{Määrätty integraali \cite[s. 17-23]{MAT-01360} }
% \cite[s. 17]{MAT-01360}
Riemannin summa		& R = \sum^n_{i=1} f(x_i^*) \Delta x_i \\
% \cite[(1.33)]{MAT-01360}
määrätty integraali	 & I = \lim_{|P| \rightarrow 0} \sum^n_{i=1} f(x_i^*) \Delta x_i = \int_a^b f(x) dx = \int_a^b f \\
% \cite[(1.42)]{MAT-01360}
integraalilaskennan väliarvolause 	& f: [a,b] \rightarrow \mathbb{R} \text{ on jatkuva} \Rightarrow \exists c \in [a,b] :\\
											& \int_a^b f(x) dx = f(c)(b-a) \\
% \cite[(1.43)]{MAT-01360}
integroituvan funktion keskiarvo 	& \hat{f} = \frac{1}{b-a} \int_a^b f(x)dx \\
% cite[(1.44)]{MAT-01360}
analyysin peruslause				& f: [a,b] \rightarrow \mathbb{R} \text{ on jatkuva} \\
									& \Rightarrow F(x) = \int_a^x f(t) dt \text{ on derivoituva } \land F'(x) = f(x) \forall x \in [a,b] \\
\end{eqtable}

Määrätyn integraalin ominaisuuksia \cite[(1.38)]{MAT-01360}
\begin{align*}
(1) \quad & \int_a^b cf(x) dx = c \int_a^b f(x)dx \\
(2) \quad & \int_a^b (f(x)+g(x))dx = \int_a^b f(x) dx + \int_a^b g(x) dx \\
(3) \quad & \int_a^b f(x) dx = \int_a^c f(x) dx + \int_c^b f(x) dx \quad (a<c<b) \\
(4) \quad & \forall x \in [a,b]: f(x) \leq g(x) \Rightarrow \int_a^b f(x) dx \leq \int_a^b g(x) dx \\
(5)	\quad & \Big| \int_a^b f(x) dx \Big| \leq \int_a^b |f(x)| dx \\
\end{align*}

\section{Vektorit ja matriisit}

\subsection{Vektorit}

\begin{eqtable}{Vektorien perusteet \cite[s. 2-6]{MAT-60000}}
Vektori						& \bm{x} = \begin{styledmatrix} x_1 \\ x_2 \\ \vdots \\ x_n \end{styledmatrix} \\ \hline
Luonnolliset kantavektorit	& \bm{e}_i = \begin{styledmatrix} 0 \\ \vdots \\ 0 \\ 1 \\ 0 \\ \vdots \\ 0 \end{styledmatrix}, \qquad i \in \mathbb{Z}^+ \\ \hline
Nollavektori				& \bm{0} = \begin{styledmatrix} 0 \\ \vdots \\ 0 \end{styledmatrix} \\ \hline
Vektorien summa				& \bm{x} + \bm{y} = \begin{styledmatrix} x_1 \\ x_2 \\ \vdots \\ x_n \end{styledmatrix} + 
							\begin{styledmatrix} y_1 \\ y_2 \\ \vdots \\ y_n \end{styledmatrix} =
                            \begin{styledmatrix} x_1 + y_1 \\ x_2 + y_2 \\ \vdots \\ x_n + y_n \end{styledmatrix}
                            \\ \hline
Vektorien erotus			& \bm{x} - \bm{y} = \bm{x} + (-\bm{y}) \\
\end{eqtable}

Vektoriavaruuksien aksioomat \cite[s. 8]{MAT-60000}
\begin{align*}
(1)	\quad & \bm{x} + \bm{y} = \bm{y} + \bm{x} \\
(2)	\quad & (\bm{x} + \bm{y}) + \bm{z} = \bm{x} + (\bm{y} + \bm{z}) \\
(3)	\quad & \bm{x} + \bm{0} = \bm{0} + \bm{x} = \bm{x} \\
(4) \quad & \bm{x} + (-\bm{x}) = (-\bm{x}) + \bm{x} = \bm{0} \\
(5) \quad & \alpha ( \bm{x} + \bm{y} ) = \alpha \bm{x} + \alpha \bm{y} \\
(6) \quad & (\alpha + \beta ) \bm{x} = \alpha \bm{x} + \beta \bm{x} \\
(7) \quad & \alpha ( \beta \bm{x} ) = (\alpha \beta) \bm{x} \\
(8) \quad & 1 \bm{x} = \bm{x}
\end{align*}

\begin{eqtable}{Sisätulo ja normi \cite[s. 9-16]{MAT-60000}}
sisätulo					& \innerp{x}{y} = \sum^n_{i=1} \overline{x_i} y_i = \bm{x}^* \bm{y} \\ \hline
sisätulon perusominaisuudet	& (1) \quad \innerp{x}{y} \geq 0 \land (\innerp{x}{y} = 0 \rightarrow \bm{x} = \bm{0}) \\
							& (2) \quad \langle \bm{x} + \bm{y}, \bm{z} \rangle = \innerp{x}{z} + \innerp{y}{z} \\
                           	& (3) \quad \langle \bm{x}, \alpha \bm{y} \rangle = \alpha \innerp{x}{y} \\
                            & (4) \quad \innerp{x}{y} = \overline{\innerp{y}{x}} \\
                            & \forall \bm{x}, \bm{y}, \bm{z} \in \mathbb{C}^n, \alpha \in \mathbb{C} \\ \hline
ortogonaalisuus				& \innerp{x}{y} = 0 \\ \hline
Kroneckerin symboli			& \delta_{ij} \begin{eqgroup} 0 \quad i \neq j \\ 1 \quad i=j \end{eqgroup} \\ \hline
vektorijoukon ortogonaalisuus	& \langle \bm{x}_i, \bm{x}_j \rangle \begin{eqgroup} 0 \quad i \neq j \\ \neq 0 \quad i=j \end{eqgroup} \\
vektorijoukon ortonormaalius	& \langle \bm{x}_i, \bm{x}_j \rangle = \delta_{ij} \\ \hline
normi						& ||\bm{x}|| = \sqrt{\innerp{x}{x}} \\ \hline
normin ominaisuudet			& (1) \quad || \bm{x} || \geq 0 \land (|| \bm{x} || = 0 \leftrightarrow \bm{x} = \bm{0}) \\
							& (2) \quad || \alpha \bm{x} || = | \alpha | || \bm{x} || \quad \forall \alpha \in \mathbb{C}^n \\
kolmioepäyhtälö				& (3) \quad || \bm{x} + \bm{y} || \leq || \bm{x} || + || \bm{y} || \\ \hline
kolmioepäyhtälö alaspäin	& || \bm{x} - \bm{y} || \geq \big| || \bm{x} || - || \bm{y} || \big| \\ \hline
Cauchy-Schwarzin epäyhtälö	& |\innerp{x}{y}| \leq || \bm{x} || \cdot || \bm{y} || \\ \hline
vektorien välinen kulma		& \cos(\phi) = \frac{\innerp{x}{y}}{|| \bm{x} || \cdot || \bm{y} ||} \\
\end{eqtable}


\subsection{Matriisit}

\begin{eqtable}{Matriisien perusteet \cite[s. 16-26]{MAT-60000}}
lineaarikuvaus				& L(a \bm{x} + b \bm{y}) = a L(\bm{x}) + b L({\bm{y})} \\ \hline
matriisin indeksointi		& A=[a_{ij}]^{m\times n} = 
                			\begin{styledmatrix} a_{11}  & a_{12}  & \dots  & a_{1n} \\ 
							a_{21}  & a_{22}  & \dots  & a_{2n} \\ 
							\vdots  & \vdots & \ddots & \vdots \\ 
							a_{m1}  & a_{m2}  & \dots  & a_{mn} \end{styledmatrix} \\
                            
							& \text{m on korkeus (vaakarivien määrä)} \\
                           	& \text{n on leveys (pystyrivien määrä)} \\ \hline

neliömatriisi				& m=n \\
korkea matriisi				& m>n \\
leveä matriisi				& m<n \\ \hline
sarakevektorit				& A = \begin{styledmatrix} \bm{a}_1, \bm{a}_2, \ldots, \bm{a}_n\end{styledmatrix} \\
vaakarivivektorit			& A = \begin{styledmatrix} \bm{a}^T_1 \\ \bm{a}^T_2 \\ \vdots \\ \bm{a}^T_m \end{styledmatrix} \\ \hline
neliömatriisin jälki		& \text{tr}(A) = \sum^n_{i=1} a_{ii} \text{ eli diagonaalialkioiden summa} \\ \hline
matriisien summa			& A + B = (a_{ij}) + (b_{ij}) = (a_{ij} + b_{ij}) \\
skalaarilla kertominen		& \alpha A = (\alpha a_{ij}) \\
matriisien erotus			& A - B = A + (-1)B = (a_{ij}) + (-b_{ij}) = (a_{ij} - b_{ij}) \\ \hline
matriisien tulo				& C = (c_{ij})_{m \times n} = \big( \sum^p_{k=1} a_{ik} b_{kj} \big)_{m \times n} \\ \hline
matriisien laskusäännöt		& (A+B)+C = A+(B+C) \\
							& (AB)C = A(BC) \\
                            & A(B+C) = AB + AC \\
                            & (A+B)C = AC + BC \\
                            & A + B = B + A \\ \hline
neliömatriisin potenssit	& A^k = AA \cdots A \\
\end{eqtable}


\begin{eqtable}{Matriisityyppejä \cite[s. 18-21, 34]{MAT-60000}}
identiteettimatriisi		& I = \begin{styledmatrix}
							\bm{e}_1, \bm{e}_2, \ldots, \bm{e}_n
							\end{styledmatrix} = 
							\begin{styledmatrix}
							1 & 0 & \cdots & 0 \\
                            0 & 1 & \cdots & 0 \\
                            \vdots & \vdots & \ddots & \vdots \\
                            0 & 0 & \cdots & 1 \\
							\end{styledmatrix} \\ \hline
lohkomatriisi				& A =
							\begin{styledmatrix}
                            A_{11} & A_{12} & \cdots & A_{1q} \\
                            \vdots & & & \vdots \\
                            A_{p1} & A_{p2} & \cdots & A_{pq} \\
                            \end{styledmatrix} \\ \hline
                            
diagonaalimatriisi			& A = \text{diag}(a_{11}, a_{22}, \ldots, a_{nn}) = 
							\begin{styledmatrix}
                            a_{11} & 0 & \cdots & 0 \\
                            0 & a_{22} & \cdots & 0 \\
                            \vdots & \vdots & \ddots & \vdots \\
                            0 & 0 & \cdots & a_{nn}
                            \end{styledmatrix} \\ \hline

yläkolmiomatriisi			& (i>j \rightarrow a_{ij} = 0) \Leftrightarrow A =
							\begin{styledmatrix}
                            a_{11} & a_{12} & \cdots & a_{1n} \\
                            0 & a_{22} & \cdots & a_{2n} \\
                            \vdots & \vdots & \ddots & \vdots \\
                            0 & 0 & \cdots & a_{mn}
                            \end{styledmatrix}
							\\
alakolmiomatriisi			& (i<j \rightarrow a_{ij} = 0) \Leftrightarrow A = 
							\begin{styledmatrix}
                            a_{11} & 0 & \cdots & 0 \\
                            a_{21} & a_{22} & \cdots & 0 \\
                            \vdots & \vdots & \ddots & \vdots \\
                            a_{m1} & a_{m2} & \cdots & a_{mn}
                            \end{styledmatrix}
                            \\ \hline
lohkodiagonaalinen matriisi	& A = diag(A_{11}, A_{22}, \ldots, A_{pp})
							\begin{styledmatrix}
                            A_{11} & O & \cdots & O \\
                            O & A_{22} & \cdots & O \\
                            \vdots & \vdots & \ddots & \vdots \\
                            O & O & \cdots & A_{pp}
                            \end{styledmatrix} \\ \hline
lohkoalakolmiomatriisi		& A = 
							\begin{styledmatrix}
                            A_{11} & A_{12} & \cdots & A_{1q} \\
                            O & A_{22} & \cdots & A_{2q} \\
                            \vdots & \vdots & \ddots & \vdots \\
                            O & O & \cdots & A_{pq}
                            \end{styledmatrix}
							\\
lohkoyläkolmiomatriisi		& A =
							\begin{styledmatrix}
                            A_{11} & O & \cdots & O \\
                            A_{21} & A_{22} & \cdots & O \\
                            \vdots & \vdots & \ddots & \vdots \\
                            A_{p1} & A_{p2} & \cdots & A_{pq}
                            \end{styledmatrix}
							\\ \hline
permutaatiomatriisi			& P = \begin{styledmatrix}
							\bm{e}^T_{r_1} \\ \bm{e}^T_{r_1} \\ \vdots \\ \bm{e}^T_{r_n}
                            \end{styledmatrix}
							= \text{esim.} \begin{styledmatrix} 0&0&1 \\ 0&1&0 \\ 1&0&0 									\end{styledmatrix} \\
\end{eqtable}


\begin{eqtable}{\href{https://en.wikipedia.org/wiki/Determinant}{Determinantti} \cite[1.4.5]{MAT-60150}}
& \text{Määritelty, kun } A=[a_{ij}] \text{ on } n \times n \text{-matriisi} \\
\hline

n=2 & \det(A) =
\begin{vmatrix}
\setlength{\extrarowheight}{0pt}
a_{11} & a_{12} \\
a_{21} & a_{22}
\end{vmatrix}
= a_{11}a_{22}-a_{21}a_{12} \\

n=3	&
\det(A) = 
\begin{vmatrix}
\setlength{\extrarowheight}{0pt}
a_{11}	& a_{12}	& a_{13} \\
a_{21}	& a_{22}	& a_{23} \\
a_{31}	& a_{32}	& a_{33}
\end{vmatrix} \\
& =
a_{11}
\begin{vmatrix}
\setlength{\extrarowheight}{0pt}
a_{22} & a_{23} \\
a_{32} & a_{33}
\end{vmatrix}
-
a_{21}
\begin{vmatrix}
\setlength{\extrarowheight}{0pt}
a_{12} & a_{13} \\
a_{32} & a_{33}
\end{vmatrix}
+
a_{11}
\begin{vmatrix}
\setlength{\extrarowheight}{0pt}
a_{12} & a_{13} \\
a_{22} & a_{23}
\end{vmatrix}
\\

\hline
i:nnen rivin suhteen		& \det(A) = \sum_{j=1}^n (-1)^{i+j} a_{ij} \det(A_{ij}) \\
j:nnen sarakkeen suhteen	& \det(A) = \sum_{i=1}^n (-1)^{i+j} a_{ij} \det(A_{ij}) \\
							& \text{joissa } A_{ij} \text{ on } A \text{, josta on poistettu rivi i ja sarake j} \\

\end{eqtable}


\begin{eqtable}{Transpoosi ja konjugaattitranspoosi \cite[s. 21, 26]{MAT-60000}}
transpoosi					& A^T = (a_{ji})_{n \times m} =
							\begin{styledmatrix}
                            a_{11} & a_{21} & \cdots & a_{m1} \\
                            a_{12} & a_{22} & \cdots & a_{m2} \\
                            \vdots & \vdots & \ddots & \vdots \\
                            a_{1n} & a_{2n} & \cdots & a_{nm} \\
                            \end{styledmatrix} \\

konjugaattitranspoosi       & A^* = (\overline{a}_{ji})_{m \times n} =
							\begin{styledmatrix}
                            \overline{a}_{11} & \overline{a}_{21} & \cdots & \overline{a}_{m1} \\
                            \overline{a}_{12} & \overline{a}_{22} & \cdots & \overline{a}_{m2} \\
                            \vdots & \vdots & \ddots & \vdots \\
                            \overline{a}_{1n} & \overline{a}_{2n} & \cdots & \overline{a}_{nm} \\
                            \end{styledmatrix}\\ \hline

transpoosille				& (AB)^T = B^T A^T \\
							& (A+B)^T = A^T + B^T \\
                            & (\alpha A)^T = \alpha A^T \\ \hline

konjugaattitranspoosille	& (AB)^* = B^* A^* \\
							& (A+B)^* = A^* + B^* \\
							& (\alpha A)^* = \overline{\alpha} A^* \\
\end{eqtable}


\begin{eqtable}{Matriisien ominaisuuksia \cite[s. 27-30]{MAT-60000}}
kommutoivuus				& AB = BA \\ \hline

symmetrisyys				& A = A^T \\
hermiittisyys				& A = A^* \\
vinosymmetrisyys			& A = -A^T \\
vinohermiittisyys			& A = -A^* \\ \hline

singulaarisuus				& det(A) = 0 \\
ei-singulaarisuus			& det(A) \neq 0 \\
							& \exists A^{-1} \\ \hline

ortogonaalisuus				& A^T A = I = AA^T \\
unitaarisuus				& U^* U = U U^* = I \\ \hline

inverssi					& A A^{-1} = I = A^{-1} A \\
							& (AB)^{-1} = B^{-1} A^{-1} \\
                           	& (A^{-1})^{-1} = A \\
\end{eqtable}


\begin{eqtable}{Lineaarinen yhtälöryhmä \cite[s. 31]{MAT-60000}}
lineaarinen yhtälöryhmä		& A \bm{x} = \bm{b} \\
							& \bm{x} = A^{-1} \bm{b} \\
\end{eqtable}


\begin{eqtable}{Hermiittisille matriiseille \cite[s. 35]{MAT-60000}}
positiivisesti definiitti		& \langle \bm{x}, A \bm{x}\rangle > 0, \forall \bm{x} \neq \bm{0} \\
positiivisesti semidefiniitti	& \langle \bm{x}, A \bm{x}\rangle \geq 0, \forall \bm{x} \\
negatiivisesti definiitti		& \langle \bm{x}, A \bm{x}\rangle < 0, \forall \bm{x} \neq \bm{0} \\
negatiivisesti semidefiniitti	& \langle \bm{x}, A \bm{x}\rangle \leq 0, \forall \bm{x} \\
indefiniitti					& \text{Jos ei ole mitään näistä} \\
\end{eqtable}


\begin{eqtable}{LU-hajotelma \cite[s. 45-56]{MAT-60000}}
LU-hajotelma	& A = LU \\
				& L_k = I - \bm{l}_k \bm{e}^T_k \\
                & \bm{l}_k = \frac{1}{x_k} \begin{styledmatrix} 0 \\ \vdots \\ 0 \\ x_{k+1} \\ \vdots \\ x_n \end{styledmatrix}, x_k \neq 0 \\
                & L_{n-1} L_{n-2} \cdots L_1 A = U \\
                & \hat{L} = L_{n-1} L_{n-2} \cdots L_1 \\
                & A = \hat{L}^{-1} U = LU \\ \hline
Jos ei onnistu suoraan, niin viimeistään	& PA = LU, \text{ jossa P on permutaatiomatriisi} \\ \hline
lineaarisen yhtälöryhmän ratkaiseminen	& A \bm{x} = P^T LU \bm{x} = \bm{b} \\
											& LU \bm{x} = P \bm{b} \\
                                            & L \bm{y} = P \bm{b} \land U \bm{x} = \bm{y} \\
                                            & x = U^{-1} \bm{y} \land \bm{y} = L^{-1} P \bm{b} \\
\end{eqtable}


\begin{eqtable}{\href{https://en.wikipedia.org/wiki/Linear_subspace}{Aliavaruus} \cite[s. 59-83]{MAT-60000}}
aliavaruus					& \alpha \bm{x} + \beta \bm{y} \in S \quad \forall \bm{x}, \bm{y} \in \mathcal{S} \\
arvojoukko (kuva-avaruus)	& \mathcal{R} (A) = \{y | \exists \bm{x} \in \mathbb{F}^n s.e. \bm{y} = A \bm{x} \} \in \mathbb{F}^m \\
ydin (nolla-avaruus)		& \mathcal{N} (A) = \{ \bm{x} | A \bm{x} = \bm{0} \} \\ \hline

ortogonaalikomplementti		& \mathcal{S}^\perp = \{ \bm{y} | \langle \bm{y}, \bm{x} \rangle = 0, \forall \bm{x} \in \mathcal{S} \} \\ \hline

							& \mathcal{N} (A) = \mathcal{R} (A^*)^\perp \\
                            & \mathcal{N} (A^*) = \mathcal{R} (A)^\perp \\ \hline

dimensio eli kantavektorien lukumäärä	& \dim (S) \\
matriisin aste							& rank (A) = \dim \mathcal{R} (A) \\
										& rank (A) + \dim \mathcal{N} (A) = n = \text{ sarakkeiden määrä} \\ \hline

projektorimatriisi			& P^2 = P \\
							& \text{Projisoi vektorit } \mathcal{R} (P) \text{:lle pitkin } \mathcal{N} (P) \text{:tä} \\
\end{eqtable}


\begin{eqtable}{\href{https://en.wikipedia.org/wiki/Eigenvalues_and_eigenvectors}{Ominaisarvot ja -vektorit} \cite[s. 90-100]{MAT-60000}}
ominaisarvot ja -vektorit	& A \bm{x} = \lambda \bm{x}, \bm{x} \neq \bm{0} \\
ominaisarvot				& \det (A - \lambda I) = 0 \\
karakteristinen polynomi	& \prod^n_{k=1} (\lambda - \lambda_k) \\
spektri						& \sigma (A) = \{ \lambda_ 1, \lambda_2, \ldots , \lambda_n \} \\
similaarisuus				& B = S^{-1} A S \text{ jossa S ei-singulaarinen}\\
unitaarinen similaarisuus	& B = U^* AU \\

Hermiten matriisille		& diag(\sigma(A)) = U^* A U \\
							& \forall \lambda \in \mathbb{R} \land n \text{ ortonormaalia ominaisvektoria} \Rightarrow \text{Hermiten matriisi} \\ \hline

Normaali matriisi			& A = UDU^* \text{ jossa D on diagonaalimatriisi} \\
\end{eqtable}


\begin{eqtable}{\href{https://en.wikipedia.org/wiki/Jordan_normal_form}{Jordanin kanoninen muoto} \cite[s. 101-107]{MAT-60000}, \cite[5]{MAT-60150}}
				& A = SJS^{-1} \\
				& J = diag(J_1, J_2, \ldots, J_p) \\
defektiivisyys	& \exists \lambda : geom(\lambda) < alg(\lambda) \\ \hline

% Old version ---
\iffalse
				& A \bm{s}^j_1 = \lambda_j \bm{s}^j_1 \\
                & A \bm{s}^j_2 = \lambda_j \bm{s}^j_2 \bm{s}^j_1 \\
                & A \bm{s}^j_i = \lambda_j \bm{s}^j_i \bm{s}^j_{i-1} \\
Jordanin ketju	& \{ \bm{s}^j_1, \bm{s}^j_2, \ldots \bm{s}^j_{r_j} \} \\
ominaisvektori	& \bm{s}^j_1 \\
yleistetyt ominaisvektorit	& \bm{s}^j_2, \ldots \bm{s}^j_{r_j} \\
kompleksiselle ominaisarvolle	& J_i = \begin{styledmatrix} \alpha & \beta \\ - \beta & \alpha \end{styledmatrix} \in \mathbb{R}^{2 \times 2} \\ \hline
käyttöohje		& \text{Etsi spektri } \sigma (A) = \{ \lambda_1, \lambda_2, \ldots \lambda_k \} \\
				& \text{Valitse mielivaltainen } \lambda \\
                & alg(\lambda) = 1 \Rightarrow \bm{s}^1_1 = \bm{x}_1 \land J_1 = \lambda_1 \\
                & alg(\lambda) = geom(\lambda) = l \Rightarrow \text{Otetaan ominaisvektorit} \land \forall J = \lambda_1 \\
                & alg(\lambda) > geom(\lambda) \Rightarrow \text{Muodostetaan ketjut erikseen} \\
               	& \lambda \text{ kompleksinen} \Rightarrow (\bm{s}^1_1 = \text{Re } \bm{x}, \bm{s}^1_2 = \text{Im } \bm{x}) \land (\overline{\lambda} \in \sigma(A)) \\
\fi
%---

				& A \bm{u}_1 = \lambda_j \bm{u}_1 \\
                & A \bm{u}_2 = \lambda_j \bm{u}_2 \bm{u}_1 \\
                & A \bm{u}_i = \lambda_j \bm{u}_i \bm{u}_{i-1} \\

Jordanin ketju	& \{ \bm{u}_1, \bm{u}_2, \ldots \bm{u}_{m_j} \} \\
ominaisvektori	& \bm{u}_1 \\
yleistetyt ominaisvektorit	& \bm{u}_2, \ldots \bm{u}_{m_j} \\

kompleksiselle ominaisarvolle	& J_i = \begin{styledmatrix} \alpha & \beta \\ - \beta & \alpha \end{styledmatrix} \in \mathbb{R}^{2 \times 2} \\ \hline
käyttöohje		& \text{Etsi spektri } \sigma (A) = \{ \lambda_1, \lambda_2, \ldots \lambda_k \} \\
				& \text{Valitse mielivaltainen } \lambda \\
                & alg(\lambda) = 1 \Rightarrow \bm{u}_1 = \bm{x}_1 \land J_1 = \lambda_1 \\
                & alg(\lambda) = geom(\lambda) = l \Rightarrow \text{Otetaan ominaisvektorit} \land \forall J = \lambda_1 \\
                & alg(\lambda) > geom(\lambda) \Rightarrow \text{Muodostetaan ketjut erikseen} \\
               	& \lambda \text{ kompleksinen} \Rightarrow (\bm{u}_1 = \text{Re } \bm{x}, \bm{u}_2 = \text{Im } \bm{x}) \land (\overline{\lambda} \in \sigma(A)) \\
\end{eqtable}


\begin{eqtable}{\href{https://en.wikipedia.org/wiki/Singular_value_decomposition}{Singulaariarvohajotelma} \cite[s. 117-128]{MAT-60000}}
singulaariarvohajotelma		& U^*AV = \Lambda \\
							& A = U \Lambda V^* \\
							& \Lambda = diag(\sigma_1, \sigma_2, \ldots , \sigma_r, 0, \ldots, 0) \\
                            & \sigma(A^*A) = \{ \sigma^2_1, \sigma^2_2, \ldots, \sigma^2_{r}, \sigma^2_{r+1}, \sigma^2_n \} \\
                            & V = [\bm{v}_1, \bm{v}_2, \ldots, \bm{v}_n] \text{ jossa v:t ortonormaaleja ominaisvektoreita} \\
                            & \bm{u}_i = \frac{A \bm{v}_i}{\sigma_i} \\
                            & \text{Toimii samoin, vaikka valittaisiin } AA^* \\ \hline
                            & rank(A) = r \\
                            & \mathcal{R}(A) = span\{ \bm{u}_1, \bm{u}_2, \ldots, \bm{u}_r \} \\
                            & \mathcal{N}(A) = span\{ \bm{v}_{r+1}, \bm{v}_{r+2}, \ldots, \bm{v}_n \} \\ \hline
                            
                            & || A || = \sigma_1 \\
Jos A on ei-singulaarinen	& \sigma_n || \bm{x} || \leq || A \bm{x} || \leq \sigma_1 || \bm{x} ||, \forall \bm{x} \in \mathbb{F}^n \\
							& || A^{-1} || = \frac{1}{\sigma_n} \\ \hline

approksimointi				& B = U diag(\sigma_1, \sigma_2, \ldots, \sigma_{r-1}, 0, \ldots, 0) V^* \\
							& ||A - B|| = \sigma_r \\ \hline

pseudoinverssi				& A^\dagger = V \Lambda^\dagger U^*\\
							& \Lambda^\dagger \text{ on } \Lambda^T, \text{jossa } \sigma_i \Rightarrow \frac{1}{\sigma_i} \\
                            & AA^\dagger A = A \\
                            & A^\dagger A A^\dagger = A^\dagger \\
lin. yhtälöryhmän 			& A\bm{x} = \bm{b} \\
yleinen ratkaisu			& \bm{x} = A^\dagger \bm{b} \\
\end{eqtable}



\clearpage

\section{Differentiaaliyhtälöt}

\begin{eqtable}{Differentiaaliyhtälöiden termejä ja kaavoja \cite{MAT-60150}}
% \cite[s. 34]{MAT-60150}
homogeeninen alkuarvo-ongelma 	& \bm{x}' = A\bm{x}, \quad \bm{x}(0) = \bm{c} \\

% \cite[s. 33]{MAT-60150}
alkuarvo-ongelma 				& \bm{x}' = A\bm{x} + \bm{f}, \quad \bm{x}(0) = \bm{c} \\

% \cite[2.4.5]{MAT-60150}
vakioinvariointikaava 			& \bm{x}(t) = e^{tA} \bm{c} + \int_0^t e^{(t-s)A} \bm{f}(s) ds \\

% \cite[1.5.3]{MAT-60150}
Wronskin determinantti			& W(t) = \det([ \bm{x}_1(t), \bm{x}_2(t), \ldots , \bm{x}_n(t)]) \\
lineaarinen riippumattomuus		& \forall t \in I: W(t) \neq 0 \\

% \cite[1.4.2]{MAT-60150}
matriisin derivaatta			& \frac{d^n}{dt^n} A = \Big[ \frac{d^n a_{ij}}{dt^n} \Big] \\
matriisin integraali			& \int_a^b A(t)dt = \Big[ \int_a^b a_{ij}(t)dt \Big] \\

% \cite[2]{MAT-60150}
eksponenttimatriisi				& e^A = \sum_{k=0}^\infty \frac{A^k}{k!} \\
								& \forall A \in \mathbb{R}^{n \times n}: \quad \exists e^A \in \mathbb{R}^{n \times n} \\
								& ||e^A|| \leq e^{||A||} \\
								& AB=BA \Rightarrow e^{A+B} = e^A e^B \\
								& (e^A)^{-1} = e^{-A} \\

nilpotentti matriisi			& \exists m \in \mathbb{N}: N^m = O \\

% \cite[s. 39]{MAT-60150}
								& e^N = \sum_{k=0}^{m-1} \frac{N^k}{k!} = I_n + N + \cdots + \frac{N^{m-1}}{(m-1)!} \\

% \cite[2.3.7]{MAT-60150}
nilpotentin matriisin käyttö 	& A = S + N \land N \text{ on nilpotentti} \Rightarrow e^A = e^S e^N \\
\end{eqtable}


\begin{table}
\centering
\caption{Reaalisen derivoinnin ja integroinnin laskusääntöjä \cite[1.4.3, 1.4.4, 1.4.5]{MAT-60150}}
\setlength{\extrarowheight}{10pt}
\begin{tabular}{| >{$} l <{$} |}
\hline
A'=O \Leftrightarrow A \text{ on vakio} \\
(AB)' = A'B + AB' (\text{tulon derivointisääntö}) \\
\int_a^b (A(t) + B(t))dt = \int_a^b A(t) dt + \int_a^b B(t) dt \\
\int_a^b C A(t)dt = C \int_a^b A(t)dt \\
\int_a^b A(t) D dt = \int_a^b A(t) dt D \\
t \in (a,b) \Rightarrow \frac{d}{dt} \int_a^t A(s)ds = A(t) \quad (\text{integraalin derivointilause}) \\
\int_a^b A'(t)dt = A(b) - A(a) \quad (\text{analyysin peruslause}) \\
\int_a^b A'(t)B(t)dt = A(b)B(b) - A(a)B(a) - \int_a^b A(t)B'(t)dt \quad (\text{osittaisintegrointikaava}) \\
\hline
\end{tabular}
\end{table}


\begin{eqtable}{\href{https://en.wikipedia.org/wiki/Matrix_norm}{Matriisinormi} \cite[2.1]{MAT-60150}}
oletukset				& A, B \in \mathbb{R}^{n \times n}, \quad \alpha \in \mathbb{R}, \quad \bm{x} \in \mathbb{R}^n \\ \hline
matriisinormi			& ||A|| = \max_{|\bm{x}| \leq 1} | A \bm{x} | \\
suljettu yksikköpallo	& \overline{B}^n = \{\bm{x} \in \mathbb{R}^n: |\bm{x}| \leq 1 \} \\
ominaisuuksia			& ||A|| \geq 0 \land (||A|| = 0 \Leftrightarrow A=O) \\
						& ||A+B|| \leq ||A||+||B|| \quad (\text{kolmioepäyhtälö}) \\
						& ||\alpha A|| = |\alpha| ||A|| \\
						& |A \bm{x}| \leq ||A|| |\bm{x}| \quad \forall \bm{x} \in \mathbb{R}^n \\
						& ||AB|| \leq ||A|| ||B|| \\
						& k \in \mathbb{N}: ||A^k|| \leq ||A||^k \\
						& \text{Jos } \sigma^2 \text{ on matriisin } A^T A \text{ suurin ominaisarvo, niin } ||A||=|\sigma | \\
\end{eqtable}


\begin{table}
\centering
\caption{Eksponenttimatriiseja}
\setlength{\extrarowheight}{10pt}
\begin{tabular}{| >{$} l <{$} | >{$} l <{$} | l |}
\hline

A = \begin{styledmatrix}
\lambda_1 \\
& \lambda_2 \\
&& \ddots \\
&&& \lambda_n
\end{styledmatrix}
&
e^A = \begin{styledmatrix}
e^{\lambda_1} \\
& e^{\lambda_2} \\
&& \ddots \\
&&& e^{\lambda_n}
\end{styledmatrix}
&
\cite[2.3.2]{MAT-60150} \\

T_\mu =
\begin{styledmatrix}
\alpha & \beta \\
-\beta & \alpha
\end{styledmatrix}
&
e^{tT_\mu} = e^{t\alpha}
\begin{styledmatrix}
\cos \beta	& \sin \beta \\
-\sin \beta	& \cos \beta
\end{styledmatrix}
&
\cite[2.3.5]{MAT-60150}
% Proposition 2.3.4 is merely a special case where \alpha = 0
\\

A = \begin{styledmatrix}
\alpha	& \gamma \\
0		& \alpha
\end{styledmatrix}
&
e^A = e^\alpha \begin{styledmatrix}
1	& \gamma \\
0	& 1
\end{styledmatrix}
&
\cite[2.3.8]{MAT-60150} \\

A = PJP^{-1} (\text{Jordanin kanoninen muoto})
&
e^{tA} = Pe^{tJ} P^{-1}
&
\cite[3.1.4]{MAT-60150} \\

J_\lambda =
\begin{styledmatrix}
\lambda & 1 \\
& \lambda & 1 \\
& & \ddots & \ddots \\
& & & \lambda & 1 \\
& & & & \lambda
\end{styledmatrix}
&
e^{tJ_\lambda} = e^{t\lambda}
\begin{styledmatrix}
1		& t			& \frac{t^2}{2}	& \cdots	& \frac{t^{k-1}}{(k-1)!} \\
		& 1			& t				& \cdots	& \frac{t^{k-2}}{(k-2)!} \\
		& 			& \ddots		&			& \vdots \\
		& 			&				& 1			& t \\
		& 			&				&			& 1
\end{styledmatrix}
&
\cite[5.5.1]{MAT-60150} \\

T_\mu =
\begin{styledmatrix}
D_{\mu} & I_2 \\
& D_{\mu} & I_2 \\
& & \ddots & \ddots \\
& & & D_{\mu} & I_2 \\
& & & & D_{\mu}
\end{styledmatrix}
&
e^{tT_\mu} =
\begin{styledmatrix}
e^{tD_\mu}	& te^{tD_\mu}	& \frac{t^2}{2!} e^{tD_\mu}	& \cdots		& \frac{t^{k-1}}{(k-1)!} e^{tD_\mu} \\
			& e^{tD_\mu}	& te^{tD_\mu}				& \cdots		& \frac{t^{k-2}}{(k-2)!} e^{tD_\mu} \\
			&				& \ddots					&				& \vdots \\
			&				&							& e^{tD_\mu}	& te^{tD_\mu} \\
			&				&							&				& e^{tD_\mu}
\end{styledmatrix}
& \cite[5.5.2]{MAT-60150} \\

\hline
\end{tabular}
\end{table}


\begin{eqtable}{Ominaisavaruudet \cite[5]{MAT-60150}}
ominaisavaruus				& E_\lambda = \{ \bm{x}: A \bm{x} = \lambda \bm{x} \} \\ % This is not from the pruju(source)
realisoitu ominaisavaruus	& F_\mu = \text{Re}(E_\mu) \cup \text{Im}(E_\mu) = \text{span}_\mathbb{R} \{ \text{Re}(\bm{v}), \text{Im}(\bm{v}) \} \\ % Would it be more logical to use \lambda instead of \mu ?
yleistetty ominaisavaruus	& K_\lambda = \{ \bm{x} \in \mathbb{C}^n: \bm{x} \text{ on yleistetty ominaisvektori} \} \\
							& E_\lambda \subset K_\lambda \\
							& K_{\lambda_i} \cap K_{\lambda_j} = \{0 \} \\
							& K_\lambda = N \big( (A- \lambda I_n)^{alg(\lambda)} \big) \\
\end{eqtable}

\begin{eqtable}{Jordanin kanoninen muoto yleisessä tilanteessa \cite[5.4.1]{MAT-60150}}
spektri			& \sigma(A) = \{ \lambda_1, \ldots, \lambda_k, \mu_1, \overline{\mu}_1, \ldots, \mu_q, \overline{\mu}_q \} \\
$\lambda_j$:tä vastaava Jordanin ketju	& j^(\lambda_j) = \{ \bm{v}^{(\lambda_j)}_1, \ldots, \bm{v}^{(\lambda_j)}_{m_j} \} \\
$\mu_r$:ää vastaava Jordanin ketju		& \{ \bm{u}^{(\mu_r)}_1, \ldots, \bm{u}^{(\mu_r)}_{n_r} \} \\
										& t^{(\mu_j)} = \{ \text{Re}( \bm{u}^{(\mu_r)}_1 ), \text{Im}( \bm{u}^{(\mu_r)}_1 ), \ldots, \text{Re}( \bm{u}^{(\mu_r)}_{n_r} ), \text{Im}( \bm{u}^{(\mu_r)}_{n_r} ) \} \\
										& P = [j^{(\lambda_1)} \cdots j^{(\lambda_k)} t^{(\mu_1)} \cdots t^{(\mu_q)} ] \\

lohkodiagonaalimatriisi	& J = P^{-1}AP =
\begin{styledmatrix}
J_{\lambda_1} \\
& \ddots \\
& & J_{\lambda_k} \\
& & & T_{\mu_1} \\
& & & & \ddots \\
& & & & & T_{\mu_q} \\
\end{styledmatrix}
\\

Jordanin lohkot	& J_{\lambda_j} = 
\begin{styledmatrix}
\lambda & 1 \\
& \lambda & 1 \\
& & \ddots & \ddots \\
& & & \lambda & 1 \\
& & & & \lambda
\end{styledmatrix}
\\

& T_{\mu_r} =
\begin{styledmatrix}
D_{\mu_r} & I_2 \\
& D_{\mu_r} & I_2 \\
& & \ddots & \ddots \\
& & & D_{\mu_r} & I_2 \\
& & & & D_{\mu_r}
\end{styledmatrix}
\\

& D_{\mu_r} =
\begin{styledmatrix}
\alpha_r	& \beta_r \\
-\beta_r	& \alpha_r
\end{styledmatrix}
\text{, kun } \mu_r = \alpha_r + i \beta_r
\\

\end{eqtable}


\begin{eqtable}{Usean värähtelijän systeemi \cite[6.1]{MAT-60150}}
usean värähtelijän systeemi	& M \bm{x}'' + C \bm{x}' + K \bm{x} = \bm{f}, \quad \bm{x}(0)=\bm{c}_1, \quad \bm{x}'(0)=\bm{c}_2 \\
ratkaistavassa muodossa	& \bm{z}' =
\begin{styledmatrix}
O	& I_n \\
-M^{-1}K	& -M^{-1}C
\end{styledmatrix}
\bm{z}
+
\begin{styledmatrix}
\bm{0} \\
M^{-1} \bm{f}
\end{styledmatrix}
, \quad
\bm{z}(0) =
\begin{styledmatrix}
\bm{c}_1 \\
\bm{c}_2
\end{styledmatrix}
\\

\end{eqtable}


\begin{eqtable}{Stabiilius \cite[6.2]{MAT-60150}}
matriisifunktio on stabiili, kun	& \lim_{t \rightarrow \infty} ||e^{tA} || = 0 \\
jos $t \mapsto e^{tA}$ on stabiili, niin	& \lim_{t \rightarrow \infty} |e^{tA} \bm{c} = 0 \quad \forall \bm{c} \in \mathbb{R}^n \\
Lyapunovin lause	& \text{matriisifunktio } t \mapsto e^{tA} (A \in \mathbb{R}^{n \times n}) \text{ on stabiili}  \Leftrightarrow \forall \lambda: \text{Re} \lambda < 0 \\
\end{eqtable}


\begin{eqtable}{Lineaariset järjestelmät \cite[6.3]{MAT-60150}}
Lineaarinen järjestelmä
&
\begin{eqgroup}
\bm{x}' = A\bm{x} + B\bm{u} \\
\bm{y} = C\bm{x} + D\bm{u}
\end{eqgroup}
\\

oletukset	& \bm{x}: [0, \infty) \rightarrow \mathbb{R}^n,
	\quad \bm{y}: [0, \infty) \rightarrow \mathbb{R}^m,
	\quad \bm{u}: [0, \infty) \rightarrow \mathbb{R}^k,
	\quad \bm{x}(0) = \bm{c} \\

vakionvariointikaavasta & \bm{x}(t) = e^{tA} \bm{c} + \int_0^t e^{(t-s)A} B\bm{u}(s)ds \\
						& \bm{y}(t) = Ce^{tA} \bm{c} + \int_0^t e^{(t-s)A} B\bm{u}(s)ds + D\bm{u} \\

tavoite				& \lim_{t \rightarrow \infty} | \bm{y}(t) - \bm{y}_r(t)| = 0 \\
systeemin ohjaus	& \bm{u} = K \bm{x} + L \\
					& \Rightarrow \bm{x}' = (A+BK)\bm{x} + BL \\
\end{eqtable}


\begin{eqtable}{\href{https://en.wikipedia.org/wiki/Laplace_transform}{Laplace-muunnos} \cite[6.4]{MAT-60150}}
\multicolumn{2}{|l|}{Matriisifunktio $A: [0,\infty) \rightarrow \mathbb{R}^{n \times m}$ on eksponentiaalista kertalukua vakiolla $k$, jos} \\
\multicolumn{2}{|l|}{ $\exists M>0, K \in \mathbb{R}$ s.e. $\forall i,j: \big| [A]_{ij}| \leq Me^{kt}, $ kun $ t \geq 0$} \\
\hline
Laplace-muunnos & \laplace{A} (s) = \int_0^\infty A(t) dt \\
määritelty joukossa	& D = \{ s \in \mathbb{C} : \text{Re}(s)>k \} \\
\hline
injektiivisyys	& \laplace{A} = \laplace{B} \Rightarrow A=B \\
				& \Rightarrow \exists \laplace{}^{-1} \\
\hline
ominaisuuksia	& (\text{oletukset: } A,B: [0, \infty) \rightarrow \mathbb{R}^{n \times n} \text{eksponentiaalista kertalukua olevia} \\
				& \text{matriisifunktioita ja } U,V \in \mathbb{R}^{n \times n}) \\
				& \text{funktio } U A(t)+B(t)V \text{ on eksponentiaalista kertalukua } \\
				& \land \laplace{UA+BV} = U\laplace{A}+\laplace{B}V \\
				& \laplace{e^{tA} A(t)}(s) = \laplace{A}(s-a) \\
				& \laplace{A(t+a)}(s) = e^{sa} \laplace{A}(s) \\
				& \laplace{A}(s) = [l_{ij}(s;A)] \rightarrow \text{komplementtifunktiot } l_{ij}(s;A) \text{ ovat} \\
				& \text{määrittelyjoukossaan kompleksitason analyyttisia funktioita} \\
				& \text{Re}(s)>||A|| \rightarrow \laplace{e^{tA}}(s) = (sI_n-A)^{-1} \\	% Lause 6.4.7
\hline
konvoluutio		& (\text{oletukset: } A: [0, \infty) \rightarrow \mathbb{R}^{n \times m}, B: [0, \infty) \rightarrow \mathbb{R}^{m \times k}) \\
				& A*B: [0, \infty) \rightarrow \mathbb{R}^{n \times k} \\
				& A*B(t) = \int_0^t A(u)B(t-u)du = \int_0^t A(t-u)B(u)du \\
				& \laplace{A*B} = \laplace{A}\laplace{B} \\
\hline
	& (\text{oletus: } \bm{x}: [0, \infty) \rightarrow \mathbb{R}^n kahdesti derivoituva \\
	& \laplace{\bm{x}'}(s) = s \laplace{\bm{x}}(s)-\bm{x}(0) \\
	& \laplace{\bm{x}''}(s) = s^2 \laplace{\bm{x}}(s) - s\bm{x}(0) - \bm{x}'(0) \\
\hline
Neumannin sarja	& \text{Olkoon } B \in \mathbb{R}^{n \times n} \text{ matriisi, jolle } ||B||<1 \Rightarrow I_n-B \text{ on kääntyvä } \land \\
				& (I_n - B)^{-1} = \sum_{k=0}^\infty B^k = I_n + B + B^2 + B^3 + \ldots \\

\end{eqtable}


\begin{eqtable}{\href{https://en.wikipedia.org/wiki/Vector_field}{Vektorikentät} \cite[7.1]{MAT-60150}}
vektorikenttä	& \vec{V}: \Omega \rightarrow \mathbb{R}^n; \bm{x} \mapsto \vec{V}(\bm{x}) \\
käyrä			& \gamma: I \rightarrow \mathbb{R}^n \\
tangenttivektorikenttä	& \vec{V}(\gamma(t)) := \gamma'(t) \\
\hline
liikkuva massapiste	& \text{avaruudessa } \mathbb{R}^3 \text{pitkin käyrää} \bm{r}(t), t \in [0,\infty) \\
nopeus		& \vec{V}(\bm{r}(t)) = \bm{r}'(t) \\
kiihtyvyys	& \vec{A}(\bm{r}(t)) = \bm{r} \\
\hline
lineaarinen vektorikenttä	& \vec{V}(\bm{x}) = A\bm{x} \\
integraalikäyrä lin. vektorikentässä	& \bm{x}(t) = e^{tA} \bm{c} \\
virtaus	& \phi: I \times \Omega \rightarrow \mathbb{R}^n \\
virtaukselle	& \text{Määrittelyjoukko } I \text{ on laajin mahdollinen} \\
				& \phi(t, \bm{c}) \text{ on pisteen } \bm{c} \in \Omega \text{ kautta kulkeva} \\
				& \text{vektorikentän } \vec{V} \text{ integraalikäyrä} \\
				& \Big( (t, \bm{c}) \in I \times \Omega \quad \land \quad (s, \phi(t, \bm{c})) \in I \times \Omega \Big) \\
				& \Rightarrow \Big( (t+s, \bm{c} \in I \times \Omega \quad \land \quad \phi(t+s, \bm{c}) = \phi(s, \phi(t, \bm{c})) \Big) \\
lineaarisen vektorikentän virtaus	& \phi: \mathbb{R} \times \mathbb{R}^n \rightarrow \mathbb{R} \\
									& \phi(t, \bm{c}) = e^{tA} \bm{c} \\
tasapainopiste	& \vec{V}(\overline{\bm{x}}) = \bm{0} \\
tpp.:n integraalikäyrä				& \gamma_{\bm{x}}: I \rightarrow \mathbb{R}^n \\
stabiili tasapainopiste	& \forall \epsilon > 0 \exists \delta > 0: \Big( | \gamma_{\bm{x}} (t)-\overline{\bm{x}}| < \epsilon \quad \forall t \in I, \text{ kun } | \bm{x} - \overline{\bm{x}} | < \delta \Big) \\
\end{eqtable}

\begin{eqtable}{Nielut, lähteet ja hyperboliset virtaukset \cite[7.2, 7.3]{MAT-60150}}
origo nieluna	& \forall \lambda \in \sigma(A): \text{Re}(\lambda) < 0 \\
				& \Rightarrow \text{kontraktio} \\
origo lähteenä	& \forall \lambda \in \sigma(A): \text{Re}(\lambda) > 0 \\
				& \Rightarrow \text{ekspansio} \\
\hline
jos origo on nielu	& \exists K, b > 0: \quad |e^{tA} \bm{c} | \leq Ke^{-tb} |\bm{c}| \quad \forall t \geq 0, \bm{c} \in \mathbb{R}^n \\
jos origo on lähde	& \exists L, c > 0: \quad |e^{tA} \bm{c} | \geq Le^{tc} |\bm{c}| \quad \forall t \geq 0, \bm{c} \in \mathbb{R}^n \\
\hline
hyperbolinen virtaus	& \forall \lambda \in \sigma(A): \text{Re}(\lambda) \neq 0 \\
						& \forall A \in \mathbb{R}^{n \times n} \text{, jolle } \phi_t=e^{tA} \text{on hyperbolinen}: \\
						\hline
						& \exists \text{ yksikäsitteiset aliavaruudet } E^s, E^u \subset \mathbb{R}^n: \mathbb{R}^n = E^s \oplus E^u \\
						& \text{ja näitä vastaava similaarimuunnnos } P^{-1}AP= \begin{styledmatrix} A_k \\ & A_e \end{styledmatrix},\\
						& \text{jossa virtaus } e^{tA_k} \text{ on kontraktio ja } e^{tA_c} \text{ ekspansio} \\
						\hline
$E^s$, $E^u$ ovat A ja $e^{tA}$-invariantteja	& AE^s \subset E^s, AE^u \subset E^u, \quad e^{tA}E^s \subset E^s, e^{tA}E^u \subset E^u, \quad t \in \mathbb{R} \\
\end{eqtable}

\begin{eqtable}{\href{https://en.wikipedia.org/wiki/Generic_property}{Geneeriset ominaisuudet} \cite[7.4]{MAT-60150}}
(oletukset)		& X \subset \mathbb{R}^{n \times n}, \quad A \in \mathbb{R}^{n \times n},\quad  U \in \mathbb{R}^{n \times n}, \quad X \in \mathbb{R}^{n \times n} \\
\hline
tiheä joukko	& \forall \epsilon > 0 \land A \in \mathbb{R}^{n \times n} \exists B \in X: ||A-B|| \leq \epsilon \\
A-keskipisteinen r-säteinen pallo	& B_r(A) = \{ C \in \mathbb{R}^{n \times n}: ||A-C|| \leq r \} \\
avoin joukkko	& U: \Big( \forall A \in U \exists r > 0: B_r (A) \subset U \Big) \\
\hline
				& \forall X_i \text{ avoimia ja tiheitä } \Rightarrow X=X_1 \cap \cdots \cap X_m \text{ on avoin ja tiheä} \\
				\hline
geneerinen ominaisuus	& \exists \text{tiheä ja avoin osajoukko, jossa se on voimassa} \\
\end{eqtable}

\clearpage

\section{Numeerisia taulukoita}

\begin{table}[ht!]
\centering
\caption{\href{https://en.wikipedia.org/wiki/Trigonometric_constants_expressed_in_real_radicals}{Trigonometristen funktioiden tarkkoja arvoja} \cite[s. 54-55]{MAOL}}
\end{table}

{
\tabulinesep=1mm
\begin{longtabu} to \linewidth {l | >{$ \displaystyle} l <{$} | >{$ \displaystyle} X <{$} | >{$ \displaystyle} X <{$} | >{$ \displaystyle} X <{$} }

asteet	& \text{radiaanit}	& \sin	& \cos	& \tan \\
\hline
\endhead

% First and second columns have been generated using SymPy http://docs.sympy.org/latest/tutorial/printing.html
0		& 0					& 0									& 1									& 0 \\
15		& \frac{\pi}{12}	& \frac{1}{4} (\sqrt{6}-\sqrt{2})	& \frac{1}{4}(\sqrt{6}+\sqrt{2})	& 2-\sqrt{3} \\
18		& \frac{\pi}{10}	& \frac{1}{4} (\sqrt{5}-1)			& \frac{1}{4}\sqrt{10+2\sqrt{5}})	& \frac{1}{5}\sqrt{25-10\sqrt{5}} \\
22,5	& \frac{\pi}{8}		& \frac{1}{2} \sqrt{2-\sqrt{2}}		& \frac{1}{2}\sqrt{2+\sqrt{2}}		& \sqrt{2}-1 \\
30		& \frac{\pi}{6}		& \frac{1}{2}						& \frac{\sqrt{3}}{2}				& \frac{1}{\sqrt{3}} \\
36		& \frac{\pi}{5}		& \frac{1}{4} \sqrt{10-2\sqrt{5}}	& \frac{1}{4}(\sqrt{5}+1)			& \sqrt{5-2\sqrt{5}} \\
45		& \frac{\pi}{4}		& \frac{1}{\sqrt{2}}				& \frac{1}{\sqrt{2}}				& 1 \\
54		& \frac{3 \pi}{10}	& \frac{1}{4}(\sqrt{5}+1)			& \frac{1}{4}\sqrt{10-2\sqrt{5}}	& \frac{1}{5}\sqrt{25+10\sqrt{5}}\\
60		& \frac{\pi}{3}		& \frac{\sqrt{3}}{2}				& \frac{1}{2}						& \sqrt{3} \\
67,5	& \frac{3\pi}{8}	& \frac{1}{2} \sqrt{2+\sqrt{2}}		& \frac{1}{2}\sqrt{2-\sqrt{2}}		& \sqrt{2}+1 \\
72		& \frac{2 \pi}{5}	& \frac{1}{4} \sqrt{10+2\sqrt{5}}	& \frac{1}{4}(\sqrt{5}-1)			& \sqrt{5+2\sqrt{5}} \\
75		& \frac{5 \pi}{12}	& \frac{1}{4}(\sqrt{6}+\sqrt{2})	& \frac{1}{4}(\sqrt{6}-\sqrt{2})	& 2+\sqrt{3} \\
90		& \frac{\pi}{2}		& 1									& 0									& \text{ei määritelty} \\
105		& \frac{7 \pi}{12}	& \frac{1}{4}(\sqrt{6}+\sqrt{2})	& -\frac{1}{4}(\sqrt{6}-\sqrt{2})	& -(2+\sqrt{3}) \\
108		& \frac{3 \pi}{5}	& \frac{1}{4}\sqrt{10+2\sqrt{5}})	& -\frac{1}{4}(\sqrt{5}-1)			& -\sqrt{5+2\sqrt{5}} \\
112,5	& \frac{5\pi}{8}	& \frac{1}{2}\sqrt{2+\sqrt{2}}		& -\frac{1}{2}\sqrt{2-\sqrt{2}}		& -\sqrt{3+2\sqrt{2}} \\
120		& \frac{2 \pi}{3}	& \frac{\sqrt{3}}{2}				& -\frac{1}{2}						& -\sqrt{3} \\
126		& \frac{7 \pi}{10}	& \frac{1}{4}(\sqrt{5}+1)			& -\frac{1}{4}\sqrt{10-2\sqrt{5}}	& -\frac{1}{5}\sqrt{25+10\sqrt{5}}\\
135		& \frac{3 \pi}{4}	& \frac{1}{\sqrt{2}}				& -\frac{1}{\sqrt{2}}				& -1 \\
% MAOL page changes here
144		& \frac{4 \pi}{5}	&\frac{1}{4}\sqrt{10-2\sqrt{5}}		& -\frac{1}{4}(\sqrt{5}+1)			& -\sqrt{5-2\sqrt{5}} \\
150		& \frac{5 \pi}{6}	& \frac{1}{2}						& -\frac{\sqrt{3}}{2}				& -\frac{1}{\sqrt{3}} \\
157,5	& \frac{7\pi}{8}	& \frac{1}{2}\sqrt{2-\sqrt{2}}		& -\frac{1}{2}\sqrt{2+\sqrt{2}}		& 1-\sqrt{2} \\
162		& \frac{9 \pi}{10}	& \frac{1}{4}(\sqrt{5}-1)			& -\frac{1}{4}\sqrt{10+2\sqrt{5}}	& -\frac{1}{5}\sqrt{25-10\sqrt{5}} \\
165		& \frac{11 \pi}{12}	& \frac{1}{4}(\sqrt{6}-\sqrt{2})	& -\frac{1}{4}(\sqrt{6}+\sqrt{2})	& \sqrt{3}-2 \\
180		& \pi				& 0									& -1								& 0 \\
195		& \frac{13 \pi}{12}	& -\frac{1}{4}(\sqrt{6}-\sqrt{2})	& -\frac{1}{4}(\sqrt{6}+\sqrt{2})	& 2-\sqrt{3} \\
210		& \frac{7 \pi}{6}	& -\frac{1}{2}						& -\frac{\sqrt{3}}{2}				& \frac{1}{\sqrt{3}} \\
225		& \frac{5 \pi}{4}	& -\frac{1}{\sqrt{2}}				& -\frac{1}{\sqrt{2}}				& 1 \\
240		& \frac{4 \pi}{3}	& -\frac{\sqrt{3}}{2}				& -\frac{1}{2}						& \sqrt{3} \\
255		& \frac{17 \pi}{12}	& -\frac{1}{4}(\sqrt{6}+\sqrt{2})	& -\frac{1}{4}(\sqrt{6}-\sqrt{2})	& 2+\sqrt{3} \\
270		& \frac{3 \pi}{2}	& -1								& 0									& \text{ei määritelty} \\
285		& \frac{19 \pi}{12}	& -\frac{1}{4}(\sqrt{6}+\sqrt{2})	& \frac{1}{4}(\sqrt{6}-\sqrt{2})	& -(2+\sqrt{3}) \\
300		& \frac{5 \pi}{3}	& -\frac{\sqrt{3}}{2}				& \frac{1}{2}						& -\sqrt{3} \\
315		& \frac{7 \pi}{4}	& -\frac{1}{\sqrt{2}}				& \frac{1}{\sqrt{2}}				& -1 \\
330		& \frac{11 \pi}{6}	& -\frac{1}{2}						& \frac{\sqrt{3}}{2}				& -\frac{1}{\sqrt{3}} \\
345		& \frac{23 \pi}{12}	& -\frac{1}{4}(\sqrt{6}-\sqrt{2}	& \frac{1}{4}(\sqrt{6}+\sqrt{2})	& \sqrt{3}-2 \\
360		& 2 \pi				& 0									& 1									& 0 \\
\end{longtabu}
}
