\part{Kemia}

\section{Fysikaalinen kemia}


\begin{eqtable-full}{ASDF}
STP-olosuhteet ("NTP")			& T = 0 \degree \text{C}, \ P = 1 \text{ atm} \\
SATP-olosuhteet					& T = 25 \degree \text{C}, \ P = 1 \text{ bar} \\
termodynamiikan 1. pääsääntö	& \Delta U = q + w\\
suljettu reitti					& \oint dU = 0 \\
työ								& w = - \int_{V_i}^{V_f} P_{ext} dV \\
reversiibeli työ				& w = -nRT \ln \frac{V_2}{V_1} \\
lämpökapasiteetti				& C = \lim_{\Delta T \rightarrow 0} \frac{q}{T_f - T_i} = \frac{\dbar q}{dT} \\
lämpökapasiteetti vakiotilavuudessa	& C_V = \Big( \frac{\partial U}{\partial T} \Big)_V \\
entalpia						& H = U + PV \newline = q_P = C_P \Delta T \\
tilavuuden terminen laajenemiskerroin	& \beta = \frac{1}{V} \Big( \frac{\partial V}{\partial T} \Big)_P \\
isoterminen puristuvuus			& \kappa = - \frac{1}{V} \Big( \frac{\partial V}{\partial P} \Big)_T \\
ASDF							& C_P = C_V + TV \frac{\beta^2}{\kappa} \newline C_{P,m} = C_{V,m} + TV_m \frac{\beta^2}{\kappa} \\
	& \Delta P = \int_{T_i}^{T_f} \frac{\beta}{\kappa} dT - \int_{V_i}^{V_f} \frac{1}{\kappa V} dV \\
Joule-Thompson-kerroin			& \mu_{J-T} = \lim_{\Delta P \rightarrow 0} \Big( \frac{\Delta T}{\Delta P} \Big)_H = \Big( \frac{\partial T}{\partial P} \Big)_H \\
								& \Big( \frac{\partial H}{\partial P} \Big)_T = -C_P \mu_{J-T} \\
\end{eqtable-full}


\begin{eqtable-full}{Ideaalikaasu}
ideaalikaasun tilanyhtälö		& pV = N k_B T = nRT \\
ideaalikaasujen seos			& pV = \sigma n_i RT \\
								& P = \sigma \frac{n_i RT}{V} = \sigma p_i \\
lämpökapasiteettien yhteys		& C_P - C_V = nR \newline C_{P,m} - C_{V,m} = R \\
\end{eqtable-full}


\begin{eqtable-full}{Isobaarinen}
lämpövirta	& q_P = \int_{T_{sys,i}}^{T_{sys,f}} C_P^{sys} dT = - \int_{T_{surr,i}}^{T_{surr,f}} C_P^{surr} dT \\
\end{eqtable-full}


\begin{eqtable-full}{Isokoorinen}
sisäenergian muutos	& \Delta U = q_V = C_V \Delta T \\
\end{eqtable-full}


\begin{eqtable-full}{Reversiibeli ja adiabaattinen prosessi}
	& \gamma = \frac{C_{P,m}}{C_{V,m}} \\
	& \ln \frac{T_f}{T_i} = (1 - \gamma) \ln \frac{V_f}{V_i} \\
	& \frac{T_f}{T_i} = \Big( \frac{V_f}{V_i} \Big)^{(1-\gamma)} \\
	& P_i V_i^\gamma = P_f V_f^\gamma \\
\end{eqtable-full}


\begin{table}[!ht]
\centering
\caption{Termokemia}
\setlength{\extrarowheight}{10pt}
\begin{tabu} to \textwidth { X | >{$\displaystyle} X <{$} }
\hline
reaktioentalpia & \Delta H \degree_R = \sigma \nu_i \Delta H \degree_{f,i} \\
Kirchhoffin laki \newline (reaktioentalpian lämpötilariippuvuus)	& \Delta H \degree_{R,T_2} = \Delta H \degree_{R,T_1} + \int_{T_1}^{T_2} \Delta C_P dT \\
lämpökapasiteetin muutos reaktiossa	& \Delta C_{P,R}(T') = \sigma \nu_i C_{P,i}(T') \\
reaktioiden sisäenergioiden ja entalpioiden kokeellinen määrittäminen	& \Delta U_R = q_{sys} = -q_{surr} = -C_{calorimeter} \Delta T \\
reaktiossa vain kiinteitä aineita ja nesteitä	& \Delta U >> \Delta (PV) \Rightarrow \Delta H \approx \Delta U \\
reaktiossa myös kaasuja						& \Delta H_R = \Delta U_R + \Delta n_g RT \\
\end{tabu}
\end{table}

\begin{eqtable-full}{Entropia sekä termodynamiikan 2. ja 3. pääsääntö}
Carnot'n lämpökoneen tehokkuus	& \epsilon = \frac{| w_{cycle} |}{q_{ab}} = \frac{T_{hot} - T_{cold}}{T_{hot}} = 1 - \frac{T_{cold}}{T_{hot}} < 1 \\
irreversiibeli hyötysuhde	& \epsilon_{irreversiibeli} < \epsilon_{reversiibeli} < 1 \\
entropia	& dS = \frac{\dbar q_{reversible}}{T} \\

% reversiibeli vakiotilavuudessa	& \Delta S = \int \frac{\dbar q_{reversible}{T} = \int \frac{n C_{V,m} dT}{T} \approx n C_{V,m} \ln \frac{T_f}{T_i} \\
% reversiibeli vakiopaineessa	& \Delta S = \int \frac{\dbar q_{reversible}{T} = \int \frac{n C_{P,m} dT}{T} \approx n C_{P,m} \ln \frac{T_f}{T_i} \\
V ja T muuttuvat	& \Delta S = nR \ln \frac{V_f}{V_i} + n C_{V,m} \ln \frac{T_f}{T_i} \\
					& \Delta S = \int_{T_i}^{T_f} \frac{C_V}{T} dT + \int_{V_i}^{V_f} \frac{\beta}{\kappa} dV = C_V \ln \frac{T_f}{T_i} + \frac{\beta}{\kappa} (V_f - V_i) \\
P ja T muuttuvat	& \Delta S = -nR \ln \frac{P_f}{P_i} + n C_{P,m} \ln \frac{T_f}{T_i} \\
faasimuutos	& \Delta S_{trans} = \frac{\Delta H_{trans}}{T_{trans}} \\
\end{eqtable-full}


\clearpage

\section{Taulukoita}

\begin{table}
\centering
\caption{Termodynaamisia arvoja \cite[A19-A22]{Zumdahl} }
\begin{tabular}{| l | >{$} l <{$} | >{$} l <{$} | >{$} l <{$} |}
\hline
Aine ja olomuoto
& \frac{ \Delta H_f^{\degree} }{ \text{kJ/mol} }
& \frac{ \Delta G_f^{\degree} }{ \text{kJ/mol} }
& \frac{ \Delta S^{\degree} }{\text{J/K} \cdot \text{mol} } \\
\hline
alumiini &&& \\
Al(s)			& 0		& 0		& 28 \\
Al$_2$O$_3$(s)	& -1676	& -1582	& 51 \\
Al(OH)$_3$(s)	& -1277 && \\
AlCl$_3$(s)		& -704	& -629	& 111 \\
&&& \\
barium &&& \\
Ba(s)			& 0		& 0		& 67 \\
BaCO$_3$(s)		& -1219	& -1139	& 112 \\
BaO(s)			& -582	& -552	& 70 \\
Ba(OH)$_2$(s)	& -946 && \\
WORK IN PROGRESS	&&& \\
\hline
\end{tabular}
\end{table}

